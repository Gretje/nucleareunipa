\chapter{Forze Nucleari}

Un \marginnote{18-03-1998}  \textit{nucleo atomico} è un aggregato di protoni 
e neutroni.

L'esistenza di nuclei stabili implica l'esistenza di forze attrattive, che sono
conseguenza  dell'energia di legame e del difetto di massa. Queste forze devono
vincere la repulsione elettromagnetica; il raggio caratteristico di questa fase
è $10^{-13} cm$, che si ricava dagli esperimenti sulla diffusione. Il modello a
goccia di liquido assume che la forza attrattiva sia indipendente dalla natura
del nucleone (sia questo un protone o un neutrone). Questa assunzione implicita
è confermata dal successo della formula per la massa che si ricava da questa
ipotesi, ma anche altri fatti sperimentali ne confermano la validità. Uno di
questi fatti si basa sui nuclei isobari speculari (si ottengono scambiando i
numeri di protoni e neutroni). La massa di questi nuclei è la stessa a meno del
termine relativo all'energia coulombiana. 

Consideriamo ad esempio due nuclei
\begin{align*}
	( Z_{1}& = Z \ ;  \quad N_{1} = Z+1 ) &  ( Z_{2}& = Z+1 \ ;  \quad 
N_{2} = Z) 
\end{align*}
una coppia di questo tipo presenta una differenza di massa 
\begin{equation*}
	 \Delta M = \Delta M_{c} 
\end{equation*}
dove $\Delta M_{c}$ è la differenza dovuta all'energia coulombiana.

\begin{equation}
\begin{split}
	\Delta M_{c} &= \dfrac{3}{5} \dfrac{e^2 (z_{2})^2}{c^2(r_{0})^2 
A^{1/3}} - \dfrac{3}{5} \dfrac{e^2 				
(z_{1})^2}{c^2(r_{0})^2 A^{1/3}} = \dfrac{3}{5} \dfrac{e^2}{c^2(r_{0})^2 
A^{1/3}} 				\bigl[ (z+1)^2-z^2 \bigr ] \\
			&= \dfrac{3}{5} \dfrac{e^2}{c^2(r_{0})^2 A^{1/3}} (2z+1)
\end{split}
\end{equation}

Un altro fatto sperimentale è la strettissima somiglianza di processi di
diffusione $n-p^{+}$ e $p^{+}-p^{+}$, prescindendo dagli effetti di natura
elettrostatica. Un altro fatto è l'apparente dilemma  sperimentale dovuto alla
non esistenza di stati legati $n-n$ o $p-p$.
Per capire questo consideriamo due \textit{fermioni} identici con spin $1/2$,
secondo il \textit{principio di Pauli} questi non possono trovarsi in uno stato
legato di tripletto. Può essere comunque che i due fermioni si trovino in uno
stato legato di singoletto. Non si capisce quindi perché non esistano stati
legati fra due protoni o due neutroni. Il deutone esiste però solo nel suo 
stato
fondamentale di tripletto, cioè la forza fra $n-p$ è tale da non consentire 
uno
stato legato di singoletto; quindi se si assume che la forza che agisce fra due
protoni o neutroni è la stessa si spiega il dilemma.
Sulla base di questi fatti \textit{Heisenberg} formulò quindi una teoria 
secondo
cui la forza fra i nucleoni non dipendeva dalla carica.
Il concetto di indipendenza dalla carica si può esprimere dicendo che la forza
nucleare non distingue fra nucleoni carichi o no. Quindi, se non si considerano
effetti elettromagnetici, neutrone e protone rappresentano una stessa particella
(il nucleone). La differenza di massa fra $n$ e $p$ ha proprio origine
elettromagnetica e non ci sarebbe se le due particelle avessero la stessa
carica. Il fatto che gli effetti elettromagnetici siano presenti implica che
l'indipendenza dalla carica è una approssimazione, in quanto solo la forza
nucleare non dipende dalla carica.

Neutrone e protone possono essere considerati in questo modello come due
differenti stati di carica di una stessa particella. A questo proposito
Heisenberg introdusse il formalismo dello \textit{spin isotopico}.
Quindi una particella ha un ulteriore grado di libertà oltre quelli spaziali e
di spin. Questo grado di libertà è espresso da una variabile interna che può
assumere solo due valori, quindi è una variabile dicotomica (come lo spin con
$s=1/2$). Per lo spin si può usare l'espressione $\dfrac{\sigma}{2}$ con
$\sigma=\pm 1$. Per lo stato di carica si usa l'espressione: $\tau$ con $\tau =
\pm 1$ dove $\tau= +1 \rightarrow \text{protone}$, e $\tau= -1 \rightarrow
\text{neutrone}$.
 $\tau$ si dice \textit{variabile di spin isotopico} o  \textit{isospin}.

Al protone si può associare una funzione d'onda di isospin $\pi(\tau)$ dove
$\pi(+1) = 1$ e $\pi(-1) = 0$. Al neutrone si associa, analogamente, la funzione
d'onda di isospin $\nu(\tau)$ dove $\nu(+1) = 0$ e $\nu(-1) = 1$.

Queste due funzioni sono una base dello spazio delle funzioni d'onda relative
allo spin isotopico; queste funzioni sono di funzioni di $\tau$. Le funzioni
d'onda $\pi$ e $\nu$ si possono rappresentare come due vettori colonna:

\begin{equation*}
\pi(\tau) =
\begin{pmatrix}
\pi(+1) \\
\pi(-1)
\end{pmatrix} =
\begin{pmatrix}
1 \\
0
\end{pmatrix} 
\qquad e \qquad
\nu(\tau) = 
\begin{pmatrix}
\nu(+1) \\
\nu(-1)
\end{pmatrix} =
\begin{pmatrix}
0 \\
1
\end{pmatrix} 
\end{equation*}

Questi due vettori sono una base ortonormale di uno spazio vettoriale a due
dimensioni, purché i due bra siano definiti dai vettori riga.
\breaknote

Lo \marginnote{18-03-1998} spin ordinario è rappresentato, in unità di 
$\hbar$, dall'operatore

\begin{equation}
\vec{S}=\dfrac{1}{2} \vec{\sigma}
\end{equation}
e si trasforma come un vettore per rotazioni del sistema fisico. Le componenti 
sono:
\begin{align*}
  S_{x} = \dfrac{\sigma_{x}}{2}, \qquad S_{y} =\dfrac{\sigma_y}{2}, \qquad 
S_{z}=\dfrac{\sigma_z}{2}
\end{align*}
dove $\sigma_{x}$, $\sigma_{y}$ e $\sigma_{z}$ sono le \textit{matrici di 
Pauli}. 

Queste matrici agiscono nello spazio a due dimensioni relativo allo spin $1/2$.
La rappresentazione di queste matrici è fatta rispetto alla base costituita dai
due autostati relativi alla direzione $z$. 

Si possono, quindi, definire per matrici formalmente identiche a quelle di Pauli
\begin{equation*}
\tau_{1}=
\begin{pmatrix}
0 & 1 \\
1 & 0
\end{pmatrix}
\qquad
\tau_{2}=
\begin{pmatrix}
0 & -i \\
i & 0
\end{pmatrix}
\qquad
\tau_{3}=
\begin{pmatrix}
1 & 0 \\
0 & -1
\end{pmatrix}
\end{equation*}
queste agiscono nello spazio a due dimensioni relativo agli stati di isospin.

La matrice $\tau_{3}$ rappresenta la variabile di isospin $\tau$ sottoforma di
operatore. Infatti si ha:
\begin{align*}
\tau_{3} \Ket{ p^+} & =\ \Ket{p^+} \qquad	\tau_{3}\ket {n} = -\ket {n}
\end{align*}

Possiamo considerare $\tau_1$,$\tau_2$ e $\tau_3$ come componenti ortogonali di
un unico operatore vettoriale, relativo ad uno spazio euclideo astratto. In
questo spazio è lecito, quindi, definire l'operatore:
\begin{empheq}[box=%
\fbox]{equation}
\vec{T}=\dfrac{\vec{\tau}}{2}
\end{empheq}

Questo spazio tridimensionale così introdotto si dice \textit{spazio dello spin
isotopico}. Questo non è lo spazio degli stati di spin isotopico (che ha
dimensione 2). Ovviamente per le componenti di $T$ e $\tau$ valgono le note
regole di commutazione. Partendo dai due operatori $T_1$ e $T_2$ si possono
definire i due operatori

\begin{align}
T_{+} \equiv (T_1+ iT_2)	\qquad	T_{-} \equiv (T_1 - iT_2)
\end{align}

\begin{equation}
T_{+} =
\begin{pmatrix}
0 & 1 \\
0 & 0
\end{pmatrix} 
\qquad
T_{-} =
\begin{pmatrix}
0 & 0 \\
1 & 0
\end{pmatrix} 
\end{equation}

Si verifica facilmente che valgono le seguenti relazioni
\begin{equation*}
T_{+}
\begin{pmatrix}
1 \\
0
\end{pmatrix} 
= 0
\qquad
T_{+}
\begin{pmatrix}
0 \\
1
\end{pmatrix} 
=
\begin{pmatrix}
1 \\
0
\end{pmatrix}
\qquad
T_{-}
\begin{pmatrix}
1 \\
0
\end{pmatrix} 
=
\begin{pmatrix}
0 \\
1
\end{pmatrix}
\qquad
T_{-}
\begin{pmatrix}
0 \\
1
\end{pmatrix} 
= 0
\end{equation*}

Equivalentemente queste relazioni si possono scrivere sotto la forma:

\begin{equation*}
T_{+} \Ket n  = \Ket {p^+} \ ; \quad T_+ \Ket {p^+} = 0 \ ; \quad T_- \Ket{ 
p^+} = \Ket n \ ; \quad T_- \Ket n  = 0
\end{equation*}

\begin{equation*}
T_3 \Ket{ p^+} = \dfrac{1}{2} \Ket{ p^+} \ ; \quad T_3 \Ket n = - \dfrac{1}{2} 
\Ket n
\end{equation*}


Ovviamente $\Ket{ p^+}$ e $\Ket n$ sono autostati di $T_3$ con autovalori
$+\dfrac{1}{2}$ e $- \dfrac{1}{2}$.
Per il significato noto degli stati $\Ket{ p^+}$ e $\Ket n $, si intuisce che
deve esistere una relazione fra l'operatore  $T_3$ e l'operatore di carica.
L'\textit{operatore di carica} $Q$ si può definire dalla relazione:

\begin{equation}
Q \Ket{ p^+} = (+1) \Ket{ p^+}  \qquad Q \Ket n = 0
\end{equation}
(in unità di carica protonica). Da queste si deduce che

\begin{empheq}[box=%
\fbox]{equation}
Q = T_3 + \dfrac{B}{2}
\end{empheq}
dove $B \Ket {p^+} = \Ket {p^+}$ e $ B \Ket n = \Ket n $ (\textit{B = operatore
unità}). B si dice \textit{operatore di numero barionico}. B coincide con
l'operatore identità solo perché stiamo analizzando solo lo spazio relativo ai
nucleoni, se si estende questo spazio B ha anche autovalori $-1$. Se si ha un
sistema di nucleoni l'operatore totale di spin isotopico è:

\begin{equation}
\vec{T}= \vec{T}_{(1)} + \vec{T}_{(2)} \qquad	\text{ come L}
\end{equation}
dove $T_{(1)}$ e $T_{(2)}$ sono operatori che agiscono su spazi differenti. Si
ha che:
\begin{equation*}
{\vec{T}_{(1)}}^2 \Ket 1  \ =  \ t_{(1)} \bigl( t_{(1)} + 1\bigr) \Ket 1   
\qquad {\vec{T}_{(2)}}^2 \Ket 2 \ = \ t_{(2)} \bigl ( t_{(2)} + 1\bigr ) \Ket 2 
 \qquad \text{come $L^2$}
\end{equation*}
dove $t_{(1)} = t_{(2)} = 1/2$. Possiamo definire l'operatore ${\vec{T}}^2$ 
\begin{equation}
\vec{T}^2= {\vec{T}_{(1)}}^2 + {\vec{T}_{(2)}}^2 + 2 \ \vec{T}_{(1)}  
\vec{T}_{(2)}.
\end{equation}

Un generico autovalore di $\vec{T}^2$ si può scrivere sotto la forma $ t \ 
\bigl
( t + 1 \bigr ) $. Deve essere sempre verificata la diseguaglianza triangolare:
\begin{equation*}
| t_{1} - t_{2} | \le t \le t_1 + t_2
\end{equation*}
e dal momento che $t$ può variare solo per salti unitari si ha:
\begin{equation*}
t = t_1+ t_2 \ , t_1 + t_2 -1 \ , \dots \ , | t_1 - t_2 |
\end{equation*}
Nel caso in cui $t_1 = t_2 = \dfrac{1}{2}$ si hanno solo i due valori $ t = 1 $
e $ t = 0 $.

Se $t = 1$ il sistema dei due nucleoni si trova in uno stato di isospin di
tripletto, se $ t = 0$ di singoletto. Diciamo che il singolo nucleone si trova
in uno stato di isospin di doppietto, cioè che gli stati di neutroni e di
protoni costituiscono un doppietto di isospin. Quindi un doppietto di isospin è
dato dalla base di autofunzioni $\pi(\tau)$ e $\nu(\tau)$. Due doppietti
identici di isospin sono costituiti dalle basi:

\begin{equation*}
\bigl\{ \pi (\tau_{(1)}) \ , \nu(\tau_{(1)}) \bigr\} \quad ;	\quad 
\bigl\{\pi(\tau_{(2)}) \ , \nu(\tau_{(2)}) \bigr\} 
\end{equation*}
Per semplicità poniamo $\tau_{(1)} = 1$ e $\tau_{(2)} = 2$. Le basi quindi 
sono:
\begin{equation*}
\bigl\{ \pi (1), \nu(1) \bigr\} \qquad; \qquad \bigl\{ \pi(2), \nu(2) \bigr\}
\end{equation*}
La base dello spazio relativo agli stati di spin di singoletto o di tripletto 
è: 
$\pi(1) \pi(2)$, $\nu(1) \nu (2)$ , $\pi(1) \nu (2)$ , $\nu (1) \pi (2)$.

Le autofunzioni di singoletto e di tripletto sono:
\[
\left.
\begin{aligned}
\pi (1) \pi (2) 	\qquad (t_3 = + 1) \\
\dfrac{1}{\sqrt{2}} \bigl [\pi(1) \nu(2) + \nu (1) \pi(2) \bigr] 	\qquad 
(t_3 = 0) \\
\nu (1) \nu (2) 	\qquad (t_3= -1) 
\end{aligned}
\right\}
\quad
\text{t = 1 \quad tripletto}
\]
\begin{equation*}
\dfrac{1}{\sqrt{2}} \bigl [\pi(1) \nu(2) - \nu (1) \pi(2) \bigr] 	\qquad 
(t_3 = 0) 	\qquad t = 0 \quad  \text{singoletto}
\end{equation*} 

Per indicare simbolicamente la composizione di due doppietti di isospin si usa
la seguente notazione:
\begin{align*}
2 &\equiv	\bigl\{ \pi(\tau) \ , \nu(\tau) \bigr\} \\
2 \times 2 &= 3 + 1
\end{align*}

\breaknote

Abbiamo \marginnote{23-03-1998} visto che due nucleoni si comportano come
fermioni per quanto riguarda le forze nucleari, quindi questi devono sottostare
al \textit{principio di esclusione di Pauli}.
La loro funzione d'onda deve essere antisimmetrica, sia nella parte radiale che
in quella di isospin, che in quella di spin (deve essere antisimmetrico tutto il
prodotto di questi termini).

\begin{equation*}
\psi = \psi_r  \ \psi_\sigma \ \psi_\tau
\end{equation*}

Nel caso in cui si hanno due protoni o due neutroni la parte di isospin risulta
simmetrica:

\begin{align*}
\psi_\tau &= \pi(1) \ \pi(2) = \pi(2) \ \pi(1) \\
\psi_\tau &= \nu(1) \ \nu(2) = \nu(2) \ \nu(1)
\end{align*}
quindi la rimanente parte $\psi_r \ \psi_\sigma$ deve risultare antisimmetrica,
in accordo con la teoria già formulata.

Consideriamo la parte della funzione $\psi_r \ \psi_\sigma$; si ha che per lo
scambio degli indici:
\begin{equation*}
\psi_r \ \psi_\sigma \xrightarrow[1\leftrightarrow2]{} (-1)^l (-1)^{s+1} \ 
\psi_r \ \psi_\sigma
\end{equation*}
abbiamo detto che questa deve essere antisimmetrica, quindi:
\begin{equation*}
  (-1)^l (-1)^{s+1} = -1 \quad \Longrightarrow \quad l+s+1 = \text{dispari}
\end{equation*}
da questa condizione si estraggono due sottocasi:
\begin{equation*}
l+s+1=
\begin{cases}
s=1,      \implies & \text{l \ = \ dispari,} \\
s=0, 	     \implies& \text{l \ = \ pari.}
\end{cases}
\end{equation*}

Nel caso generico in cui i due nucleoni possono anche essere un protone e un
neutrone dobbiamo considerare tutta la funzione d'onda. 
Per lo scambio degli indici si ha la seguente trasformazione:

\begin{equation*}
\psi_r \ \psi_\sigma \ \psi_\tau  \xrightarrow[1\leftrightarrow2]{} (-1)^l 
(-1)^{s+1} (-1)^{t +1}  \ \psi_r \ \psi_\sigma \ \psi_\tau
\end{equation*}
quindi deve essere verificata la condizione:
\begin{equation*}
(-1)^l (-1)^{s+1} (-1)^{\tau +1} \ = \ -1
\end{equation*}

La regola di selezione generalizzata è:
\begin{empheq}[box=%
\fbox]{equation}
l + s + t = \text{dispari}
\end{empheq}

Questa regola si applica al deutone (=stato legato di un protone e di un
neutrone). Questo sistema è caratterizzato dai numeri quantici $l \ = \ 0$ e $ 
s
\ = \ 1$. Applicando la regola di selezione si deduce che $t$ deve essere pari,
ma dal momento che $t$ può assumere solo valori 0 e 1 si deduce che deve essere
$t \ = \ 0$ per il deutone.

\chapter{Conservazione dello spin isotopico} 
Abbiamo visto che per un nucleone esiste una ben determinata relazione fra la
carica elettrica e la terza componente dello spin isotopico. Questa relazione in
termini di operatori è:

\begin{equation*}
Q = T_3 + \dfrac{B}{2}
\end{equation*}

Viste le proprietà dell'operatore $\vec{T}$ si può porre la analogia:

\begin{equation*}
T_3 \longleftrightarrow J_z 	\qquad \vec{T}^2 \longleftrightarrow \vec{J}^2
\end{equation*}

Se si opera una rotazione dello spazio di isospin deve quindi essere:

\begin{equation*}
  \delta \mean{T_3} \ne 0 \qquad \delta \mean{\vec{T}^2} = 0
\end{equation*}
questa trasformazione trasforma lo stato iniziale del nucleone $\Ket{N}$ in uno
stato $\Ket{N'}$, che si può scrivere come:
\begin{equation*}
  \Ket{N} \Rightarrow \Ket{N'} = C'_1 \Ket{p^+} + C'_2 \Ket{n} \qquad 
\text{con} \ \abs{C'_1}^2+\abs{C'_2}^2=1
\end{equation*}

Gli stati $\Ket{p^+}$ e $\Ket{n}$ hanno lo stesso numero barionico (B =1),
quindi per lo stato $\Ket{N'}$ si avrà sempre lo stesso numero barionico:

\begin{equation*}
\delta \mean{B} = 0
\end{equation*}
cioè il numero barionico si comporta come uno scalare nello spazio di isospin.
Quindi si può scrivere la relazione:

\begin{equation*}
  \delta \mean{Q} = \delta \mean{T_3}
\end{equation*}

In base a quanto detto le trasformazioni
\begin{equation*}
n  \longleftrightarrow p^+ \qquad p^+  \longleftrightarrow n
\end{equation*}
possono considerarsi come particolari rotazioni nello spazio di isospin. 

L'indipendenza della carica dalle forze nucleari si può esprimere
matematicamente come invarianza per rotazioni nello spazio di isospin.

Se chiamiamo interazione forte quella responsabile del legame nucleone e se
$H_F$ è l'Hamiltoniana di interazione si può, quindi, dimostrare che $H_F$ non
cambia per rotazioni nello spazio di isospin. 

Se indichiamo con $R_\tau$ la trasformazione indotta dalla rotazione nello
spazio degli stati si avrà che:

\begin{equation*}
\Ket{N'} = R_\tau \Ket{N}
\end{equation*}
quindi deve risultare:

\begin{equation*}
R_\tau H_F = H_F R_\tau \Longrightarrow \fbox{$\bigl[R_\tau \ , H_F \bigr] = 0$ 
}
\end{equation*}

Se assumiamo che il $n$ e il $p^+$ abbiano la stessa massa si può scrivere più
in generale:

\begin{equation*}
  \bigl[R_\tau \ , H \bigr] = 0 \qquad \text{dove } H = H_0 + H_F
\end{equation*}

L'invarianza per rotazioni nello spazio di isospin implicherà la conservazione 
dello spin isotopico (in analogia con $J$). Si può scrivere: 

\begin{equation*}
R_\tau = e^{-i \theta \hat{n} \vec{T}}
\end{equation*}
e per una rotazione infinitesima:

\begin{equation*}
R_\tau= 1 - i \ d\theta \ \hat{n} \ \vec{T}
\end{equation*}

Da questo segue che:

\begin{equation*}
\fbox{$\bigl[R_\tau \ , H \bigr] = 0 \Longleftrightarrow \bigl[\vec{T} \ , H 
\bigr] = 0 \Longleftrightarrow \vec{T}$ si  conserva}
\end{equation*}

Quindi il principio di indipendenza della carica è equivalente alla condizione 
$\bigl[\vec{T} \ , H \bigr] = 0$. Si deve notare che questo principio di 
conservazione dello spin isotopico non ha validità generale, ma solo quando si 
considerano solamente le interazioni forti. Non appena si considerano 
interazioni elettromagnetiche questa conservazione non vale più.

Sperimentalmente per i processi che coinvolgono solo interazioni forti non si 
verifica mai che:

\begin{equation*}
  \mean{\vec{T}^2}_\text{fin} \ne \mean{\vec{T}^2}_\text{in} \qquad
  \mean{\vec{T}_3}_\text{fin} \ne \mean{\vec{T}_3}_\text{in}
\end{equation*}
questo fatto sperimentale è una conferma dell'indipendenza dalla carica delle 
interazioni forti.

\part{Elementi di Fisica Subnucleare}
\chapter{Teoria di Yukawa del mesone $\pi$}

Il primo tentativo di una interpretazione dinamica dell'interazione forte fu
fatta da Yukawa. Il suo intento era quello di fornire una interpretazione
autoconsistente dell'interazione forte sul modello della interpretazione
quantistica dei processi elettromagnetici. Il presupposto fondamentale della
teoria di Yukawa è il concetto relativistico del \textit{campo}.

Secondo la teoria della relatività una interazione non può propagarsi
istantaneamente, questo significa che non è più concepibile un'interazione a
distanza tra due sistemi con un trasferimento istantaneo d'energia. Quindi deve
esistere una terza entità fisica che abbia la funzione di trasmettere l'energia
emessa e di mediare l'interazione; questa terza entità si identifica con il
campo.

In realtà quello che succede è che ciascuno dei due sistemi interagisce
localmente con il campo generato dall'altro sistema. Se questo concetto viene
trasferito nell'abito della meccanica quantistica si arriva ad un modello
quantizzato del campo, quindi si devono attribuire al campo proprietà sia
corpuscolari che ondulatorie.

Secondo la visione quantistica, l'interazione non locale tra due sistemi sarebbe
mediata da singoli quanti che vengono localmente emessi ed assorbiti da ciascuno
dei due sistemi. Tutto questo si applica, in particolare, al campo
elettromagnetico il cui quanto fondamentale è il \textit{fotone}. L'interazione
fra due cariche viene interpretata in termini di fotoni emessi ed assorbiti
dalle due cariche.

Yukawa intuì che un modello analogo dovesse in qualche modo valere anche per
l'interazione forte fra due nucleoni. Apparentemente, infatti, tale interazione
cominciava a farsi sentire ad una distanza di $10^{-13}$ cm, e quindi dato che
non si poteva propagare da un nucleone all'altro doveva essere mediata da un
quanto analogo al fotone. A questo ipotetico quanto Yukawa diede il nome
di \textit{mesone $\pi$}. Esiste, comunque, una
proprietà fondamentale che distingue le due interazioni e cioè che 
l'interazione
forte è a corto raggio ( a differenza di quella elettromagnetica ). Questa
differenza ha profondi implicazioni di natura fisica. Per capire quali esse
siano conviene partire dall'equazione classica di propagazione del potenziale
elettromagnetico nel vuoto:

\begin{equation}
\label{potenziale_elettromag}
\nabla^2 V - \dfrac{1}{c^2} \dfrac{\partial^2 V}{\partial t^2} = 0
\end{equation}

Questa ha un unica soluzione statica del tipo $V \propto 1/ r$, questa soluzione
implica che l'interazione elettromagnetica ha raggio infinito. Ovviamente per un
potenziale a corto raggio, non può valere la stessa equazione. Si possono
interpretare in maniera quantistica gli operatori che compaiono nella
\ref{potenziale_elettromag}:

\begin{align*}
\nabla^2 &\equiv - \dfrac{1}{\hbar^2} \Bigl(-i \hbar \vec{\nabla} \Bigr) \Bigl( 
-i \hbar \vec{\nabla} \Bigr) \\
\dfrac{\partial^2}{\partial t^2} &\equiv - \dfrac{1}{\hbar^2} \Biggl( i \hbar 
\dfrac{\partial}{\partial t} \Biggr) \Biggl( i \hbar \dfrac{\partial}{\partial 
t} \Biggr) 
\end{align*}
dove $-i \hbar \vec{\nabla}$ è l'operatore quantità di moto 

Consideriamo ora la soluzione del tipo:
\begin{gather*}
V \propto e^{i \bigl(kx - \omega t \bigr)} \\
\nabla^2 V - \dfrac{1}{c^2} \dfrac{\partial^2}{\partial t^2} V = \Biggl (-K^2 + 
\dfrac{\omega^2}{c^2} \Biggr) V = \dfrac{1}{\hbar^2} \Biggl(\dfrac{E^2}{c^2} - 
p^2 \Biggr) V = 0
\end{gather*}

dove si è posto $E=\hbar \omega$ e $p=\hbar K$. Deve, quindi, essere: 

\begin{equation*}
\dfrac{E^2}{c^2} - p^2 = 0
\end{equation*}

$E$ e $p$ sono l'energia e la quantità di moto del singolo fotone di massa 
nulla
associato all'onda piana. Per una particella con massa $m$ non nulla si deve
avere:

\begin{equation*}
\dfrac{E^2}{c^2} - p^2 = m^2 c^2
\end{equation*}

Sostituendo queste espressione si può generalizzare l'equazione d'onda nella 
forma:

\begin{equation*}
\nabla^2 V - \dfrac{1}{c^2} \dfrac{\partial^2}{\partial t^2} V = \mu^2 V \qquad 
\qquad
\mu^2 = \dfrac{m^2 c^2}{\hbar^2} \Rightarrow \fbox{$\mu=\dfrac{mc}{\hbar}$}
\end{equation*}

Si può dimostrare che esiste la soluzione statica del tipo:

\begin{equation*}
\fbox{$V \propto \dfrac{1}{r} \ e^{-\mu r}$}
\end{equation*}

Questo potenziale esprime l'interazione entro un raggio caratteristico
$r_0=1/\mu$, per distanze maggiori di $r_0$ l'interazione diviene molto piccola.
Quindi il campo di interazione forte non può essere associato ad un quanto di
massa nulla, ma dovrà essere associata ad un quanto con massa tale da 
soddisfare
la condizione:

\begin{equation*}
r_0=\dfrac{\hbar}{mc} \sim 10^{-13} cm 
\end{equation*}

Secondo questo modello ogni nucleone genera un potenziale statico:

\begin{equation*}
\fbox{$V_{_F}(r) = - g_{_F} \dfrac{e^{-\mu r}}{r}$}
\end{equation*}
con $g_{_F} =$ \textit{costante
di accoppiamento di Yukawa}. Il segno meno implica un potenziale attrattivo. Il
valore di $g_F$ è determinato empiricamente ed il suo ruolo è analogo a quello
della carica elettrica.

L'energia associata all'interazione di due cariche elementari è data da:
\begin{equation*}
e V(r) = \dfrac{e^2}{r}
\end{equation*}
mentre l'intensità dell'interazione è determinata dalla \textit{costante di
struttura fine}:
\begin{equation*}
\alpha = \dfrac{e^2}{\hbar c} \simeq \dfrac{1}{137} \implies e V(r) = \alpha 
\Biggl(\dfrac{\hbar c}{r}\Biggr)
\end{equation*}
Nel caso di due cariche forti, l'energia analoga a quella elettrostatica è:
\begin{equation*}
g_{_{F}}V_{_F}(r) = - {g^2_{_{F}}} \ \dfrac{e^{-\mu r}}{r}
\end{equation*}
Se si prescinde dal fattore esponenziale, l'intensità dell'interazione forte 
può
essere espressa in termini della costante dimensionale:
\begin{equation*}
\dfrac{{{g}_{_F}}^2}{\hbar c} \simeq 14.5
\end{equation*}
Quindi, l'intensità forte è quasi $2000$ volte più intensa di quella
elettromagnetica. Sperimentalmente si ricava per il parametro $\mu$ il valore:
\begin{equation*}
\mu = \dfrac{1}{r_0} \simeq 0.7 	\times 10^{13} cm
\end{equation*}
Quindi la massa del mesone $\pi$ dovrebbe essere:
\begin{align*}
m_{\Pi} = \dfrac{\hbar \mu}{c} &\simeq \dfrac{(6.58 \times 10^{-22} \text{MeV
sec}) (0.7 \times 10^{13} \text{cm}^{-1}) (3 \times 10^{10} \text{cm/sec} 
)}{c^2} \\
\Rightarrow m_{\pi} &\simeq 138 \text{MeV}/c^2
\end{align*}

\breaknote

Abbiamo \marginnote{27-3-1998} visto che, in base alla teoria di Yukawa, il
mesone $\pi$ ha una massa pari a circa:

\begin{equation*}
  m_{\pi} \simeq 138 \text{MeV}/c^2
\end{equation*}

Supponiamo che il singolo nucleone emetta o assorba un mesone $\pi$ dando luogo
ad uno dei processi di Yukawa:

\begin{equation*}
  \underbrace{N \to N' + \pi}_\text{Emissione} \qquad \qquad \underbrace{N' +
  \pi \to N}_\text{Assorbimento}
\end{equation*}
Analizziamo ad esempio l'emissione. Questo processo non può essere compatibile
con la conservazione dell'energia, infatti l'energia disponibile, $Q$, deve
essere tale che\footnote{la differenza di massa fra un neutrone e un protone
è $\simeq 1\text{MeV}/c^2$}:

\begin{equation*} 
  m_N c^2 = m_{N'} c^2 + Q \qquad Q_\text{max} \simeq 1 \text{MeV}
\end{equation*}

Se un mesone $\pi$ viene prima emesso da un nucleone e poi assorbito da un altro
nucleone, l'energia del sistema dei due nucleoni si conserva. L'energia in più
creata nell'emissione viene poi distrutta nell'assorbimento del mesone $\pi$. Si
conclude, quindi, che il processo di scambio di un mesone $\pi$ fra due nucleoni
è un processo virtuale e quindi non osservabile; cioè questo processo avviene
solo all'interno dell'indeterminazione dell'energia dei nucleoni conseguente al
principio d'indeterminazione di Heisenberg.

Consideriamo l'equazione classica del potenziale a corto raggio:

\begin{equation*}
V(r) \propto \dfrac{1}{r} e^{-r/r_0}
\end{equation*}
dove $r_0 =$ è la \textit{distanza efficace}.

Nel caso dell'interazione forte si ha:

\begin{equation*}
r_0 = r_{0_{F}} = \dfrac{\hbar}{m_{\pi} c}
\end{equation*}
La natura virtuale di questo processo di scambio chiarisce ancora meglio questa 
relazione. 
Sia $\Delta t$ l'intervallo di tempo che in media intercorre fra emissione e
assorbimento. In questo intervallo, l'energia dei nucleoni ha una
indeterminazione:

\begin{equation*}
\Delta E = \dfrac{\hbar}{\Delta t}
\end{equation*}
Affinché avvenga questo processo di scambio deve essere:
\begin{equation*}
\Delta E > m_{\Pi} c^2  \implies \Delta t = \dfrac{\hbar}{\Delta E} < 
\dfrac{\hbar}{m_{\pi}c^2}
\end{equation*}

In questo intervallo il mesone può percorrere al massimo la distanza media:
\begin{equation*}
\bar{d} \bigl(v_\text{max} \bigr) = v_\text{max} \Delta t = c \Delta t < 
\dfrac{\hbar}{m_{\pi}c}
\end{equation*}
quindi lo scambio di un mesone $\pi$ fra due nucleoni può risultare virtuale
solo se la distanza media fra i due nucleoni è tale che:

\begin{equation*}
\bar{d} < \dfrac{\hbar}{m_{\pi}c} = r_{0_{F}}
\end{equation*}

Nel modello di Yukawa gioca un ruolo essenziale il \textit{principio di 
indipendenza della carica}.
In base a questo principio, il mesone $\pi$ può trasportare o una carica 
nulla, o una carica $e$. 
Se all'inizio abbiamo un protone e un neutrone possono avvenire due processi:

\[
\left.
\begin{aligned}
p^+ + n &\implies p^+ + \pi^0 + n \implies p^+ + n \\
p^+ + n &\implies n + \pi^+ + n \implies n + p^+
\end{aligned}
\right\}
\quad
\text{mesone $\pi$ emesso dal protone}
\]

Se invece è il neutrone ad emettere può avvenire anche il processo:

\begin{equation*}
n + p^+ \implies p^+ + \pi^- + p^+ \implies p^+ + n 
\end{equation*}
Quindi il mesone $\pi$ può esistere con tre varietà di carica. Questi sono un
tripletto di isospin cioè $\tau^{(\pi)} = 1 =$ \textit{numero quantico di 
isospin
del mesone $\pi$}. 

Per verificare questa basta applicare il principio di conservazione dello spin
isotopico ad uno qualsiasi dei processi di sopra. Per quanto riguarda la terza
componente dello spin isotopico si ha che:
\begin{equation*}
  \tau_{3}^{(\pi^+)} = +1 \qquad \tau_{3}^{(\pi^0)} = 0 \qquad 
\tau_{3}^{(\pi^-)} = -1
\end{equation*}

Questi autovalori coincidono con quelli dell'operatore $Q$. Quindi ricaviamo le
seguenti relazioni
\begin{equation*}
Q\Ket{\pi} = \Biggl( T_3 + \dfrac{B}{2} \Biggr) \Ket{\pi} = T_3 \Ket{\pi} 
\implies B \Ket{\pi} = 0
\end{equation*}
il numero barionico del mesone $\pi$ è zero. 

Il processo di scambio del mesone $\pi$ si può studiare con la \textit{teoria
delle perturbazioni} come un processo del secondo ordine, cioè un processo con
uno stato intermedio fra lo stato iniziale e quello finale. Consideriamo ad
esempio il processo:
\begin{equation*}
p^+ + p^+ \implies p^+ + \pi^0 + p^+ \implies p^+ + p^+
\end{equation*}

Gli stati da considerare sono:
\[
\left.
\begin{aligned}
\Ket{i}= \Ket{p^+ \ , p^+}  
\end{aligned}
\right\}
\text{stato iniziale}
\]
e
\[
  \begin{aligned}
  \left. \begin{aligned}
	\Ket{m} = \Ket{p^+ \ , \pi^0 \ , p^+}
  \end{aligned}\right\} &\text{stato intermedio}\\
  \left.\begin{aligned}
    \Ket{f} &= \Ket{p^+ \ , p^+}
  \end{aligned}\right\} &\text{stato finale}
\end{aligned}
\]

L'intero processo presenterà due vertici che sono i punti spazio-temporali che
corrispondono all'emissione e all'assorbimento del mesone $\pi$. Vale in questo
caso la \textit{regola d'oro di Fermi}, dove si ha:
\begin{equation*}
T_{if} = H_{if} = \sum_{\substack{m}} \dfrac{H_{im}H_{mf}}{E_i - E_m}
\end{equation*}
dove la somma è estesa agli stati intermedi con energia $E_m$. 

$H_{im}$ e $H_{mf}$ si riferiscono all'emissione e all'assorbimento di un mesone
con una data carica, e nella somma si considerano i diversi stati di carica.
Il valore $E_m$ si ricava applicando la conservazione della quantità di moto,
che si assume valida nei processi di Yukawa. In particolare si ha che:
\begin{equation*}
E_{\pi} = \sqrt{\vec{p}_{_{\pi}}^2 c^2 + {m_{\pi}}^2 c^4} \qquad \qquad 
\vec{p}_{\pi} = \vec{p}_{N^0} - \vec{p}_{N'}
\end{equation*}

\section{Diagrammi di Feynman}
La dinamica di questo processo può essere studiata tramite i diagrammi di 
Feynman. Consideriamo ad esempio i processi:
\begin{align*}
p^+ + p^+ &\implies p^+ + \pi^0 + p^+ \implies p^+ + p^+ \\
n + n &\implies n + \pi^0 + n \implies n + n
\end{align*}
e questi sono caratterizzati dallo stesso stato intermedio
\begin{equation*}
H_{if} = \dfrac{H_{im}H_{mf}}{E_i - E_m}
\end{equation*}

In questo caso i diagrammi sono quelli di \autoref{fig:feyn_diag}.
\begin{figure}[!htbp]
  \centering
  \caption{Diagrammi di Feynman.}
  \label{fig:feyn_diag}
  \begin{tikzpicture}[>=stealth, auto, scale=1.3]
	% diag 1
	\draw[->] (-.25,-.25) -- (5,-.25) node [below] {$s$};
	\draw[->] (-0.25,-.25) -- (-.25,3) node [left] {$t$};
	\draw[->] (0,0) -- (0.5,1) node [below right] {$n$};
	\draw (.5,1) -- (0.75,1.5) node [left] {$c$};
	\draw[->] (0.75,1.5) -- node [right] {$n$} (0,3);
	\draw[dashed] (.75,1.5) -- (3,1.5);
	\node (a) at (2,1.5) [above] {$\pi^0$};
	\draw[->] (3.75,0) -- (3.25,1) node [below left] {$n$};
	\draw (3.25,1) -- (3,1.5) node [right] {$c$};
	\draw[->] (3,1.5) -- node [left] {$n$} (3.75,3);
	% diag 2
	\begin{scope}[shift={(5.5,0)}]
    \draw[->] (-.25,-.25) -- (5,-.25) node [below] {$s$};
	\draw[->] (-0.25,-.25) -- (-.25,3) node [left] {$t$};
	\draw[->] (0,0) -- (0.5,1) node [below right] {$p^+$};
	\draw (.5,1) -- (0.75,1.5) node [left] {$b$};
	\draw[->] (0.75,1.5) -- node [right] {$p^+$} (0,3);
	\draw[dashed] (.75,1.5) -- (3,1.5);
	\node (a) at (2,1.5) [above] {$\pi^0$};
	\draw[->] (3.75,0) -- (3.25,1) node [below left] {$p^+$};
	\draw (3.25,1) -- (3,1.5) node [right] {$b$};
	\draw[->] (3,1.5) -- node [left] {$p^+$} (3.75,3);
	\end{scope}
  \end{tikzpicture}
\end{figure}
$c$ e $b$ indicano le costanti di accoppiamento degli elementi di matrice 
$H_{mi}$ e $H_{mf}$. Nel primo caso la costante complessiva di accoppiamento è 
$c^2$, nel secondo caso è $b^2$. Per il \textit{principio di indipendenza 
della carica elettrica} deve essere:
\begin{equation*}
c^2 = b^2
\end{equation*}

\breaknote

Nella \marginnote{30-3-1998} trattazione fatta si è assunta una trasmissione
istantanea del mesone $\pi$ da un nucleone all'altro.

Consideriamo ora l'interazione fra due nucleoni diversi. Esistono due possibili
stati intermedi a secondo che il processo avvenga con, o senza, scambio di
carica.

La sommatoria che dà $H_{if}$ sarà, quindi, composta solo da due termini:
\begin{equation}
\label{somma}
H_{if} = \dfrac{H^{(1)}_{im} H^{(1)}_{mf}}{E_i - E^{(1)}_m} + 
\dfrac{H^{(2)}_{im} H^{(2)}_{mf}}{E_i - E^{(2)}_m} 
\end{equation}

Se si trascurano gli effetti elettromagnetici si ha $E^{(1)}_m = E^{(2)}_m$. La
\eqref{somma} può essere rappresentata dalla somma dei seguenti diagrammi:

\begin{figure}[http!]
  \centering
\caption{Diagrammi di Feynman per il mesone $\pi$}
\label{diagrammi2}
  \begin{tikzpicture}[>=stealth, auto, scale=1.3]
	% diag 1
	\draw[->] (-.25,-.25) -- (5,-.25) node [below] {$s$};
	\draw[->] (-0.25,-.25) -- (-.25,3) node [left] {$t$};
	\draw[->] (0,0) -- (0.5,1) node [below right] {$p^+$};
	\draw (.5,1) -- (0.75,1.5) node [left] {$b$};
	\draw[->] (0.75,1.5) -- node [right] {$p^+$} (0,3);
	\draw[dashed] (.75,1.5) -- (3,1.5);
	\node (a) at (2,1.5) [above] {$\pi^0 \rightarrow$};
	\node (b) at (2,1.5) [below] {$\leftarrow \pi^0$};
	\draw[->] (3.75,0) -- (3.25,1) node [below left] {$n$};
	\draw (3.25,1) -- (3,1.5) node [right] {$c$};
	\draw[->] (3,1.5) -- node [left] {$n$} (3.75,3);
	% diag 2
	\begin{scope}[shift={(5.5,0)}]
    \draw[->] (-.25,-.25) -- (5,-.25) node [below] {$s$};
	\draw[->] (-0.25,-.25) -- (-.25,3) node [left] {$t$};
	\draw[->] (0,0) -- (0.5,1) node [below right] {$p^+$};
	\draw (.5,1) -- (0.75,1.5) node [left] {$a$};
	\draw[->] (0.75,1.5) -- node [right] {$n$} (0,3);
	\draw[dashed] (.75,1.5) -- (3,1.5);
	\node (a) at (2,1.5) [above] {$\pi^+ \rightarrow$};
	\node (b) at (2,1.5) [below] {$\leftarrow \pi^-$};
	\draw[->] (3.75,0) -- (3.25,1) node [below left] {$n$};
	\draw (3.25,1) -- (3,1.5) node [right] {$a$};
	\draw[->] (3,1.5) -- node [left] {$p^+$} (3.75,3);
	\end{scope}
  \end{tikzpicture}

\end{figure}

Per lo stato 2\footnote{vedi grafico \ref{diagrammi2} a destra} il mesone $\pi$
può avere carica positiva o negativa a seconda di quale nucleone lo ha emesso.
Il primo diagramma ha una costante di accoppiamento complessiva $bc$, mentre il
secondo $a^2$. I due termini ($1$ e $2$) possono differire solo per queste
costanti di accoppiamento, quindi la costante di accoppiamento dell'interno
processo è:

\begin{equation*}
bc + a^2
\end{equation*}

Per l'indipendenza della carica deve essere:

\begin{equation*}
bc + a^2 = b^2 = c^2
\end{equation*}

La prima uguaglianza è compatibile con la seconda solo se $b = -c$, perché
altrimenti risulterebbe $a = 0$. Quindi si trova che:

\begin{equation*}
a^2=2b^2
\end{equation*}

Possiamo porre $b^2 = - g^2_{_{F}}$ dove $g_{_{F}}$ è la \textit{costante di
accoppiamento di Yukawa}.
Si può, dunque, scrivere in definitiva:

\begin{equation*}
b= i g_{_{F}} \qquad \qquad  c = - i g_{_{F}} \qquad \qquad a = i \sqrt{2}  
g_{_{F}}
\end{equation*}

(Si poteva anche scegliere $ a = - i \sqrt{2}  g_{_{F}}$, la scelta è 
arbitraria
e convenzionalmente si fa la scelta di cui sopra).

Tutti i quattro diagrammi presentano due vertici, questo esprime il fatto che si
tratta di processi del secondo ordine. Mentre per l'interazione debole il
processo fondamentale è di primo ordine, per come abbiamo visto. In realtà
vedremo che anche l'interazione debole è un processo del secondo ordine, dove
però la massa del mesone intermedio è estremamente grande, quindi il range di
interazione è estremamente piccolo (questo spiega perché in prima
approssimazione si può trattare come un processo del primo ordine). 

Un'altra previsione del modello di Yukawa riguarda lo spin del mesone $\pi$. 
In teoria il mesone $\pi$ può avere soltanto uno spin di valore $0$ o $1$,
questo perché i due nucleoni hanno spin $1/2$. Tenendo presente che il deutone
ha spin $1$ si ricava subito che l'unico valore consentito è:

\begin{equation*}
S^{(\pi)} = 0
\end{equation*}
infatti comunque scelta la direzione $z$ si ha che nel deutone:

\begin{equation*}
  \mean{S_z}_n \ = \ \mean{S_z}_{p^+} \implies \mean{S_z}_{\pi} \ = \ 0
\end{equation*}

Quindi il mesone $\pi$ è un bosone con spin 0. 
La teoria di Yukawa può spiegare anche perché il neutrone pur avendo carica
nulla ha un momento di dipolo magnetico non nullo, con rapporto giromagnetico
negativo. Si può giustificare qualitativamente questo fatto supponendo che 
anche
per un neutrone libero si possa avere la dissociazione:

\begin{equation*}
n \to p^+ + \pi^-
\end{equation*}

Il mesone $\pi^-$ ha un momento di dipolo magnetico con g. negativo in quanto
ruoterà attorno al protone.

\section{Scoperta del pione e sue proprietà}
La prima conferma sperimentale dell'esistenza del mesone $\pi$ avvenne nel 1947.
Venne scoperta nei raggi cosmici una particella di massa $m \simeq 139.6
\text{MeV}/c^2$ con spin $0$. Nell'anno successivo fu prodotta in laboratorio 
una
particella identica con carica opposta. Due anni dopo fu scoperta una particella
con spin $0$ e massa $m \simeq 135 \text{MeV}/c^2$. Queste particelle furono 
chiamate
\textit{pioni}. Dalla loro somiglianza con i mesoni $\pi$ fu dedotto che queste
dovevano essere proprio i mesoni $\pi$. La differenza di massa fra i diversi
stati di carica si può giustificare in base ad effetti elettromagnetici. In
realtà i pioni possono essere prodotti mediante urti nucleone-nucleone. Il caso
più semplice è:

\begin{equation*}
N + N \to N' + N' + \pi
\end{equation*}
processi di questo tipo sono ad esempio:

\begin{align*}
p^+ + p^+ &\to p^+ + p^+ + \pi^0 \\
p^+ + p^+ &\to p^+ + n + \pi^+ \\
n + p^+ &\to n + p^+ + \pi^0 \\
n + p^+ &\to p^+ + p^+ + \pi^- \\
n + p^+ &\to n + n + \pi^+
\end{align*}

In queste reazioni il pione reale prodotto non va confuso con quello virtuale
che viene scambiato fra i nucleoni. Un altro modo di creare pioni in 
laboratorio è:

\begin{align*}
\gamma + p^+ &\to p^+ + \pi^0 \\
\gamma + p^+ &\to n + \pi^+
\end{align*}

Due \marginnote{1-4-1998} tipici processi forti che danno luogo ad un pione
positivo e ad uno neutro sono caratterizzati dalle seguenti reazioni:

\begin{align*}
\Ket{\dfrac{1}{2}+\dfrac{1}{2}} + \Ket{\dfrac{1}{2}+\dfrac{1}{2}} = \Ket {1 \ 
1} &\implies \Ket{0 \ 0} + \Ket {1 \ 1} = \Ket{ 1 \ 1} \\
p^+ + p^+ &\implies  d + \pi^+ \qquad \qquad \qquad \bigl(\Ket{t \ , t_3} = 
\Ket{1 \ , 1}\bigr) \\
\\
\Ket{\dfrac{1}{2}+\dfrac{1}{2}} + \Ket{\dfrac{1}{2}-\dfrac{1}{2}} = \Ket {1 \ 
0} &\implies \Ket{0 \ 0} + \Ket {1 \ 0} = \Ket{ 1 \ 0} \\
p^+ + n &\implies  d + \pi^0 \qquad \qquad \qquad \bigl(\Ket{t \ , t_3} = 
\Ket{1 \ , 0} \bigr)
\end{align*}
in queste reazioni i nucleoni finali appaiono legati in un deutone.
Quest'ultimo ha isospin 0 mentre $\pi$ ha isospin $1$, quindi entrambi gli stati
finali hanno isospin $1$. In particolare $d$ e $\pi^+$ caratterizzano
l'autostato di isospin $\Ket{t \ , t_3} = \Ket{1 \ , 1}$, mentre $d$ e $\pi^0$
formeranno l'autostato $\Ket{t \ , t_3} = \Ket{ 1 \ , 0}$. Dalle regole di
composizione del momento angolare sappiamo che:

\begin{align*}
\Ket{1 \ , 1} &= \Ket{p^+} \Ket{p^+} \\
\Ket{ 1 \ , 0 } &= \dfrac{1}{\sqrt{2}} \Bigl[\Ket{p^+}\Ket{n} + \Ket{n} 
\Ket{p^+} \Bigr] \\
\Ket{ 0 \ , 0 } &= \dfrac{1}{\sqrt{2}} \Bigl[\Ket{p^+}\Ket{n} - \Ket{n} 
\Ket{p^+} \Bigr]
\end{align*}

Nella prima reazione si ha all'inizio lo stato $\Ket{1 \ , 1}$ così come alla 
fine. 
Nella seconda invece si ha:

\begin{equation*} \footnote{gli stati $\Ket{p^+}$ e $\Ket{n}$ corrispondono 
rispettivamente a $\Ket{1/2 \ , 1/2}$ e a $\Ket{1/2 \ , - 1/2}$. Combinando 
linearmente questi due stati si possono ottenere due autostati di $T_{TOT} : 
\Ket{ 1 \ , 0} , \Ket{ 0 \ , 0}$.}
\Ket{p^+}\Ket{n} =  \dfrac{1}{\sqrt{2}} \Bigl[ \Ket{1 \ , 0} + \Ket{ 0 \ , 0} 
\Bigr] 
\end{equation*}
come stato iniziale. Alla fine si ha un autostato $\Ket{1 \ , 0}$. Riepilogando
possiamo porre per la prima reazione e per la seconda:

\begin{align*}
&1) \qquad \Ket{1 \ , 1} \implies \Ket{1 \ , 1} \qquad \qquad \qquad &\sigma^+ 
\\
&2) \qquad \dfrac{1}{\sqrt{2}} \Bigl[ \Ket{1 \ , 0} + \Ket{ 0 \ , 0} \Bigr]  
\implies \Ket{1 \ , 0} \qquad \qquad &\sigma^0
\end{align*}
$\sigma^+$ e $\sigma^0$ sono le sezioni d'urto delle due reazioni. La prima
conserva l'isospin quindi $\sigma^+ \ne 0$. Per la seconda l'unico contributo
alla sezione d'urto dovrebbe venire dal canale che conserva l'isospin, cioè

\begin{equation*}
\Ket{1 \ , 0} \implies \Ket{ 1 \ , 0}
\end{equation*}

Indichiamo con $\sigma^0_{(1)}$ la sezione d'urto relativa a questo canale. Si
può concludere che deve essere:

\begin{equation*}
\sigma^0 = \dfrac{1}{2} \sigma^0_{(1)}
\end{equation*}
in quanto il sistema ha probabilità $1/2$ di trovarsi nello stato iniziale
$\Ket{1 \ ,0}$ ($\sigma$ si interpreta come probabilità che avvenga la
diffusione).

Supponendo che i due nucleoni iniziali si trovino nello stesso stato cinematico
e di spin dei due nucleoni iniziali della prima reazione, deve essere:

\begin{equation*}
\sigma^0_{(1)} = \sigma^+
\end{equation*}
trascurando le differenze di massa fra $p^+$ e $n$ e fra $\pi^+$ e $\pi^0$. Per
la conservazione dell'isospin deve dunque essere:

\begin{equation*}
\dfrac{\sigma^0}{\sigma^+} = \dfrac{1}{2}
\end{equation*}

Questa previsione teorica è in ottimo accordo con i risultati sperimentali. 
Esistono altre previsioni teoriche di questa teoria che si accordano pienamente
con i dati sperimentali, quella presentata era una delle più semplici come
trattazione. 

\chapter{Classificazione delle particelle elementari}

Tenendo conto della loro dinamica tutte le particelle elementari si possono
raggruppare in quattro famiglie di massa crescente.

\begin{description}
 \item[Fotone] bosone di spin 1;
 \item[Leptoni] fermioni di spin $1/2$ tutti più leggeri del protone tranne il
   leptone pesante $\tau^-$. I leptoni non sono soggetti all'interazione forte
   ma solo a quella elettromagnetica e a quella debole di Fermi;
 \item[Mesoni] particelle di spin intero e massa diversa da zero; il più 
leggero
   mesone è il $\pi^0$;
 \item[Barioni] questi comprendono i nucleoni e gli iperoni, cioè i fermioni 
più
   pesanti del neutrone eccetto il $\tau^-$. Questi come i mesoni sono soggetti
   a tutte le interazioni, per questo mesoni e barioni si dicono
   \textit{Adroni}.
\end{description}

Tutte queste particelle sono soggette alla forza gravitazionale che però è
perfettamente trascurabile rispetto alle altre interazioni. Possiamo confrontare
la forza gravitazionale considerando una costante dimensionale analoga a quella
di struttura fine ($e^2/\hbar c \simeq 1/137$). Questa costante è

\begin{equation*}
\dfrac{K \ m^2_p}{\hbar c} \simeq 5.88 \times 10^{-39} \qquad (K = G \simeq 6.67
\times 10^{-8} \text{dina} \times \text{cm}^2 \times g^{-2};\; m_p = \text{massa
protone})
\end{equation*}

Tutte le interazioni conosciute rispettano l'invarianza per inversione temporale
a livello microscopico. Quindi quanto più alta è la probabilità di produrre 
una
particella instabile, tanto più alta è la probabilità che questo decada, e 
tanto
più breve sarà il tempo di vita media. Le particelle che subiscono decadimenti
di tipo forte hanno un tempo caratteristico di vita media:

\begin{equation*}
\tau_F \sim \dfrac{r{_0{_F}}}{c} = \dfrac{\hbar}{m_\pi c^2} \sim 10^{-23}
\text{sec} \quad \text{(tempo caratteristico di un processo forte)}
\end{equation*}

Per i decadimenti di natura elettromagnetica o debole si ha invece:

\begin{align*}
  10^{-20} \text{sec} &\le \tau_E \le 10^{-15} \text{sec} \\
  \tau_D &\ge 10^{-10} \text{sec}
\end{align*}

$\tau$ si allunga al diminuire dell'intensità dell'interazione. Molte 
particelle
vivono un tempo così breve da non lasciare tracce osservabili che permettano 
una
loro rivelazione. Queste particelle si manifestano come risonanza. Per
\textit{risonanza} si intende uno stato legato fra particelle che collidono,
individuato da un picco nel grafico della sezione d'urto in funzione
dell'energia di collisione. Una particella si manifesta come risonanza quando il
suo $\tau$ è inferiore a $10^{-12}$ sec.

\chapter{Risonanza}
Supponiamo di avere due fasci di particelle identiche, uno di tipo $a$ e uno di
tipo $b$, che si muovono in direzioni opposte e collidono. Consideriamo una
particella di tipo $a$ e una di tipo $b$ e assumiamo che abbiamo spin zero ($S_a
= S_b = 0$). Sia $\hbar K$ il modulo della quantità di moto delle particelle 
nel
sistema del loro centro di massa. Sia $E$ la loro energia totale. 
L'effetto di risonanza si può avere solo per un ben definito valore di $l$
(\textit{momento orbitale}). Infatti $l$ sarà poi il valore dello spin nello
stato legato che si genera (cioè $l$ sarà lo spin della particella risonante).
In prossimità della risonanza la sezione d'urto ha un particolare andamento in
funzione di $E$ (con $l$ ben definito):
\begin{equation*} \footnote{$\sigma_\text{el} = $si assume che la diffusione 
sia di tipo elastico, cioè dopo la risonanza si hanno sempre due particelle 
$a$ e $b$}
\fbox{$\sigma^{(l)} (E) = \sigma_\text{el}^{(l)} (E) = \dfrac{4\pi}{K^2_R} 
(2l+1) \dfrac{\Gamma^2/4}{(E-E_R)^2 + \Gamma^2/4}$}
\end{equation*}
$E_R =$ energia di risonanza, cioè per $E=E_R \quad \sigma^{(l)}$ è max.
$K_R =$ corrispondente valore di risonanza di K.

Questa formula vale  per:

\begin{equation*}
\abs{E - E_R} \le \Gamma \ll E_R
\end{equation*}

Questa formula è la cosiddetta \textit{sezione d'urto di produzione della
risonanza} (\autoref{fig:risonanza}).
\begin{figure}[http!]
  \centering
  \caption{Funzione $\sigma^{(0)}(E)$}
  \label{fig:risonanza}
  \begin{tikzpicture}[line cap=round,line join=round,>=stealth,x=5.00912315129098cm,y=0.3490485173323042cm]
  \draw[->,color=black] (-0.05,0) -- (1.75,0) node [below] {$E$};
  \foreach \x in {,0.2,0.4,0.6,0.8,1,1.2,1.4,1.6}
  \draw[shift={(\x,0)},color=black] (0pt,2pt) -- (0pt,-2pt);
  \draw[->,color=black] (0,-0.71) -- (0,13.61) node [left] {$\sigma^{(0)}$};
  \foreach \y in {,2,4,6,8,10,12}
  \draw[shift={(0,\y)},color=black] (2pt,0pt) -- (-2pt,0pt);
  \draw[smooth,samples=100,domain=-0.046248828371781535:1.7504728189449685]
  plot(\x,{4*3.1415926535*1/4*1/(((\x)-1)^2+1/4)});
  \draw [dashed] (0,12.57)-- (1,12.57);
  \draw [dashed] (0,6.28)-- (0.5,6.28);
  \draw [dashed] (0.5,0)-- (0.5,6.28);
  \draw [dashed] (1.5,0)-- (1.5,6.28);
  \draw [dashed] (1,12.57)-- (1,0);
  \begin{scope}[>=triangle 45]
    \draw [<->] (0.5,6.28) -- (1.5,6.28);
  \end{scope}
	\fill [color=\MinorColor] (1,12.57) circle (1.5pt);
	\fill [color=\MinorColor] (0.5,6.28) circle (1.5pt);
	\fill [color=\MinorColor] (1.5,6.28) circle (1.5pt);
	\fill [color=\MinorColor] (0,12.57) circle (1.5pt);
	\draw[color=\MinorColor] (0.0,12.57) node [left] {$\sigma_\text{max}$};
	\fill [color=\MinorColor] (0,6.28) circle (1.5pt);
	\draw[color=\MinorColor] (0.0,6.28) node [left]  {$\sigma_\text{max}/2$};
	\fill [color=\MinorColor] (0.5,0) circle (1.5pt);
	\draw[color=\MinorColor] (0.5,0) node [below]  {$E - \Gamma/2$};
	\fill [color=\MinorColor] (1.5,0) circle (1.5pt);
	\draw[color=\MinorColor] (1.5,0) node [below] {$E+ \Gamma/2$};
	\fill [color=\MinorColor] (1,0) circle (1.5pt);
	\draw[color=\MinorColor] (1.0,0) node [below] {$E_R$};
	\node (g) at (1,6.28) [above left] {$\Gamma$};
\end{tikzpicture}

\end{figure}
$\Gamma =$ larghezza della curva di risonanza. In realtà $\Gamma$ ha un
significato fisico molto più profondo.

\begin{equation*}
\sigma \Biggl(E_R - \dfrac{\Gamma}{2}\Biggr) = \sigma \Biggl(E_R + 
\dfrac{\Gamma}{2}\Biggr) = \dfrac{1}{2} \sigma_\text{max}\footnote{
$\sigma_\text{max} = \dfrac{4\pi}{K^2_R} (2l + 1) \qquad \text{di risonanza} (E 
= E_R)$}. 
\end{equation*}

\breaknote

Consideriamo \marginnote{3-4-1998} due fasci di particelle ($a$ e $b$) che si
scontrano.
\begin{figure}[!h]
  \centering
  \caption{Fasci in collisione di spin $0$}
  \label{fig:fasci}
  \begin{tikzpicture}[>=triangle 45]
  \foreach \y in {0, .25, .5, .75, 1}
  {
    \draw[->] (0, \y) -- (4, \y);
    \draw[<-] (5, \y) -- (9, \y);
  }
  \node (a) at (-.25,.37) [left] {$a$};
  \node (b) at (9.25,.37) [right] {$b$};
  \node (sa) at (2,-.25) [below] {$s_a=0$};
  \node (sb) at (7,-.25) [below] {$s_b=0$};
  \node (spazio) at (0,1.25) {};
\end{tikzpicture}

\end{figure}
Se associamo allo stato risonante una funzione d'onda quasi
stazionaria con un andamento temporale del tipo:

\begin{equation*}
\Psi(t) = \Psi(0) e^{-\dfrac{i}{\hbar} (E_R - i\Gamma_R/2) t}
\end{equation*}

Si può dimostrare che questa $\Gamma_R$ è proprio il parametro di larghezza
della curva di risonanza ($\Gamma$): $\Gamma_R \equiv \Gamma$

Sappiamo pure che:

\begin{equation*}
  \abs{\Psi(t)}^2 = \exp\left[-\frac{\Gamma_R}{\hbar}t\right]\footnote{Questa è
  la probabilità che la particella esista quindi si può considerare che sia
stato fatto un integrale sullo spazio.}
\end{equation*}
quindi il tempo di vita media dello stato risonante è:

\begin{equation*}
\fbox{$\tau_R = \dfrac{\hbar}{\Gamma_R} = \dfrac{\hbar}{\Gamma}$}
\end{equation*}

$\tau_R$ si può ricavare sperimentalmente dalla larghezza della curva di
risonanza. Si usa questa tecnica per ricavare i tempi di vita media delle
particelle che decadono in modo forte.

Abbiamo assunto che le particelle $a$ e $b$ sono prive di spin, quindi:

\begin{equation*}
2 l + 1 = 2 J_R + 1 \qquad \qquad \qquad J_R = \text{valore di spin della 
risonanza}
\end{equation*}
quindi la formula della sezione d'urto si può interpretare come somma di $2 J_R
+ 1$ termini, che sono sezioni d'urto identiche.

$2 J_R + 1$ è il \textit{peso statistico della risonanza}, cioè il numero di
autostati. Il momento angolare orbitale di due particelle $a$ e $b$ può avere 
un
qualsiasi valore in un piano perpendicolare alla loro velocità relativa. Di
conseguenza scegliendo l'asse $z$, su questo piano la risonanza può essere
indifferentemente prodotta in $2 J_R + 1$ autostati diversi. Quindi la sezione
d'urto considerata è quella complessiva, ottenuta sommando su questi possibili
autostati differenti. Questo è il significato fisico del fattore $(2 l + 1)$.

Consideriamo ora il caso in cui lo spin delle particelle $a$ e $b$ è non nullo,
$S_a \ne 0$ e $S_b \ne 0$; supponiamo di avere due fasci non polarizzati, cioè
lo spin delle particelle può essere orientato in una qualsiasi direzione.
Calcoliamo ora la probabilità che due particelle con un $l$ fissato abbiano una
giusta orientazione di spin per produrre una risonanza con un dato $J_R$. Questa
probabilità è:

\begin{equation*}
\dfrac{N_R}{N} 
\end{equation*}
\begin{description}
\item[$N_R$] numero di autostati dello spin della risonanza.
\item[$N$] numero di possibili autostati del momento angolare totale delle due
  particelle.
\end{description}

\begin{equation*}
N_R = 2 J_R + 1	\qquad  \qquad	N = (2 S_a + 1)(2 S_b + 1)(2 l + 1)
\end{equation*}

Quindi la probabilità che cerchiamo è:

\begin{equation*}
g_{_{R}} = \dfrac{N_R}{N} = \dfrac{(2 J_R + 1)}{(2 S_a + 1)(2 S_b + 1)(2 l + 1)}
\end{equation*}
adesso $J_R$ non è più uguale a $l$ appunto perché $S_a \ne 0$ e $S_b \ne 0$.
L'effettiva sezione d'urto della risonanza in questo caso è:

\begin{equation*}
\sigma^{(J_R)} (E) = g_{_{R}} \sigma^{(l)} (E) = \dfrac{4 \pi}{K^2_R} \dfrac{(2 
J_R +1)}{(2 S_a + 1)(2 S_b + 1)} \dfrac{\Gamma^2 /4}{(E-E_R)^2 + \Gamma^2/4}
\end{equation*}

Se poniamo $S_a = S_b = 0$ si ha che $J_R = l$ e si riottiene la formula nota. 
Anche questa sezione d'urto vale solo nell'intorno di $E_R$. La sezione d'urto
totale mostra solitamente diversi picchi.

\chapter{Antiparticelle}
L'esperienza mostra che ogni particella ha una antiparticella. Dal punti di
vista teorico si può prevedere l'esistenza delle antiparticelle già nella
relatività ristretta, dove l'energia non è più definita positiva:

\begin{equation*}
E^2 = m^2 c^4 + c^2 p^2 \implies E = \pm \sqrt{m^2 c^4 + c^2 p^2}
\end{equation*}
$E$ è la componente temporale del quadrimpulso della particella

\begin{equation*}
p^i = (E/c \ , \vec{p})
\end{equation*}
le due radici di $E^2$ corrispondono ad un moto avanti o indietro nel tempo,
così come le due radici di $p^2$ corrispondono a due moti avanti e indietro
nello spazio. In effetti un moto con $E < 0$ sembra a prima vista inaccettabile
in quanto apparentemente viola il principio di causalità.

Consideriamo una particella che viene emessa nel punto $(c t_a \ , \vec{r}_a)$ e
venga assorbita nel punto $(c t_b \ , \vec{r}_b )$ con $t_b > t_a$, cioè questa
particella si muove in avanti nello spazio tempo. Se la particella si muovesse
nella direzione spazio-temporale opposta verrebbe emessa in $(c t_b \ ,
\vec{r}_b)$ e assorbita in $(c t_a \ , \vec{r}_a)$, dove è sempre $t_a < t_b$.
In questo caso la particella si muoverebbe indietro nel tempo e la sua morte
precederebbe la sua nascita. Questa è un'apparente violazione del principio
causa-effetto. Per superare questo dilemma basta considerare che l'emissione o
l'assorbimento di un quadrimpulso.

\begin{equation*}
- p^i = (- E/c \ , - \vec{p})
\end{equation*}
equivale all'assorbimento o emissione di un quadrimpulso opposto

\begin{equation*}
p^i = ( E/c \ , \vec{p})
\end{equation*}

Quindi si può rappresentare questa situazione di equivalenza nel seguente modo:

\begin{equation*}
(c t_a \ , \vec{r}_a) \xrightarrow[(E/c \ , \vec{p}) \to ]{\gets (-E/c \ , - 
\vec{p})} (c t_b \ , \vec{r}_b)
\end{equation*}

Questo è un discorso analogo alla corrente elettrica. Quindi l'assorbimento e 
la
successiva emissione di una particella con $E < 0$ si può interpretare come
emissione e successivo assorbimento di una particella con $E$ e $\vec{p}$
opposti. Se la particella con $E < 0$ possiede una carica elettrica $-e$ l'altra
particella deve possedere una carica $+e$. In realtà questo ragionamento si 
può
estendere ad un qualsiasi numero quantico additivo che caratterizza internamente
la particella.

Un moto indietro nel tempo di una particella con $E < 0$ è equivalente ad un
moto avanti nel tempo della corrispondente antiparticella. Con questa
rappresentazione non viene più violato il principio di causa ed effetto. Si 
può
dunque attribuire un significato fisico alla radice negativa dell'energia solo
se si considera l'antimateria. Questo procedimento fu introdotto per la prima
volta da Feynman. Secondo questa definizione relativistica, l'antiparticella ha
un spin uguale e momento magnetico opposto rispetto alla particella.

Se la particella è caratterizzata dai numeri quantici $( t \ , t_3)$ di isospin
l'antiparticella avrà numeri quantici di isospin $(t \ , - t_3)$.
Ad esempio $\bar{p}$ e $\bar{n}$ sono sempre un doppietto di isospin però:
 
 \begin{equation*}
 T_3 \Ket{\bar{p}} = -\dfrac{1}{2} \Ket{\bar{p}} \qquad \qquad T_3 
\Ket{\bar{n}} = \dfrac{1}{2} \Ket{\bar{n}}
 \end{equation*}

\chapter{Coniugazione di carica}

Siano \marginnote{6-4-1998}$\Ket{a}$ e $\Ket{\bar{a}}$ gli stati di particella 
e antiparticella.
Consideriamo l'operatore unitario $C$ tale che:
\begin{equation*}
C\Ket{a} = \Ket{\bar{a}} \qquad \text{e} \qquad C^\dag \Ket{\bar{a}} = \Ket{a}
\qquad \qquad (\implies C^\dag = C^{-1})
\end{equation*}
ma deve ovviamente essere:
\begin{equation*}
C^2 = I = \text{operatore identità} \qquad \qquad \qquad (\implies C^\dag = C)
\end{equation*}
con questa scelta risulta pure hermitiano. 

I suoi autovalori sono quindi reali e il loro quadrato  deve sempre essere $1$,
quindi gli autovalori sono solamente $+1$ e $-1$. Il numero quantico associato a
$C$ non è additivo ma moltiplicativo, cioè l'autovalore di un sistema
complessivo è dato dal prodotto degli autovalori dei sistemi costituenti.
L'effetto dell'operatore $C$ è quello di cambiare il segno di tutte le 
"cariche"
delle particelle o antiparticelle. Ad esempio:
\begin{equation*}
C\Ket{\pi^-} = \Ket{\pi^+}	\qquad \qquad \qquad C\Ket{\pi^+} = \Ket{\pi^-}
\end{equation*}
quindi una particella assolutamente "neutra" rappresenta un autostato
dell'operatore $C$. Quindi questa particella è l'antiparticella di se stessa.
Questo è il caso, ad esempio, del fotone $\gamma$ e del pione neutro, infatti:
\begin{equation*}
C\Ket{\gamma} = - \Ket{\gamma} 	\qquad \qquad \qquad C\Ket{\pi^0} = \Ket{\pi^0}
\end{equation*}
l'autovalore $-1$ attribuito a $\Ket{\gamma}$ si ricava dal fatto che il
potenziale vettore elettromagnetico, che rappresenta la funzione d'onda del
fotone, cambia segno se cambiano segno le cariche che lo generano. L'autovalore
$+1$ per $\pi^0$ si può ricavare dalla reazione:
\begin{align*}
\pi^0 &\to \gamma+\gamma \\
C\Ket{2\gamma} &= (-1)(-1) \Ket{2\gamma} = \Ket{2\gamma}
\end{align*}

In realtà si è implicitamente ammesso, in questa deduzione, che l'operatore 
$C$
rappresenta una grandezza che si conserva durante il processo di decadimento.
Questa proprietà di conservazione in realtà è collegata col fatto 
sperimentale
che la coniugazione di carica è una operazione di simmetria nei processi
elettromagnetici. Infatti grandezze come le sezioni d'urto, le probabilità, le
costanti di disintegrazione, ecc., rimangono invariate. In un sistema isolato se
si applica $C$ al sistema, allora $C$ sarà una operazione di simmetria se non
cambia \textit{l'equazione di Schr\"odinger}. Poniamo:
\begin{equation*}
\Psi^{(C)} \equiv C \Psi = \text{funzione d'onda coniugata di carica}
\end{equation*}
allora $C$ è una operazione di simmetria se e solo se:
\begin{equation*}
i\hbar \dfrac{\partial}{\partial t} \Psi^{(C)} = H \Psi^{(C)} \implies i\hbar 
\dfrac{\partial}{\partial t} C \Psi = H C \Psi
\end{equation*}

$C$ non dipende dal tempo, si può dunque scrivere:
\begin{equation*}
 i\hbar \dfrac{\partial}{\partial t} C \Psi = C i \hbar 
\dfrac{\partial}{\partial t} \Psi = C H \Psi \implies C H \Psi = H C \Psi 
\implies \fbox{$[C, H] = 0$}
\end{equation*}

Se $C$ è una operazione di simmetria allora è una costante del moto ed il suo
numero quantico si conserva.
Tutto questo discorso convalida la scelta come autovalore di $\Ket{\pi^0}$ del
numero $+1$.
Vedremo che non è verificato che $C$ è una operazione di simmetria nei 
processi
deboli, mentre lo è nei processi forti ed in quelli elettromagnetici.
Consideriamo il neutrino, che può comparire solo nei processi deboli, e con
l'ipotesi di massa nulla può esistere solo nel suo stato sinistrorso, cioè con
elicità $-1$:

\begin{align*}
\Ket{\nu} &= \Ket{\nu_s} \\
C \Ket{\nu_s} &= \Ket{\bar{\nu}_s}
\end{align*}

$C$ per definizione lascia invariata l'elicità in quanto non cambia né lo spin
né la quantità di moto. Quindi se si avesse simmetria rispetto a $C$ dovrebbe
esistere lo stato $\Ket{\bar{\nu}_s}$, mentre sappiamo che l'antineutrino
esiste solo nello stato destrorso, cioè $\Ket{\bar{\nu}_D}$.

Consideriamo un elettrone $\beta$ emesso longitudinalmente rispetto alla
direzione del suo spin, il suo stato si può scrivere come una
mistura:
\begin{equation*}
\Ket{e^-_\beta} = C_s \Ket{e^-_s} + C_D \Ket{e^-_D}
\end{equation*}

$C_s$ e $C_D$ sono definiti a meno di una costante di fase, $\abs{C_s}^2$ e
$\abs{C_D}^2$ ci danno la probabilità dei due stati di elicità. Si ha che:
\begin{equation*}
  \mean{H}_{e_{\beta}} = - \abs{C_s}^2 + \abs{C_D}^2 = - \frac{V}{C} \qquad 
\text{(risultato sperimentale)}
\end{equation*}
e la semplice applicazione di $C$ a questo stato comporta
\begin{equation*}
C\Ket{e^-_\beta} = C_s C \Ket{e^-_s} + C_D C \Ket{e^-_D} = C_s \Ket{e^+_s} + 
C_D\Ket{e^+_D}
\end{equation*}
dunque se il processo debole rispettasse la simmetria rispetto a $C$ dovrebbe
esistere un processo simmetrico che produce un positrone con lo stesso valore di
aspettazione per $H$.  Questo invece non
succede, in quanto nell'antiprocesso $\beta$ il valore di aspettazione di $H$ di
un positrone emesso con polarizzazione longitudinale è opposto a quello
dell'elettrone $\beta$. Lo stato del positrone $\beta$ si può scrivere come
\begin{equation*}
\Ket{e^+_\beta} = \bar{C}_s \Ket{e^+_s} + \bar{C}_D \Ket{e^+_D}
\end{equation*}
dove $\bar{C}_s$ e $\bar{C}_D$ sono tali che: $\abs{\bar{C}_s}^2 = \abs{C_D}^2$
e $\abs{\bar{C}_D}^2 = \abs{C_s}^2$.

Quindi l'interazione debole oltre a violare la simmetria per inversione spaziale
viola pure la simmetria per coniugazione di carica.

Esiste comunque una simmetria fra il processo debole e l'antiprocesso e questa
si ottiene applicando insieme la coniugazione di carica e l'inversione spaziale.

L'esistenza di questa simmetria è suggerita dal fatto che grandezze come le
sezioni d'urto complessive e le costanti di disintegrazione sono identiche fra
processo e antiprocesso debole. 

\chapter{Operatore parità}

Definiamo l'operatore parità $P$ come l'operatore che agisce nello spazio degli
stati e che rappresenta la trasformazione associata all'inversione spaziale
ordinaria. Per definizione quindi $P$ trasforma un sistema quantistico nella sua
immagine speculare. A meno di una costante di fase si può dunque scrivere:

\begin{equation*}
| P \Ket{\vec{r}} = \Ket{-\vec{r}} \qquad \qquad P \Ket{\vec{p}}=\Ket{-\vec{p}} 
\qquad \qquad P \Ket{ j m_j} = \Ket{j m_j}
\end{equation*}

Consideriamo adesso l'applicazione successiva dei due operatori $P$ e $C$ allo
stato del neutrino sinistrorso:

\begin{equation*}
C P \Ket{\nu_s} = C \Ket{\nu_D} = \Ket{\bar{\nu}_D}
\end{equation*}

Analogamente si ha per lo stato $\Ket{e^-_\beta} = C_s\Ket{e^-_s} + C_D
\Ket{e^-_D}$:

\begin{align*}
C P \Ket{e^-_\beta} &= C_s C P \Ket{e^-_s} + C_D C P \Ket{e^-_D} = \\
&= C_s C \Ket{e^-_D} + C_D C \Ket{e^-_s} = C_s \Ket{e^+_D} + C_D \Ket{e^+_s}
\end{align*}

Quindi a meno di una costante di fase si ha:
\begin{equation*}
  CP  \Ket{e^-_\beta} = \Ket{e^+_\beta}
\end{equation*}

Nei processi deboli conosciuti, tranne una sola eccezione, l'operazione $C P$
sembra lasciare invariate tutte le probabilità di transizione. I processi
deboli, quindi, non obbediscono alla simmetria per le operazioni rappresentate
da $C$ e da $P$, ma obbediscono a quella relativa all'intera operazione
rappresentata da $CP$. Questo consente di formulare una più ampia simmetria
speculare secondo cui la vera immagine speculare di un processo debole si
ottiene applicando una inversione spaziale e contemporaneamente cambiando di
segno tutte le cariche.

Definiamo ora l'operatore $T$ come quell'operatore che inverte velocità e
impulso lasciando invariata l'energia.

Sussiste il \textit{teorema $CPT$} secondo cui tutti i processi dinamici
realizzabili in natura devono essere simmetrici rispetto all'operazione $CPT$.
L'enunciato del teorema può essere dimostrato sotto ipotesi ben poco
restrittive:

\begin{itemize}
\item[1)] invariata rispetto alle trasformazioni di Lorentz;
\item[2)] la validità della relazione tra spin e statistica. 
\end{itemize}

Una delle più importanti conseguenze del teorema è l'identità di massa fra
particella e antiparticella. L'operatore $CPT$ trasforma la particella
nell'antiparticella con identico impulso ed identica energia e quindi eguale
massa. Inoltre rimane invariata la larghezza del livello energetico della
particella e quindi corrisponderà un tempo di vita media $\tau = \hbar / 
\Gamma$
uguali per la particella e la sua antiparticella.

L'operatore parità $P$ è un operatore hermitiano e unitario:

\begin{equation*}
P^{-1} = P^+ = P \qquad \qquad \qquad P^2 = 1
\end{equation*}
quindi i possibili autovalori di $P$ sono solo $+1$ e $-1$; così come avviene
per l'operatore di carica, il numero quantico associato a $P$ è moltiplicativo.
Sia nei processi forti che in quelli elettromagnetici l'inversione spaziale è 
un
processo di simmetria; questo significa che se un sistema isolato	evolve
dinamicamente attraverso un canale forte o elettromagnetico, la sua
\textit{equazione di Schr$\ddot{o}$dinger} ammetterà sia una soluzione
$\Psi(\vec{r},t)$, sia quella spazialmente riflessa: $\Psi^{(P)} (\vec{r} , t)
\equiv P \Psi (\vec{r}, t)$. Se $H$ è l'hamiltoniana del sistema, allora si
dovrà avere che:

\begin{equation*}
i \hbar \dfrac{\partial}{\partial t} P \Psi = P H \Psi = H P \Psi \implies 
\fbox{$[P \ , H]=0$}
\end{equation*}
quindi sarà possibile parlare di parità. Consideriamo in particolare una
particella priva di spin immersa in un campo esterno radiale. Se la funzione
d'onda della particella è un'autofunzione dell'energia del tipo:

\begin{equation*}
\Psi(\vec{r}) = \Psi ( r \ , \theta \ , \varphi) = \chi(r) \Psi^m_l (\theta \ , 
\varphi)
\end{equation*}
allora essa sarà anche un'autofunzione della parità. 

Verifichiamo questo enunciato.
Dato che il campo è a simmetria sferica, l'\textit{equazione di
  Schr$\ddot{o}$dinger} avrà anche come soluzione la funzione:

\begin{equation*}
\Psi(-\vec{r}) = \Psi ( r \ , \pi-\theta \ , \pi+\varphi) = \chi(r) \Psi^m_l 
(\pi-\theta \ , \pi+\varphi)
\end{equation*}

Si può quindi porre, a meno di una costante di fase:

\begin{equation*}
\Psi(P)(\vec{r}) = \Psi (-\vec{r})
\end{equation*}
considerando che per le armoniche sferiche risulta:

\begin{equation*}
\Psi^m_l (\pi-\theta \ , \pi+\varphi) = (-1)^l \Psi^m_l(\theta \ , \varphi)
\end{equation*}

dove $l$ è il numero quantico orbitale della particella rispetto al centro del
campo. In definitiva si ha:

\begin{equation*}
P \Psi(\vec{r})= \Psi (-\vec{r}) = \chi (r) \Psi^m_l (\pi - \theta \ , 
\pi+\varphi) = (-1)^l \Psi(\vec{r})
\end{equation*}

dunque la particella si trova in un autostato della parità con autovalore 
$(-1)^l$.
Quando detto si applica anche al caso di un sistema isolato di due particelle
interagenti in quanto il loro moto relativo equivarrà a quello della
corrispondente massa ridotta in un campo radiale esterno. In generale, la
conservazione della parità implica che un sistema isolato con una data parità
iniziale, abbia alla fine ancora la stessa parità. Ciò risulta banale nel caso
di una particella priva di spin immerso in un campo centrale in quanto questo
sistema è caratterizzato da un momento angolare orbitale che si conserva. Per 
un
sistema qualunque può succedere tuttavia che il numero di particelle che lo
compongono, vari nel tempo ( a causa di creazioni o di distruzioni), in questo
caso per ottenere un'effettiva conservazione della parità non sempre basta
considerare la parità legata al numero quantico orbitale del sistema, infatti
molto spesso accade che il solo momento angolare orbitale non viene conservato
nel processo. Quando si verifica ciò, la conservazione della parità può 
essere
ottenuta solo se alla particella si attribuisce una parità intrinseca.
