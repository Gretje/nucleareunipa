%!TEX root = nucleare.tex
%DA PAG 41 A 61
Cioè si ha che: 
\begin{equation}
\vec{J} = \vec{L} + \vec{S}
\end{equation}
Se la particella non ha spin $\left( s = 0\right)$, $\vec{J}$ coincide con
$\vec{L}$ e, quindi, $j$ può assumere solo valori interi. Al posto di $j$ di
solito si usa $l$, che si dice numero quantico orbitale. \textit{Lo spin,
invece, è caratterizzato sia da valori interi che seminteri.}

Consideriamo ora il caso in cui si hanno due distinti momenti angolari,
$\vec{J_{1}}$ e $\vec{J_{2}}$, associati rispettivamente al sistema 1 e 2.
Supponiamo che sia:
\begin{equation}
\left[\vec{J_{1}}, \vec{J_{2}} \right] = 0
\end{equation}
cioè ogni componete di $\vec{J_{1}}$ commuta con ogni componente di
$\vec{J_{2}}$. Ogni autostato del sistema 1 sarà caretterizzato dai valori medi:
\begin{equation}
\mean{\vec{J_{1}}} = \left( \mean{J_{1x}}, \mean{J_{1y}}, \mean{J_{1z}} \right)
\end{equation}
e, ovviamente, anche per il sistema 2 si definisce:
\begin{equation}
\mean{\vec{J_{2}}} = \left( \mean{J_{2x}}, \mean{J_{2y}}, \mean{J_{2z}} \right)
\end{equation}
Si pone:
\begin{align}
\mean{\vec{J_{1}}} \cdot \mean{\vec{J_{1}}} &= j_{1}^{2}\hbar ^2 & \mean{\vec{J_{1}^{2}}} &= j_{1}\left(j_{1} +1\right)\hbar ^{2} \\
\mean{\vec{J_{2}}} \cdot \mean{\vec{J_{2}}} &= j_{2}^{2}\hbar ^2 & \mean{\vec{J_{2}^{2}}} &= j_{2}\left(j_{2} +1\right)\hbar ^{2} \\  
\end{align}
Definiamo l'operatore 
\begin{equation}
\vec{J} = \vec{J_{1}} + \vec{J_{2}}
\end{equation}
che è ancora un momento angolare: si possono verificare le regole di
commutazione per le sue componenti.

Si avrà che
\begin{align}
\mean{\vec{J}}\mean{\vec{J}} &= j^{2} \hbar ^{2} & \mean{\vec{J^{2}}} &= j\left(j+1\right)\hbar 
\end{align}
Si dimostra che fissati $j_{1}$ e $j_{2}$, $j$ può assumere solo dei valori ben
determinati, in quanto deve essere verificata la relazione
\begin{equation}
\abs{j_{1} - j_{2}} \leq j \leq j_{1} + j_{2} 
\end{equation}
Questa si dice anche \textit{condizione tringolare}. Questa stessa condizione si
può riscrivere nella forma:
\begin{equation}
\abs{j_{1} - j_{2}} \leq j = \left(m_{j}\right)_\text{\scshape max} \leq j_{1} + j_{2}
\end{equation}
Ma si ha che il generico autovalore $m_{j}$ è dato da
\begin{equation}
m_{j} = m_{j_{1}} + m_{j_{2}}
\end{equation}
Quindi, essendo nota la variabilità di $m_{j_{1}}$ e $m_{j_{2}}$, si deduce che
\begin{equation}
j = j_{1} + j_{2}, j_{1} + j_{2}-1, \dots , \abs{j_{1} - j_{2}}
\end{equation}
Se $j_{1}<j_{2}$, sono $2j_{1} + 1$ valori. Se $j_{1}>j_{2}$, sono $2j_{2} + 1$
valori.

In meccanica classica si aveva già la condizione triangolare, la proprietà nuova
è che $j$ non può assumere tutti i valori fra il minimo ed il massimo. 

\textit{La degenerazione del sottospazio che si ottiene con $j_{1}$ e $j_{2}$
fissati è $\left(2j_{1} + 1\right)\left(2j_{2}+1\right)$}.

Consideriamo gli stati $m_{j}$, autostati di $\vec{J}^{2}$ e $J_{z}$. Si ha
sempre che, fissato $j$, si hanno $2j +1$ autostati. Questa molteplicità si dice
\textit{peso statistico} della stato con momento angolare $J$.

Fissati $j_{1}$ e $j_{2}$, il numero totale di stati è 
\begin{equation}
N = \sum _{j} \left(2j +1\right)
\end{equation}
con $j=\abs{j_{1} - j_{2}}, \dots , j_{1} +j_{2}$.
Se, ad esempio, $j_{1}<j_{2}$ la sommatoria è data da $2j_{1} + 1$ termini
\begin{equation}
N = \left(2j_{1} + 1\right)\left(2j_{2}+1\right)
\end{equation}
Questo numero coincide con il numero di stati 
\begin{equation}
\mid m_{j}\rangle = \mid m_{j_{1}}\rangle \mid m_{j_{2}}\rangle
\end{equation}
con $j_{1}$ e $j_{2}$ fissati. Effettivamente questi due insieme di vettori (
$\mid m_{j_{1}}\rangle \mid m_{j_{2}}\rangle$ e $\mid m_{j}\rangle$ con $j_{1}$
e $j_{2}$ fissati) sono due basi che generano lo stesso sottospazio ma non coincidono.

Abbiamo\marginnote{19-12-1997} detto che gli insiemi di stati $ \mid m_{j_{1}}
m_{j_{2}}\rangle$ e $\mid m_{j}\rangle$ sono due basi di un sottospazio di
dimensione $ N = \left( 2j_{1} + 1\right) \left(2j_{2} +1\right)$. Questi due
insiemi di stati non possono coincidere in quanto l'operatore $\vec{J}^{2}$
\textit{non commuta con gli operatori $J_{1z}$ e $J_{2z}$}. Quello che si può
fare è esprimere gli stati $\mid m_{j}\rangle$ in funzioni degli stati $\mid
m_{j_{1}} m_{j_{2}}\rangle$. Supponendo che tutti questi stati siano
normalizzati si può scrivere che:
\begin{equation}
\mid m_{j}\rangle = \sum _{m_{j_{1}}} \langle m_{j_{1}} m_{j_{2}} \mid m_{j}
\rangle \mid m_{j_{1}} m_{j_{2}} \rangle
\end{equation}
I coefficienti $\langle m_{j_{1}} m_{j_{2}} \mid m_{j} \rangle$ si dicono
\textit{coefficienti di Clebsh-Gordan}.

Sappiamo che deve valere la condizione $m_{j}= m_{j_{1}} + m_{j_{2}}$, quindi,
si ha
\begin{equation}
\langle m_{j_{1}} m_{j_{2}} \mid m_{j} \rangle = 0 \quad \quad \text{se $m_{j_{1}} + m_{j_{2}} \neq m_{j}$}
\end{equation}
Vediamo ora come si calcolano gli altri coefficienti di Clebsh-Gordan
\footnote{Per la \eqref{eq:1}: Si tratta di moltiplicare a sinistra per il BRA
$\langle m_J + 1 \mid$ [NdT]}
\begin{align}
\label{eq:1}
J_{+} \mid m_{j} \rangle &= c_{+} \mid m_{j} + 1 \rangle & \text{dove } c_{+} &= \langle m_{j} +1 \mid J_{+} \mid m_{j} \rangle \\
J_{-} \mid m_{j} +1 \rangle &= c_{-} \mid m_{j} \rangle & \text{dove } c_{-} &= \langle m_{j} \mid J_{-} \mid m_{j} +1 \rangle 
\end{align}
Dato che $J_{-}$ è il coniugato hermitiano di $J_{+}$ si ha che:
\begin{equation}
\langle m_{j} \mid J_{-} \mid m_{j} +1 \rangle = \left( \langle m_{j} +1 \mid J_{+} \mid m_{j} \rangle \right) ^{*} \Rightarrow c_{-} = c_{+} ^{*}
\end{equation}
Imponendo che $c_{-}$ e $c_{+}$ siano reali (c'è sempre una arbitrarietà nella
scelta della fase) \footnote{Si può fare questa scelta (entrambi reali) perchè
sono complessi coniugati [NdT].} si ha che $c_{-} = c_{+} = c$. Da questo di
può dedurre che
\begin{equation}
J_{-}J_{+} \mid m_{j} + 1 \rangle = c^{2} \mid m_{j} + 1 \rangle
\end{equation}
Ma si può scrivere che
\begin{equation}
J_{+}J_{-} = \left( J_{x} + iJ_{y} \right)\left( J_{x} - iJ_{y} \right) = J_{x}^{2}  + J_{y}^{2} - i\left(J_{x}J_{y} - J_{y}J_{x} \right) = J_{x}^{2}  + J_{y}^{2} + \hbar J_{z} = \vec{J}^{2} - J_{z} ^{2} + \hbar J_{z}
\end{equation}
Da questa espressioni di $J_{+}J_{-}$ si può ricavare il valore di $c^{2}$.
Infatti:
\begin{equation}
J_{+}J_{-} \mid m_{j} +1 \rangle = \hbar ^{2} \left[j\left(j+1\right) -
\left(m_{j} +1\right)^{2} + \left(m_{j} + 1 \right) \right] \mid m_{j} + 1
\rangle = c^{2}\mid m_{j} +1 \rangle
\end{equation}
\begin{equation}
\begin{split}
c^{2} &= \hbar ^{2}  \left[j\left(j+1\right) - \left(m_{j} +1\right)^{2} +
\left(m_{j} + 1 \right) \right] \\
&= \hbar ^{2} \left[ j\left(j+1\right) - m_{j}^{2} - m_{j}\right] \\
&= \hbar ^{2} \left[j\left(j+1\right) - m_{j}\left(m_{j} +1\right)\right]
\end{split}
\end{equation}
Quindi abbiamo determinato l'azione di $J_{+} e J_{-}$. Infatti
\begin{align}
J_{+} \mid m_{j} \rangle &= \hbar \sqrt{j\left(j+1\right) - m_{j}\left(m_{j} +1\right)} \mid m_{j} +1 \rangle \\
J_{-} \mid m_{j}\rangle &= \hbar \sqrt{j\left(j+1\right) - m_{j}\left(m_{j} -1\right)} \mid m_{j} -1 \rangle
\end{align}
Questi coefficienti si annullano se $m_{j} = j$ per il primo e se $m_{j} =-j$
per il secondo, quindi:
\begin{align}
J_{+}\mid j\rangle &= 0 \\
J_{-}\mid -j\rangle &= 0
\end{align}
Ritorniamo al problema del calcolo dei coefficienti di Clabsh-Gordan.

Sfruttando l'azione di $J_{+}$ e $J_{-}$ sullo stato $\mid m_{j}\rangle$ e sulla
sua scomposizione in temini degli stati $\mid m_{j_{1}} m_{j_{2}} \rangle$ si
determinano due relazioni di ricorrenza per i coefficienti di Clabsh-Gordan.
Tramite queste regole e partendo da uno stato $\mid m_{j} \rangle$ noto, si
determinano le forme esplicite degli altri stati.

Come esempio determiniamo i coeffidienti di Clabsh-Gordan nel caso in cui si
debbano sommare due spin con $j=1/2$. Quindi si ha che:
\begin{equation}
\begin{split}
\vec{J_{1}} =\vec{S_{1}} \quad \quad \vec{J_{2}} =\vec{S_{2}} & \Rightarrow \\
\Rightarrow \vec{J} = \vec{S} = \vec{S_{1}} + \vec{S_{2}} & \quad \quad \text{e} \quad \quad \vec{J}^{2} = \vec{S}^{2} = \vec{S_{1}}^{2} + \vec{S_{2}}^{2} + 2\vec{S_{1}}\vec{S_{2}}
\end{split}
\end{equation}
$\vec{S}$ si dice \textit{operatore di spin totale}. Consideriamo il caso in cui
i due sistemi (1 e 2) siano costituiti ciascuno da una particella di spin $1/2$.
Quindi si ha:
\begin{equation}
j_{1} = j_{2} = s_{1} = s_{2} = \dfrac{1}{2}
\end{equation}
\[
j = s =
\begin{cases}
\frac{1}{2} + \frac{1}{2} = 1 \\
\frac{1}{2} - \frac{1}{2} = 0
\end{cases}
j = j_{1} + j_{2}, \dots, \abs{j_{1} - j_{2}}
\]
dove $j = s_{1} + s_{2}$.

Quindi lo spettro di $S$ è costituito solo dai due valori $1$ e $0$. La
dimensione totale dello spazio è $\left(2j_{1} +1\right)\left(2j_{2} +1 \right)
= 2 \cdot 2 = 4$. I valori di $m_{j} = m_{s}$ sono:
\[
m_{j} = m_{s} = 
\begin{cases}
1, 0, -1 & \text{per $s=1$ \ dimensione 3} \\
0 & \text{per $s=0$ \ dimensione 1}
\end{cases}
\]
$m_{s}= 1,0,-1$, per $s=1$, si dice stato di \textit{tripletto}, con peso
statistico 3. $m_{s} = 0$, per $s=0$ si dice stato di \textit{singoletto}, con
peso statistico 1.

Indichiamo per semplicità:
\begin{align}
\alpha (1) & \equiv \mid m_{s_{1}} = +1/2 \rangle & \beta (1) & \equiv \mid m_{s_{1}} = -1/2 \rangle \\
\alpha(2) & \equiv \mid m_{s_{2}} = +1/2 \rangle & \beta (2) & \equiv \mid m_{s_{2}} = -1/2 \rangle 
\end{align}
Quindi gli stati della base disaccoppiata sono: 
\begin{align}
\alpha (1) \alpha (2), & \alpha (1) \beta (2), & \beta (1) \alpha (2), & \beta(1) \beta (2) 
\end{align}
Gli stati di tripletti e di singoletto si possono scrivere come combinazione
lineare di questi 4 stati di base.

Indichiamo gli stati della base accoppiata con
\[
  \chi_{s}^{(\text{tripletto})} (1,2) = 
\begin{cases}
\alpha (1) \alpha(2) & m_{s} = +1 \\
\dfrac{1}{\sqrt{2}}\left[\alpha(1) \beta(2) + \beta(1)\alpha(2)\right] & m_{s}
=0 \\
\beta(1)\beta(2) & m_{s} = -1
\end{cases}
\]
\begin{equation}
  \chi_{s}^{\text{(\text{singoletto})}} (1,2) = \dfrac{1}{\sqrt{2}}\left[ \alpha(1)\beta(2) - \beta(1)\alpha(2)\right] 
\end{equation}
Per ricavare questi coefficienti conviene ritornare al formalismo con i ket.
\begin{align}
\mid m_{j} = +1 \rangle = \sum _{m_{j_{1}}m_{j_{2}}} \langle m_{j_{1}}m_{j_{2}}
\mid m_{j} = 1 \rangle \mid m_{j_{1}}m_{j_{2}}\rangle & \quad \text{con $j=1,$
  $j_{1}=1/2,$ $j_{2}=1/2$}
\end{align}
In  questa sommatoria sono nulli tutti i termini tranne uno, quello per cui
$m_{j_{1}} + m_{j_{2}} = 1 \Rightarrow m_{j_{1}} = 1/2; m_{j_{2}} = 1/2$. Quindi
si può porre:
\begin{equation}
\mid m_{j} =1 \rangle \propto\, \mid m_{j_{1}} = \dfrac{1}{2}, m_{j_{2}} =
\dfrac{1}{2} \rangle
\end{equation}
Si può scegliere questa costante uguale a 1 (questa scelta renderà i
coefficienti di Clabsh-Gordan reali), cioè:
\begin{equation}
\mid m_{j} = 1 \rangle = \mid m_{j_{1}} = \dfrac{1}{2}, m_{j_{2}} =\dfrac{1}{2} \rangle
\end{equation}
Stesso discorso si può fare per $m_{j} = -1$ e si ottiene:
\begin{equation}
\mid m_{j} = -1 \rangle = \mid m_{j_{1}} = -\dfrac{1}{2}, m_{j_{2}} =-\dfrac{1}{2} \rangle
\end{equation}
Rimane da determinare i due stati con $m_{j} = 0$. Per fare ciò conviene usare i
due operatori $J_{+}$ e $J_{-}$:
\begin{align}
J_{+} &= J_{x} + iJ_{y} = J_{1+} + J_{2+} \\
J_{-} &= J_{x} - iJ_{y} = J_{1-} + J_{2-}
\end{align}
Applichiamo $J_{-}$ allo stato $m_{j} = +1$. Si ottiene che:
\begin{equation}
\begin{split}
J_{-}\mid m_{j} = +1 \rangle &= \left(J_{1-} + J_{2-}\right) \mid m_{j_{1}} = \dfrac{1}{2}, m_{j_{2}} = \dfrac{1}{2}\rangle = \\
&= J_{1-}  \mid m_{j_{1}} = \dfrac{1}{2}\mid m_{j_{2}} = \dfrac{1}{2}\rangle + J_{2-} \mid m_{j_{1}} = \dfrac{1}{2}\mid m_{j_{2}} = \dfrac{1}{2}\rangle = \\
&= \hbar \sqrt{\dfrac{1}{2}\dfrac{3}{2}-\dfrac{1}{2}\left(-\dfrac{1}{2}\right)} \mid m_{j_{1}} = -\dfrac{1}{2}, m_{j_{2}} = \dfrac{1}{2} \rangle + \hbar \sqrt{\dfrac{1}{2}\dfrac{3}{2}-\dfrac{1}{2}\left(-\dfrac{1}{2}\right)} \mid m_{j_{1}} = \dfrac{1}{2}, m_{j_{2}} = -\dfrac{1}{2} \rangle = \\
&= \hbar \left[ \mid m_{j_{1}} = -\dfrac{1}{2}, m_{j_{2}} = \dfrac{1}{2} \rangle + \mid m_{j_{1}} = \dfrac{1}{2}, m_{j_{2}} = -\dfrac{1}{2} \rangle \right]
\end{split}
\end{equation}
Ma questo deve essere uguale a:
\begin{equation}
J_{-} \mid m_{j} = +1 \rangle = \hbar \sqrt{2}\mid m_{j} = 0 \rangle
\end{equation}
Quindi abbiamo trovato che:
\begin{equation}
\mid m_{j} = 0 \rangle = \dfrac{1}{\sqrt{2}}\left[ \mid m_{j_{1}} = -\dfrac{1}{2}, m_{j_{2}} = \dfrac{1}{2} \rangle + \mid m_{j_{1}} = \dfrac{1}{2}, m_{j_{2}} = -\dfrac{1}{2}\rangle \right] 
\end{equation}
con $j=1$. 
\breaknote
Rimane\marginnote{12-1-1998} da ricavare i coefficienti di Clebsh-Gordan per lo
stato di singoletto
\begin{align}
\mid m_{j} = j = 0 \rangle &= a \mid m_{j_{1}} = \dfrac{1}{2}, m_{j_{2}} = -\dfrac{1}{2} \rangle + b \mid m_{j_{1}} = -\dfrac{1}{2}, m_{j_{2}} = \dfrac{1}{2} \rangle \\
\abs{a} ^{2} + \abs{b} ^{2} &= 1  \quad \quad \text{\textit{condizione di normalizzazione}}
\end{align}
Per ricavare $a$ e $b$ basta applicare l'operatore $J_{-} = J_{1-} + J_{2-}$:
\begin{equation}
\begin{split}
J_{-} \mid m_{j} = j = 0 \rangle = 0 &= a\left(J_{1-} \mid m_{j_{1}} = \dfrac{1}{2} \rangle \right) \mid m_{j_{2}} = -\dfrac{1}{2}\rangle + b\mid m_{j_{1}} = -\dfrac{1}{2} \rangle \left( J_{2-} \mid m_{j_{2}} = \dfrac{1}{2} \rangle \right) = \\
&= \hbar \left[a \mid m_{j_{1}} = -\dfrac{1}{2}, m_{j_{2}} = -\dfrac{1}{2}\rangle + b\mid m_{j_{1}} = -\dfrac{1}{2}, m_{j_{2}} = -\dfrac{1}{2}\rangle \right] 
\end{split}
\end{equation}
Quindi si può scegliere, ricordando anche la condizione di normalizzazione, 
\begin{align}
a &= \dfrac{1}{\sqrt{2}} & b &= -\dfrac{1}{\sqrt{2}}
\end{align}
a meno di un fattore di fase.

Se consideriamo la funzione d'onda di tripletto con le due particelle scambiate,
si ha che:
\[
\chi _{s} ^{(\text{tripletto})} (2,1) =
\begin{cases}
\alpha(2) \alpha(1) \\
\dfrac{1}{\sqrt{2}} \left[\alpha(2)\beta(1) + \beta(2)\alpha(1)\right] \\
\beta(2)\beta(1) 
\end{cases}
= \chi _{s} ^{(\text{tripletto})} (1,2)
\]
Quindi, \textit{$\chi _{s} ^{(\text{tripletto})}$ è simmetrica rispetto allo
scambio di 1 e 2}.

Per le funzioni di singoletto si ha che:
\begin{equation}
\chi _{s} ^{(\text{singoletto})} (2,1) = \dfrac{1}{\sqrt{2}} \left[ \alpha(2)\beta(1) - \beta(2)\alpha(1) \right] = -\chi _{s} ^{(\text{singoletto})} (1,2)
\end{equation}
quindi \textit{$\chi_{s} ^{(\text{singoletto})}$ è antisimmetrica rispetto allo
scambio di 1 e 2}.

Se i due fermioni, con spin $1/2$, 1 e 2 sono identici, cioè devono obbedire
alla statistica di Fermi-Dirac, supponendo che questi siano in uno stato
stazionario, si ha che
\begin{equation}
\Psi _\text{TOT} (1,2) = \chi _{s} (1,2) \phi (\vec{r_{1,2}})
\end{equation}
$\phi (\vec{r_{1,2}}) =$ funzione d'onda spaziale associata alla massa ridotta
del sistema (si sta lavorando nel sistema di rifermento del centro di massa)

Deve però essere verificato che:
\begin{equation}
\Psi _\text{TOT} (2,1) = - \Psi _\text{TOT} (1,2)
\end{equation}
in quanto le due particelle sono due fermioni identici. Dimostreremo in seguito
che
\begin{equation}
\phi (\vec{r_{2,1}}) = \phi (\vec{-r_{1,2}}) = (-1)^{l}\phi (\vec{r_{1,2}})
\end{equation}
dove $l$ è il numero quantico orbitale del sistema dei due fermioni. Quindi
tenendo conto di questo ulteriore risultato si arriva alla seguente
\textit{regola di selezione}:
\begin{itemize}
\item \textit{se i due fermioni si trovano nello stato di tripletto, $l$ deve essere dispari};
\item \textit{se i due fermioni si trovano nello stato di singoletto, $l$ deve essere pari}.
\end{itemize}
In maniera più compatta di può scrivere:
\begin{empheq}[box=% 
\fbox]{align*}
s &= 1 \quad \longleftrightarrow \quad \text{l \ dispari} \\
s &= 0 \quad \longleftrightarrow \quad \text{l \ pari} 
\end{empheq}
La verifica sperimentale della regola di selezione porta ad una verifica della
validità della statistica di Fermi-Dirac, dato che deve essere:
\begin{align}
& \Psi _\text{TOT} (1,2) = - \Psi _\text{TOT} (2,1) \\
& \chi _{s} ^{(\text{tripletto})} (1,2) = \chi _{s} (2,1) \\
& \chi _{s} ^{(\text{singoletto})} (1,2) = -\chi _{s} ^{(\text{singoletto})} (2,1) \\
& \phi (\vec{r_{2,1}}) = \phi (\vec{-r_{1,2}}) = (-1)^{l}\phi (\vec{r_{1,2}})
\end{align}


\part{Elementi di Fisica Nucleare}
\chapter{Esperimento di Rutherford e scoperta del nucleo atomico}
Il nucleo atomico fu scoperto nel 1911 da Rutherford. Egli mandò un fascio di
particella $\alpha$ contro un sottile strato metallico. Prima di questo
esperimento si pensava che l'atomo seguisse il modello a ``panettone''. Si
sapeva, però, che la quasi totalità della massa atomica era dovuta alle cariche
positive. Secondo questo modello \textit{una particella $\alpha$ con alta
energia e grande massa avrebbe dovuto subire una deflessione molto piccola}
nell'attraversare lo strato e, inoltre, questa deflessione non poteva certo
essere dovuta ad altre forze, oltre a quelle elettrostatiche. Rutherford
osservò, invece, \textit{grandi angoli di deflessione} che erano in buon accordo
con la sezione d'urto differenziale:
\begin{empheq} [box=%
\fbox] {align}
\dfrac{d\sigma}{d\Omega} &= \dfrac{1}{4}\left[\dfrac{ze(2e)}{m_{\alpha}v^{2}} \right] ^{2} \dfrac{1}{\sin ^{4}(\theta /2)} 
\end{empheq}
dove $d\Omega = 2 \pi \sin \theta \ d\theta$.

\begin{wrapfigure}{l}{7cm}
	\begin{tikzpicture}[>=triangle 45]
	\draw [dashed] (0,1) -- (6,1) node [circle,draw=black,fill=black,label=above:$ze$,inner sep=0pt,minimum size=1mm] {};
	\draw [dashed,name path=asseX] (0,0) node [circle,draw=black,fill=black,label=below:$2e$,inner sep=0pt,minimum size=1mm] {} -- (6,0);
	\draw [->] (1.5,1) -- node [label=right:$p$] {} (1.5,0);
	\node (2e) at (6,1) {};
	\path [dashed,name path=guida] (2e) to +(-120:5cm);
	\draw [name intersections={of=asseX and guida,by={centro}}, dashed] (centro) to +(-120:3cm) node (blu) {};
	\draw (centro) +(.5,0) arc (0:-120:.5cm);
	\path (centro) -- +(-70:1.1cm) node [label=$\theta$] {};
	\draw (0,0) .. controls (4.5,0) and (centro) .. (blu);
\end{tikzpicture}

	\caption{Diffusione di particelle di carica $2e$ da un centro diffusore di carica $ze$.}
\end{wrapfigure}
Questa è nota come \textit{formula di Rutherford} e si può ricavare teoricamente
in termini classici considerando un fascio uniforme di estensione infinita, di
particelle di carica $2e$, che viene diffuso da un centro fisso e puntiforme di
carica $ze$ (la formula vale anche se la carica $ze$ fosse negativa).

Assumere che \textit{il centro di diffusione sia fisso equivale ad assumere una
massa infinita}.
Il parametro $p$ rappresenta la distanza minima che la particella rangiungerebbe
dal centro nel caso in cui non si avesse diffussione. $p$ si dice
\textit{parametro d'urto}. Ovviamente l'unica cosa che si può misurare è il
numero di particelle deflesse in una particolare direzione nell'unità di tempo.

La forza è
\begin{equation}
F \propto \dfrac{2ze^{2}}{r^{2}}
\end{equation}
Quindi per $p$ piccoli, si avranno $\theta$ grandi e viceversa. Si può affermare
che verranno deflesse nell'angolo compreso tra $\theta$ e $\theta +d\theta$
tutte e sole le paarticelle che hanno un parametro d'urto $p$ compreso tra $p$ e
$p + dp$. Questo perchè \textit{$p(\theta)$ è monotona e, quindi, iniettiva}. Si
avrà che il numero di particelle che nell'unità di tempo viene deflesso
nell'angolo tra $\theta$ e $\theta + d\theta$ si può scrivere come:
\begin{equation}
  \frac{dN}{dt} = I2\pi p(\theta) \cdot dp = \dfrac{dN}{\text{sec}}
\end{equation}
Per definizione si ha che:
\begin{equation}
  d\sigma = \dfrac{dN/\text{sec}}{I}
\end{equation}
Quindi si ottiene che:
\begin{equation}
2\pi p(\theta) \cdot dp(\theta) = d\sigma
\end{equation}
Confrontando questa con la formula di Rutherford si deduce che:
\begin{equation}
2\pi p(\theta) \cdot \dfrac{dp(\theta)}{d\Omega} = \dfrac{1}{4}\left[\dfrac{ze(2e)}{m_{\alpha}v^{2}} \right] ^{2} \dfrac{1}{\sin ^{4} (\theta /2)}
\end{equation}
Da questa equazione differenziale si può trovare la funzione $p(\theta)$.

Quindi, da $\theta$ si può risalire a $p$. Supponiamo che la sezione d'urto
sperimentale sia in accordo con quella di Rutherford per $\theta$ compreso tra
$\theta _\text{min}$ e $\theta _\text{max}$. Da questo intervallo si può
risalire all'intervallo $\left[p_\text{min} (\theta _\text{max}), p_\text{max}
(\theta _\text{min}) \right]$. Nell'esperimento compiuto da Rutherford si aveva
che:
\begin{align}
p_\text{max} (\theta _\text{min}) & \sim 10^{-8} \ cm \\
p_\text{min} (\theta _\text{max}) & \sim 10^{-13} \ cm 
\end{align}
All'interno di questo intervallo l'accordo era ottimo. Per $p > p_\text{max}$ si
osservava un rapido decrescere della sezione d'urto sperimentale, in
contraddizione con la formula di Rutherford. Questi risultati portavano alla
conclusione che la carica positiva dell'atomo doveva essere concentrata in un
volume dell'ordine di $10^{-13}\ cm$ mentre nel modello a panettone si parlava
di $10^{-8} \ cm$. Il fatto che per $p$ grandi la sezione d'urto andava a zero,
si spiega dicendo che la carica efficace che provoca la deflessione andava a
zero in quanto si sovrapponeva alla carica del nucleo l'effetto di schermo degli
elettroni.
\chapter{Proprietà statistiche dei nuclei atomici}
Il numero atomico $Z$ \marginnote{14-1-1998} è uguale al numero di elettroni ed
è uguale anche al numero di cariche positive del nucleo. Quindi \textit{la
carica nucleare $ze$ è una proprietà intrinseca di un elemento. La massa di un
atomo influenza le proprietà chimiche di un elemento molto meno di quanto non
faccia la carica}. Infatti esistono gli isotopi. Questi sono caratterizzati
dallo stesso numero atomico $Z$ e da un diverso numero di massa $A$. Per
misurare le masse nucleari si usa \textit{lo spettrografo di massa}. Questo
strumento misura la massa di particelle cariche la cui carica sia nota. Sia $q$
la carica delle particelle di cui si vuole trovare la massa. Si deve, prima,
selezionare un fascio di queste particelle in modo che abbiano tutte una ben
determinata velocità: per far questo si sfrutta la carica della particella. La
forza su questa particella è:
\begin{equation}
\vec{F} = q\left[ \vec{E} + \dfrac{\vec{v}}{c}\wedge \vec{H} \right]
\end{equation}
Si regolano $E$ ed $H$ in modo che il moto sia rettilineo ed uniforme, cioè deve
essere:
\begin{equation}
\mid \vec{E} \mid = \dfrac{1}{c} \mid \vec{v} \wedge \vec{H} \mid
\end{equation}
Se $\vec{H}$ è perpendicolare a $\vec{v}$, si può scrivere:
\begin{equation}
\dfrac{\mid \vec{E} \mid}{\mid \vec{H} \mid} = \dfrac{v}{c}
\end{equation}
A questo punto se si utilizza il filtro così costituito,
\begin{figure}[!htbp]
	\centering
	\begin{tikzpicture}[>=triangle 45]
	\draw (0,.25) -- (0,2) -- (6,2) -- (6,.25);
	\draw (0,-.25) -- (0,-2) -- (6,-2) -- (6,-.25);
	\draw [->] (-3,0) -- (-.1,0);
	\draw [->] (6.1,0) -- (9,0);
	\node at (3,1) {\huge $\vec{E},\,\vec{H}$};
\end{tikzpicture}
	\caption{Filtro}
\end{figure}
le particelle che escono saranno tutte e solo quelle che hanno velocità pari a 
\begin{equation}
v = \dfrac{\mid \vec{E}\mid}{\mid \vec{H} \mid}c
\end{equation}
Se questo fascio viene poi immerso in un campo magnetico uniforme $\vec{H}$,
perpendicolare a $v$, allora le particelle descriveranno orbite circolari il cui
raggio è
\begin{equation}
r = \dfrac{v}{\omega _{s}} = \gamma (v) \dfrac{mcv}{q \mid \vec{H} \mid}
\end{equation}
dove $\omega _s (s) = \dfrac{1}{\gamma (v)} \dfrac{q \mid \vec{H} \mid}{mc}$.

Da queste \textit{si determina la massa $m$, misurando il raggio $r$}. Quando si
vuole misurare la massa di particelle neutre si usa lo spettrografo,
utilizzando, \textit{però, anche informazioni fornite da reazioni nucleari}. Ad
esempio per trovare la massa del neutrone si utilizza la reazione
\begin{equation}
n + p \longrightarrow d + \gamma
\end{equation}
dove $d$ è il nucleo nell'atomo di deuterio.
$p$ e $d$ hanno masse che si possono misurare, in quanto sono particelle dotate
di carica. Per la conservazione dell'energia deve essere
\begin{equation}
m_{n} + m_{p} = m_{d} + \dfrac{T_{d}}{c^{2}} + \dfrac{\hbar \omega}{c^{2}}
\end{equation}
dove si è supposto che $n$ e $p$ fossero a riposo prima della reazione. PEr la
conservazione della quantità di moto deve essere anche che la quantità di moto
del deuterio deve essere uguale a quella del fotone. Quindi si può scrivere per
le energie cinetiche
\begin{equation}
T_{d} = \dfrac{1}{2} \dfrac{p^{2}}{m_{d}} = \dfrac{1}{2} \dfrac{\hbar ^{2} \omega ^{2}}{m_{d} c^{2}}
\end{equation}
Quindi per \textit{la massa del neutrone} si ha:
\begin{equation}
m_{n} = m_{d} - m_{p} + \dfrac{\hbar \omega}{c^{2}} \left[1 + \dfrac{1}{2} \dfrac{\hbar \omega}{m_{d} c^{2}} \right]
\end{equation}
Ritornando alle proprità dei nuclei si osservò che la massa $A$ è sempre
maggiore di $Z$. La prima ipotesi che di fece fu che nel nucleo vi fossero $A$
protoni e $A-Z$ elettroni. Questo modello però non era in accordo con le misure
di massa dei nuclei.

Oggi si sa che, invece, \textit{nel nucleo ci sono $Z$ protoni ed $A-Z$
neutroni}. Atomi con lo stesso numero di neutroni di dirono \textit{isòtoni}.
Atomi con lo stesso numero di massa vengono detti \textit{isobari}. Atomi con
uguali $Z$ si dicono isotopi.

Riassumendo:
\begin{description}
  \item[Isotoni] atomi con lo stesso numero di neutroni N;
  \item[Isobari] atomi con lo stesso numero di massa A;
  \item[Isotopi] atomi con lo stesso numero atomico Z;
\end{description}
\textit{L'unità di massa nucleare} è la dodicesima parte della massa effettiva
del $\ce{^{12}C}$. Quindi si ha:
\begin{equation}
1 \ UM \simeq 1.66 \times 10^{-24} \ g \simeq 931.5016 \ MeV/c^{2}
\end{equation}
Da ciò segue che la massa del protone e del neutrone sono:
\begin{align}
m_{p} &\simeq 1.0073 \ UM \\
m_{n} &\simeq 1.0087 \ UM 
\end{align}
Indichiamo con $M$ la massa effettiva di un atomo. Sia $\Delta M$, invece, il
difetto di massa. Si ha che
\begin{equation}
\Delta E = \Delta Mc^{2} = \left( zm_{p} + Nm_{n} + zm_{e} - M \right) c^{2}
\end{equation}
Questa è l'energia di legame atomica. Si può anche scrivere
\begin{equation}
  \Delta E = \Delta E_{a} + \Delta E_\text{nucl}
\end{equation}
dove $\Delta E_{a}$ è l'energia di legame degli elettroni (energia di legame
atomica) e $\Delta E_\text{nucl}$ è l'energia di legame del nucleo. Se
$M_\text{nucl}$ è la massa effettiva del nucleo, si ha:
\begin{empheq}[box=%
\fbox]{align}
\Delta E_{a} &= \left( zm_{e} + M_\text{nucl} - M\right) c^{2} \\
\Delta E_\text{nucl} &= \left( zm_{p} + Nm_{n} - M_\text{nucl} \right) c^{2}
\end{empheq}
Sperimentalmente si verifica che:
\begin{empheq}[box=%
\fbox] {equation}
\Delta E_{a} \ll zm_{e}c^{2} \ll Mc^{2}
\end{empheq}
Tenuto conto di ciò, è lecito scrivere:
\begin{equation}
M_\text{nucl} = M - zm_{e} + \dfrac{\Delta E_{a}}{c^{2}} \simeq M - zm_{e} \simeq M
\end{equation}
Con queste approssimazioni si può scirvere per $\Delta E_\text{nucl}$:
\begin{equation}
\Delta E_\text{nucl} = \left( zm_{p} + Nm_{n} - M \right) 
\end{equation}
Sia $M(A,Z)$ la massa di un necleo. Si definice la quantità:
\begin{empheq} [box=%
\fbox] {equation}
f = \dfrac{M(A,Z) - A \ UM}{A \ UM}
\end{empheq}
\textit{frazione di impacchettamento}, che dà una misura di quanto sia il
difetto di massa. ($A \ UM$ è la massa che il nucleo avrebbe se ogni particella
pesasse $1 \ UM$)

Il \textit{raggio del nucleo} si può definire in diversi modi e dipende dal
metodo sperimentale usato. Di solito, però, si intende per raggio del nucleo
\textit{la distanza minimi oltra la quale il potenziale del nucleo si discosta
apprezzabilmente da quello coulombiano}. Rutherford per primo effettuà una
misura in questo senso e trovò:
\begin{equation}
R = r_{0}A^{1/3}
\end{equation}
dove $r_{0} \simeq 1,2 \times 10^{-13} \ cm$; $R$ è il parametro d'urto minimo;
$A$ è il numero di massa della sostanza bersaglio.

Si osservò che \textit{quando $p$ è minore di $R$ si hanno interazioni più
intense di quelle coulombiane}, il che fece pensare ad un contatto con il
nucleo. Esperimenti successivi, eseguiti con protoni e neutroni come particelle
incidenti (invece delle particelle $\alpha$), fornirono lo stesso risultato.

Questo stesso valore è stato ottenuto osservando la diffusione di un fascio di
neutroni (quindi non si fa più uso del potenzioale coulombiano) a causa
dell'interazione forte. Quindi è lecito scrivere:
\begin{empheq} [box=%
\fbox] {equation}
V(A) = \dfrac{4}{3}\pi r_{0} ^{3} A = \dfrac{4}{3} \pi R^{3}
\end{empheq}
cioè il volume nucleare risulta proporzionale al numero di massa. Dal momento
che $M(A) = A \ UM$, si trova che \textit{la densità è pressochè costante per
tutti i nuclei}:
\begin{equation}
  \rho_{m} = \dfrac{M(A)}{V(A)} = \dfrac{A \ UM}{V(A)} = \dfrac{3}{4\pi} \dfrac{1 \
  UM}{r_{0}^{3}} \simeq 10^{14} \ \text{g/cm}^{3}
\end{equation}
Da questa densità \textit{si può immaginare il nucleo costituito da un fluido
incomprimibile} (in quanto $\rho$ non dipende da $Z$).
\chapter{Fattore di forma}
In fisica nucleare \marginnote{16-1-1998} le misure sulla diffusione di
elettroni ad alta energia forniscono anche informazioni sulla distribuzione
della carica del nucleo. L'energia minima che gli elettroni devono avere è $1 \
GeV$. Consideriamo un elettrone libero che si trova in uno stato stazionario,
quindi la funzione d'onda è
\begin{align}
\phi _{e} &(\vec{r}) \propto e^{(i\vec{k} \cdot \vec{r})} \\
\vec{P_{e}} &= \hbar \vec{k} \Rightarrow \lambda _{e} = \dfrac{2\pi}{K} = \text{lunghezza d'onda di De Broglie}
\end{align} 
Se il nucleo fosse puntiforme, la funzione d'onda dell'elettrone avrebbe nel
nucleo valore $\phi _{e} (0)$. Se, invece, il nucleo non fosse puntiforme,
allora
\begin{align}
\phi _{e} (\vec{R}) & \neq \phi _{e}(0)  & R = \text{raggio del nucleo} \\
\phi _{e} (\vec{R}) & \simeq \phi _{e}(0) & \text{per} \ KR \ll 1
\end{align}
Dunque in questo secondo caso l'elettrone vede sempre un nucleo puntiforme.
Questo non è più vero quando $\lambda _{e}$ diventa più piccola. Quindi si ha:
\begin{empheq} [box=%
\fbox]{equation}
\text{\textcrlambda} = \dfrac{\lambda _{e}}{2\pi} = \dfrac{1}{K} \gg R \simeq
10^{-13} \ \text{cm} \Rightarrow \text{nucleo puntiforme}
\end{empheq}
Per cominciare a risolvere la distribuzione di carica nel nucleo
$\text{\textcrlambda}$ deve essere almeno di un ordine di grandezza più piccola
di $R$. Infatti \textit{l'energia di $1 \ GeV$ corrisponde a
  $\text{\textcrlambda} = 1,95 \times 10^{-14} \ cm$}. Supponendo che la
  diffusione sia elastica, vediamo cosa accadrebbe se il nucleo fosse puntiforme
  e fisso. In questo caso, dovrebbe valere in termini non quantistici la formula
  di Rutherford:
\begin{equation}
\dfrac{d\sigma _0}{d\Omega} = \left(\dfrac{d\sigma}{d\Omega}\right) _R = \left( \dfrac{ze^2}{2p_ev_e} \right) ^2 \dfrac{1}{\sin ^{4}(\theta /2)} 
\end{equation}
dove $d\Omega = 2 \pi \sin \theta \ d\theta$.
Questa formula è anche relativisticamente corretta per piccoli valori di
$\theta$, cioè quando si può approssimare
\begin{equation}
\sin ^{4} \dfrac{\theta}{2} \simeq \left(\dfrac{\theta}{2}\right)^{4}
\end{equation}
Passiamo, ora, ad una trattazione quantistica. L'hamiltoniana totale in questo
caso ideale si può scrivere come
\begin{equation}
H = H_{0} + V_{0}(r)
\end{equation}
dove $H_{0}$ è l'hamiltoniana imperturbata e $V_{0}(r)$ è il potenziale del
nucleo puntiforme fisso.

Sia $\phi (\vec{r})$ un'autofunzione di $H$ con autovalori $E$. Si può
dimostrare che $\phi (\vec{r})$ per $r$ molto grandi assume la forma asintotica
\begin{equation}
\phi (\vec{r}) \propto \left[e^{i\vec{k} \cdot \vec{r}} + f_{0}(\theta) \dfrac{e^{ikr}}{r}\right]
\end{equation}
per $r\rightarrow \infty$. 

Il primo termine è un autostato di $H_{0}$. Il secondo termine è un'onda sferica
uscente dall'origine la cui ampiezza dipende da $\theta$.
\textit{$f_{0}(\theta)$ ha le dimensioni di una lunghezza e si dice ampiezza di
scattering}. Infatti $\mid f_{0} (\theta) \mid ^{2} d\Omega$ è probabilità che
l'elettrone venga diffuso sull'elemento di angolo solido $\Omega$.

Quindi, si ha:
\begin{empheq}[box=%
\fbox] {equation}
d\sigma _{0} = \mid f_{0} (\theta) \mid ^{2} d\Omega 
\Rightarrow \dfrac{d\sigma _{0}}{d\Omega} = \mid f_{0} (\theta) \mid ^{2}
\end{empheq}
Questa coincide con quello di Rutherford se si fa l'approssimazione di Bohr per
potenziali deboli. Questa approssimazione è abbastanza cottetta se le particelle
incidenti hanno un'alta energia. 

Nella realtà, il nucleo non è puntiforme e, per distanze inferiori ad R, il
potenziale non è più coulombiano. Indichiamo con
\begin{equation}
  \dfrac{d\sigma}{d\Omega} = \abs{ f (\theta)}^{2}
\end{equation}
\textit{la sezione d'urto nell'ipotesi di nucleo fisso ma non puntiforme.}
Rimaniamo sempre nell'approssimazione di Born per potenziali deboli. Dal punto
di vista formale di può scrivere
\begin{equation}
\dfrac{d\sigma}{d\Omega} = \dfrac{d\sigma _{0}}{d\Omega} \dfrac{\mid f(\theta) \mid ^{2}}{\mid f_{0}(\theta) \mid ^{2}} = \dfrac{d\sigma _{0}}{d\Omega} \mid F \mid ^{2}
\end{equation}
dove si è posto $\mid F \mid ^{2} = \dfrac{\mid f(\theta) \mid ^{2}}{\mid
  f_{0}(\theta) \mid ^{2}}$.
$F$ è \textit{il fattore di forma elastico del nucleo.}

In generale si dice fattore di forma \textit{il rapporto fra l'ampiezza di
scattering con cariche estese e l'ampiezza di scattering nel caso in cui le
stesse cariche siano concentrate in un punto.} Ovviamente il fattore di forma
dipensa da come le cariche sono distribuite. Vediamo di ricordare, quindi, la
dipendeza di $F$ dalla distribuzione.

Indiachiamo con
\begin{align}
\vec{P_{e}} &= \hbar \vec{k} & \quad \text{quantità di moto iniziale} \\
\vec{P'_{e}} &= \hbar \vec{k'} & \quad \text{quantità di moto finale} 
\end{align}
Per l'ipotesi di nucleo fisso, deve essere $\mid \vec{P_{e}} \mid = \mid \vec{P'_{e}} \mid$ (per la conservazione dell'energia). Deve essere che
\begin{equation}
\mid \vec{P'_{e}} - \vec{P_{e}} \mid = 2P_{e} \sin (\theta /2)
\end{equation}
Il modulo di questo vettore coincide con il modulo dell'impulso trasferito al nucleo. Introduciamo la quantità
\begin{equation}
\vec{q} = \vec{k'} - \vec{k} \Rightarrow \mid \vec{q}\mid = \dfrac{2P_{e} \sin (\theta /2)}{\hbar}
\end{equation}
Definiamo la densità di carica
\begin{equation}
\rho _{c} (\vec{r}) \equiv ze \rho (\vec{r})
\end{equation}
dove la funzione $\rho (\vec{r})$ è tale che 
\begin{equation}
\int \rho (\vec{r}) d^{3}\vec{r} = 1
\end{equation}
Trascurando gli effetti magnetici, si può dimostrare che
\begin{empheq}[box=%
\fbox] {equation}
\dfrac{f(\theta)}{f_{0}(\theta)} = F = F(\vec{q}) = \int \rho (\vec{r}) e^{i\vec{q} \cdot \vec{r}} d^{3}\vec{r}
\end{empheq}
cioè \textit{$F$ è la trosformata di Fourier di $\rho(\vec{r})$}. Da questa formula si vede il legame tra $F$ e $\rho$.

Se si considera il nucleo non fisso, si può dimostrare che rimane valida l'espressione per $F$ appena riportata. 

Si ha, ovviamente, che
\begin{equation}
F(0) = 1
\end{equation}
Questo significa che se l'impulso trasferito al nucleo fosse nullo, risulterebbe:
\begin{equation}
\dfrac{d\sigma}{d\Omega} = \dfrac{d\sigma _{0}}{d\Omega} \mid F(0) \mid ^{2} = \dfrac{d\sigma _{0}}{d\Omega} = \text{sezione d'urto di Rutherford}
\end{equation}
\begin{equation}
\lim _{\vec{P_{e}} \rightarrow 0} \dfrac{d\sigma}{d\Omega} = \dfrac{d\sigma _{0}}{d\Omega}
\end{equation}
Quindi, \textit{se la quantità di moto dell'elettrone non è sufficientemente
grande l'elettrone vedrà il nucleo puntiforme}. Questa è, dunque, una buona
prova rigorosa di quanto detto in precedenza.

Il rapporto tra $\mid f(\theta) \mid ^{2}$ e $\mid f_{0} (\theta) \mid ^{2}$ è
sperimentalmente misurabile e, quindi, dalla determinazione di $\mid F \mid
^{2}$ si può risalire alla funzione $\rho (\vec{r})$.
