\chapter{Riferimenti - Formule e costanti numeriche}
\section{Masse}
\begin{table}[!h]
  \centering
  \caption{Dati relativi alle masse.}
  \begin{tabular}{>{\scshape}lr}
	\toprule
    $\gamma - 1$ elettrone          & 2$\div$20\\
	$\gamma - 1$ protone/neutrone   & $< 1$\%\\
	Difetto di massa $^4$He         & $\simeq 28$ MeV\\
	U.m.a.                          & $\simeq 931,5$Mev/c$^2 = 11,66\cdot 10^{-24}$g\\
	Densità nucleare                & $10^{14}$g/cm$^3$\\
    \bottomrule
  \end{tabular}
\end{table}


\begin{table}[!h]
  \centering
  \caption{Massa particelle (in MeV/c$^2$, eccetto il neutrino)}
  \subfloat[][]{
	\begin{tabular}{>{\scshape}lr}
	  \toprule
	  Elettrone ($e$) & 0,5\\
	  Protone ($p$)   & 938 ovvero 1,0073 \textsc{um}\\
	  Neutrone ($n$)  & 939 ovvero 1,0087 \textsc{um}\\
	  \bottomrule
	\end{tabular}
  }\\
  \subfloat[][]{
	\begin{tabular}{>{\scshape}lr}
	  \toprule
	  Neutrino $\nu$     & < 60 eV\\
	  Muone ($\mu$)      & 105\\
	  Muone ($\mu^-$)    & 106\\
	  Pione ($\pi$)      & 138\\
	  Quark up ($u$)     & 336\\
	  Quark down ($d$)   & 336\\
	  	  \bottomrule
	\end{tabular}
  }
  \subfloat[][]{
	\begin{tabular}{>{\scshape}lr}
	  \toprule
	  Mesone $K^0$         & 497,7\\
	  Quark strange ($s$) & 538\\
	  Eta ($\eta$°)       & 549\\
	  Iperone $\Lambda$   & 1115,6\\
	  Quark charm ($c$)   & 1650\\
	  Tauone ($\tau$)     & 1782\\
	  \bottomrule
	\end{tabular}
  }
\end{table}
\section{Energie}
\begin{empheq}[box=\fbox]{align*}
  h &= 6,626\cdot 10^{-34}J\,s = 4,135\cdot 10^{-15}eV\,s\\
  \hslash &= \frac{h}{2\pi} = 1,054\cdot 10^{-34}J\,s = 6,582\cdot
  10^{-16}eV\,s\\
  1 eV &= 1,6\cdot 10^{-19}J = 1,6\cdot 10^{-12}\text{erg}\\
  e^- &= 1,6\cdot 10^{-19}C
\end{empheq}

\begin{table}[!h]
  \centering
  \caption{Energie di soglia di creazione}
  \begin{tabular}{>{\scshape}l>{$}r<{$}}
	\toprule
	Elettrone-Positrone & \simeq 1,022 \text{MeV}\\
	Protone-Antiprotone & \simeq 1,9 \text{GeV}\\
	\bottomrule
  \end{tabular}
\end{table}

\begin{table}[!h]
  \centering
  \caption{Energia dei fotoni}
  \begin{tabular}{>{\scshape}rl}
	\toprule
	Microonde & $\sim$ 1 meV\\
	IR        & $\sim$ 1 eV\\
	Visibile  & $\sim$ 10 eV\\
	UV        & $\sim$ 1 keV\\
	Soft-X    & $\sim$ 10 keV\\
	Hard-X    & $\sim$ 1-100 MeV\\
	$\gamma$  & $\sim$ >100 MeV\\
	\bottomrule
  \end{tabular}
\end{table}

\section{Miscellanea}
\begin{description}
  \item[Sezione d'urto tipica] centibarn - barn (1 barn = $10^{-24}$ m$^2$)
  \item[Parametro d'urto esperimento di Rutherford] $p(\theta)_\text{min} =
	10^{-13}$cm e $p(\theta)_\text{max} = 10^{-8}$cm
  \item[$R$ parametro d'urto minimo] $R = r_0A^{1/3}$ dove $A$ è il numero
	atomico e $r_0 \simeq 1,2\cdot 10^{-13}$cm
  \item[Lunghezza d'onda elettronica] $E = 1$ GeV
	$\rightarrow$\textcrlambda$\simeq 1,95\cdot 10^{-14}$cm
  \item[Magnetone nucleare] $\mu_N = \dfrac{e\hslash}{2m_pc}\simeq
	0,505\cdot 10^{-23}$erg/gauss
\end{description}

\section{Formule}
\begin{description}
  \item[Energia di soglia]
	\begin{equation}
	  E_s = \frac{M_\text{min}^2 - M^2 - m^2}{2M}c^2
	\end{equation}
  \item[Energia di legame]
	\begin{equation}
	  \Delta E_0 = (E_{01} + E_{02}) - E_0 = \Delta M c^2
	\end{equation}
  \item[Sezione d'urto di Rutherford]
	\begin{equation}
	  \frac{\text{d}\sigma}{\text{d}\Omega} =
	  \frac{1}{4}\left[\frac{Ze(2e)^2}{m_\alpha v^2}\right]^2
	  \frac{1}{\sin^4(\theta/2)}
	\end{equation}
  \item[Frazione di impacchettamento]
	\begin{equation}
	  f = \frac{M(A,Z) - A\,UM}{A\,UM}
	\end{equation}
  \item[Fattore di forma] 
	\begin{gather}
	  \abs{F}^2 = \frac{\abs{f(\theta)}^2}{\abs{f_0(\theta)}^2}\\
	  F = F(\vec{q}) = \int \rho(\vec{r})e^{i\vec{q}\cdot\vec{r}}\text{d}^3r
	\end{gather}
  \item[Formula di Saxon] $c = 1,07A^{1/3}\cdot 10^{-13}$cm, $Z_1 = 0,545\cdot
	10^{-13}$cm, $\rho_1 =$ costante di normalizzazione
	\begin{equation}
	  \rho(r) = \frac{\rho_1}{e^{(r-c)/Z_1}+1}
	\end{equation}
  \item[Momento magnetico nucleare]
	\begin{equation}
	  \vec{\mu}_I = \frac{g_I\mu_N\vec{I}}{\hslash} = \gamma \vec{I}
	\end{equation}
  \item[Regola degli intervalli] 
	\begin{equation}
	  \frac{W_\alpha-W_{\alpha+1}}{W_{\alpha+1}-W_{\alpha+2}} =
	  f_\alpha/f_{\alpha+1}
	\end{equation}
  \item[Momento di quadrupolo elettrico del nucleo]
	\begin{equation}
	  Q = \int \rho_c^N(3Z^{\prime 2}-r^2)\text{d}V
	\end{equation}
  \item[Energia elettrostatica dovuta al quadrupolo]
	\begin{equation}
	  \Delta W = \frac{1}{4}\frac{\partial^2\varphi}{\partial
	  Z^2}Q\left( \frac{3}{2}\cos^2(\theta) - \frac{1}{2} \right)
	\end{equation}
  \item[Energia di quadrupolo elettrico semiclassica]
	\begin{equation}
	  \Delta W_Q = \frac{1}{4}\frac{\partial^2\varphi}{\partial Z^2}Q\left(
	  \frac{3}{2}K_f^2-2i^2j^2 \right)\frac{1}{4i^2j^2}
	\end{equation}
  \item[Energia di quadrupolo eletrico quantistica]
	\begin{equation}
	  \Delta W_Q = \frac{1}{4}\frac{\partial^2 \varphi}{\partial Z^2}Q\left(
	  \frac{3}{2}K_f(K_f+1)-2i(i+1)j(j+1)
	  \right)\frac{1}{i(2i-1)j(2j-1)}
	\end{equation}
  \item[Formula di Weizs\"acker] $a_\text{vol}=15,67\quad a_\text{sup} =
	17,23\quad 3a_c/5 = 0,7\quad a_\text{simm} = 93,15\quad \delta = 1,2$
	\begin{multline}
	  M(A,Z) = \left[ m_pZ + m_N(A-Z) \right] -a_\text{vol}A +
	  a_\text{sup}A^{2/3}+\\
	  +\frac{3}{5}a_c\frac{Z^2}{A^{1/3}} +
	  a_\text{simm}\frac{(A/2-Z)^2}{A} + \frac{\delta}{a^{1/2}}
	\end{multline}
  \item[Fattore di Gamow]
	\begin{equation}
	  G(T_\alpha,Z',R') =
	  \frac{2e^2Z'}{\hslash}\sqrt{\frac{2m_\alpha}{T_\alpha}} =
	  \frac{4}{\hslash}\sqrt{m_\alpha Z'R'}
	\end{equation}
  \item[Regola d'oro di Fermi]
	\begin{equation}
	  W = \frac{2\pi}{\hslash}\abs{H_{if}}^2\rho_f(E_i)\qquad \rho_f(E_i) =
	  \left.\frac{\text{d}N}{\text{d}E_f}\right|_{E_f=E_i}
	\end{equation}
  \item[Sezione d'urto di produzione della risonanza (no spin)]
	\begin{equation}
	  G^{(\ell)} = \frac{4\pi}{K^2_R}(2\ell +
	  1)\frac{\Gamma^2/4}{(E-E_R)^2+\Gamma^2/4}
	\end{equation}
  \item[Sezione d'urto di produzione della risonanza (con spin)]
	\begin{equation}
	  G^{(\ell)} = \frac{4\pi}{K^2_R}
	  \frac{(2J_R + 1)}{(2S_a + 1)(2S_b + 1)}\frac{\Gamma^2/4}{(E-E_R)^2+\Gamma^2/4}
	\end{equation}
\end{description}
\section{Reazioni nucleari}
\begin{table}[!h]
  \centering
  \caption{Alcune reazioni basilari.}
  \begin{tabularx}{\textwidth}{>{$}r<{$}X}
	\toprule
	\gamma + N \rightarrow e^+ + e^- + N & Creazione coppia elettrone -
	positrone\\
	n + p \rightarrow d + \gamma & Reazione usata per misurare la massa di $n$
	($d$ è il deutone)\\
	\text{U}^{238} \rightarrow \text{Th}^{234} + \alpha & $T_\alpha\simeq 4,2$ MeV\\
	N(A,Z) \rightarrow N(A,Z+1) + e^- + \bar{\nu}\quad(n\rightarrow p^+) &
	Decadimento $\beta^-$\\
    N(A,Z) \rightarrow N(A,Z-1) + e^+ + \nu\quad(p^+\rightarrow n) &
	Decadimento $\beta^+$\\
	p^+ + e^- \rightarrow n + \nu & Processo di cattura $K$\\
	\pi^- + p^+ \rightarrow \Lambda^0 + K^0 & Processo di produzione (associata)
	di $\Lambda^0$ e $K^0$\\
	\Lambda^0 \rightarrow  p^+ + \pi^-,\quad K^0 \rightarrow \pi^+ + \pi^- &
	Decadimenti deboli di $\Lambda^0$ e $K^0$\\
	\pi^- + p^+ \rightarrow K^+ + K^- + n & Processo di produzione di $K^+$\\
	K^- + p^+ \rightarrow \bar{K}^0 + n & Produzione di $\bar{K}^0$\\
	K^- + N \rightarrow \pi + \Lambda^0, \quad K^- + N \rightarrow \pi + \Sigma &
	Assorbimento del mesone $K^-$\\
	Kì- + p^+ \rightarrow \eta^0 + \Lambda^0 & Produzione del mesone $\eta^0$\\
	\pi^- \rightarrow \mu^- + \bar{\nu}_\mu & Produzione del muone\\
	\mu^- \rightarrow e^- + \bar{\nu}_e + \nu_{\mu} & Decadimento debole del
	muone\\
	\bottomrule
  \end{tabularx}
\end{table}

\begin{table}[!h]
  \centering
  \caption{Processi di produzione mesone $\pi$}
  \begin{tabular}{>{$}l<{$}|>{$}c<{$}|>{$}r<{$}}
	\toprule
	p^+ + p^+ \rightarrow p^+ + p^+ + \pi^0 & n + p^+ \rightarrow n + p^+ +
	\pi^0 & p^+ + n \rightarrow d + \pi^0\\
    p^+ + p^+ \rightarrow p^+ + n + \pi^+ & n + p^+ \rightarrow p^+ + p^+ +
	\pi^- & \gamma + p^+ \rightarrow p^+ + \pi^0\\
    p^+ + p^+ \rightarrow d + \pi^+ & n + p^+ \rightarrow n + n + \pi^+ & \gamma
	+ p^+ \rightarrow n + \pi^+\\
	\bottomrule
  \end{tabular}
\end{table}

\begin{table}[!h]
  \centering
  \caption{Processi di produzione degli iperoni}
  \begin{tabular}{>{$}l<{$}|>{$}c<{$}|>{$}r<{$}}
	\toprule
	K^- + p^+ \rightarrow \pi^+ + \Sigma^- &K^- + p^+ \rightarrow \pi^0 +
	\Sigma^0 &K^- + p^+ \rightarrow \pi^0 + \Sigma^+\\
    K^- + p^+ \rightarrow K^+ + \Xi^- &K^- + p^+ \rightarrow K^0 + \Xi^0 &K^- + p^+
	\rightarrow K^0 + \Omega^- + K^+ \\
	\bottomrule
  \end{tabular}
\end{table}

\begin{table}[!h]
  \centering
  \caption{Processi di decadimento (non leptonici) del mesone $K^+$}
  \begin{tabular}{>{$}l<{$}|>{$}c<{$}|>{$}r<{$}}
	\toprule
  K^+ \rightarrow \pi^+ + \pi^0 & K^+ \rightarrow \pi^+ + \pi^+ + \pi^- & K^+
  \rightarrow \pi^+ + \pi^0 + \pi^0\\
	\bottomrule
  \end{tabular}
\end{table}
