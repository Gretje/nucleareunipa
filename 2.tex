%!TEX root = nucleare.tex
% fino a pag circa 41
\section{Spazio di Hilbert}

Supponiamo \marginnote {1-12-1997}
di avere un sistema quantistico costituito da una particella puntiforme e
descritto da una funzione d'onda scalare.
Sia essa $\phi(\vec{r},t)$. Il sistema risulta definito all'istante \textit{t}
da questa funzione. Poniamo:
$\phi(\vec{r},t) =\phi_{p}(t)$
dove $p$ è il punto individuato da $\vec{r}$. Quindi l'insieme dei valori
${\phi(t)}$ descrive ed individua il nostro sistema.

Lo stato all'istante $t$ si può geometricamente rappresentare come un
\textit{vettore appartenente ad uno spazio di dimensione infinita} le cui
componenti siano le $\phi_{p}(t)$. Questo spazio si dice \textit{Spazio di
Hilbert} o spazio degli stati ed i corrispondenti vettori si dicono vettori
dello stato.

Un generico vettore di questo spazio si indica con:

\begin{equation}
 \ket{t}  \qquad \text{KET}
\end{equation}%(nota dell'autore: "picciò˜ non ho idea di come si faccia, 
spero sia venuto bene, ah e non so manco mettere le note!")

questo descrive lo stato del sistema all'istante $t$.
Supponiamo ora che all'istante $t$ il sistema sia localizzato in un certo punto
$p'$ (cioè la funzione d'onda è nulla in tutti i punti diversi da $p'$), 
quindi
si ha:

\[
\phi_{p}(t)=0 \quad \text{per } p\neq p'
\]

Il ket che individua questo stato si indica con 
$\ket{\vec{r'},t}$

Consideriamo ora l'insieme di tutti questi ket al variare di $\vec{r'}$, questa
è una buona base o un insieme completo, dello spazio di Hilbert. Questa base è
ortogonale.

Per quanto detto un generico ket $\ket{t}$ si può esprimere in funzione della
base ortogonale $\ket{\vec{r'},t}$, quindi le componenti $\phi_{p}(t)$  sono
ortogonali e si ha che:
\begin{equation}
\int \psi^*_{p}(t)\psi_{p}(t) d^3\vec{r}= \int 
\psi^*(\vec{r},t)\psi(\vec{r},t)d^3\vec{r}%la formula è uguale a quella degli 
appunti...
\end{equation}
cioè il modulo quadro del vettore $\ket{t}$ è la somma (integrale) dei moduli
quadri delle componenti.

Si definisce anche il prodotto scalare fra due stati partendo dal fatto che il
modulo quadro di $\ket{t}$ si può interpretare come il prodotto scalare con se
stesso. Questo si indica con:
\begin{equation}
\braket{t|t} \quad \text{dove} \bra{t} \text{  si dice BRA}
\end{equation}

Quindi ad ogni vettore $\ket{t}$ si associa il suo duale $\bra{t}$ e si ha la
corrispondenza biunivoca\footnote{Sotto certe condizioni. }:
\begin{equation}
\ket{t}\equiv \{\psi_{p}(t)\}\longrightarrow \bra{t} \equiv\{\psi^*_{p}(t)\}
\end{equation}

Quindi il prodotto scalare fra il vettore $\ket{t,a}$ e $\ket{t,b}$ si definisce
come:
\begin{equation}
\braket{t,b|t,a}=\int\psi^{*(b)}_p(t)\psi^{(a)}_p(t) d^3\vec{r}
\end{equation}

Per quanto definito si ha che:
\begin{equation}
\braket{t,a|t,b}=\braket{t,b|t,a}^*
\end{equation}

Due vettori si dicono ortogonali quando il loro prodotto scalare è nullo.
Ovviamente la condizione di ortogonalità dovrà valere per i vettori di base.
Questi vettori di base possono essere definiti in modo che la base risulti
ortonormale, questo, per definizione, significa che:
\begin{equation}
\psi_p(t)\equiv \braket{\vec{r},t|t}
\end{equation}
cioè quando per una base succede questo la base si dice ortonormale.

Se $\ket{t}$ è lo stato che descrive il nostro sistema e se questo stato è
normalizzato, cioè $\braket{t|t}=1$, allora si ha che:
\begin{equation}
|\braket{\vec{r},t|t}|^2=\psi^*(\vec{r},t)\psi(\vec{r},t)
\end{equation}
Questa è la densità di probabilità trovare la particella nel punto$\vec{r}$
all'istante $t$. Il generico $\ket{t}$
 si può esprimere in funzione della base, secondo la formula:
 
\begin{equation}
\ket{t}= 
\int{\psi_{p}\braket{\vec{r},t}d^3\vec{r}}=\int{\braket{\vec{r},t|t}}\ket{\vec{r
},t}d^3\vec{r}
\end{equation}

Determiniamo quale condizione devono soddisfare i vettori di base affinché
effettivamente la base risulti ortonormale. Deve essere verificato che:

\begin{equation}
\braket{\vec{r},t|t}=\int{\psi _{p}(t) \braket{\vec{r}',t
|\vec{r},t}d^3\vec{r}}=\psi_{p'}(t)
\end{equation}
da questa segue:

\begin{empheq}[box=\fbox]{equation}
\braket{\vec{r},t|\vec{r},t}=\delta^3(\vec{r}-\vec{r'})
\end{empheq}

Con questa posizione i vettori di base risultano automaticamente ortogonali, la 
$\delta^3$ si può esprimere nella forma fattorizzata:
\begin{equation}
\delta^3(\vec{r}-\vec{r'})=\delta(x-x')\delta(y-y')\delta(z-z')
\end{equation}
\breaknote

Tutte \marginnote{3-12-1997} le grandezze osservabili sono rappresentate da 
operatori  \textit{hermitiani} che agiscono sui Ket.
A questi operatori si associano matrici (anch'esse hermitiane) di dimensione 
infinita.

Si ha che:
\begin{equation}
A\ket{t,1}=\ket{t,2}
\end {equation}
dove $A$ è l'operatore hermitiano e $\ket{t,1}$ e $\ket{t,2}$ sono due Ket le 
cui componenti sono:
\begin{align}
\ket{t,1}&=\int\psi_{p}^{(1)}(t)\ket{\vec{r},t}d^3\vec{r}\\
\ket{t,2}&=\int\psi_{p}^{(2)}(t)\ket{\vec{r},t}d^3\vec{r}
\end{align}

Per definizione si ha:
\begin{equation}
\psi_{p'}(t)=\braket{\vec{r'},t|t,2}=\braket{\vec{r'},t|A|t,1}=
 \int\psi_{p}^{(1)}(t)\braket{\vec{r'},t|A|\vec{r},t}d^3\vec{r}=
 \int\psi_{p}^{(1)}(t)A_{p'p}d^3\vec{r}
\end{equation}
 dove si è posto:
 \begin{equation}
 A_{p'p}=\braket{\vec{r'},t|A|\vec{r},t}\qquad \textit{Elementi di matrice di A}
 \end{equation}
 
Così definiti gli elementi di matrice sono relativi alla base.
Quando si cambia base questi in genere cambiano. Se si prende come base quella 
degli autovalori di A allora l'espressione matriciale assume forma diagonale, 
con gli elementi reali. 
In generale lo spettro degli autovalori di un operatore può essere sia 
discreto, sia continuo.
\subsection{Spettro discreto}
$a_{k}$ sia l'autovalore, $\ket{a_{k},t}$ sia il corrispondente autostato (per 
semplicità si assume che non vi sia degenerazione).

Quindi, per definizione, si ha:

\begin{equation}
A\ket{a_{k},t}=a_{k}\ket{a_{k},t}
\end{equation}

Un generico stato $\ket{t}$ si potrà scrivere come:

\begin{equation}
\ket{t}=\sum_{k}c_{k}(t)\ket{a_{k},t}
\end{equation}
e con la normalizzazione $\braket{a_{k},t|a_{k'},t}=\delta_{kk'}$ si ha che :

\begin{equation}
c_{k}(t)=\braket{a_{k},t|t}
\end{equation}

La quantità:

\begin{equation}
|\braket{a_{k},t|t}|^2=|c_{k}(t)|^2
\end{equation}
fornisce la probabilità che una misura della grandezza A fornisca il valore 
$a_{k}$ quando il sistema si trova nello stato$ \ket{t}$.

La quantità:
\begin{equation}
\braket{t|A|t}     \qquad \textit{Valore medio o valore di aspettazione}
\end{equation}
si dice valore medio di A rispetto allo stato$\ket{t}$. Si verifica facilmente 
che:

\begin{equation}
\braket{t|A|t}=\sum_{k}|\braket{a_{k},t|t}|^2a_{k}= \sum_{k}|c_{k}|^2a_{k}
\end{equation}

\subsection{Spettro continuo}
Sia $a$ l'autovalore e $\ket{a,t}$ il corrispondente autovettore. Un generico 
ket $\ket{t}$ si potrà scrivere come:

\begin{equation}
\ket{t}=\int\braket{a,t|t}\ket{a,t}da
\end{equation}

\begin{equation}
\braket{a,t|a',t}=\delta(a-a')
\end{equation}

\begin{equation}
|\braket{a,t|t}|^2=P(a)
\end{equation}

L'ultima quantità scritta rappresenta la densità di probabilità della 
distribuzione degli autovalori relativa allo stato $\ket{t}$. Cioè la 
probabilità di ottenere con una misura un valore compreso tra $a$e $a+da$ è 
data da $|\braket{a,t|t}|^2$.

\subsection{Spettro quasi continuo}
Definiamo ora la situazione di spettro quasi continuo. Consideriamo un sistema
che sia confinato in un volume grande ma finito e tale che una funzione d'onda
assuma valori uguali sulla frontiera di questo insieme. Sotto queste ipotesi si
può passare dal caso continuo al caso discreto:

\begin{equation}
a \longleftrightarrow a_k
\end{equation}

\begin{equation}
\int|\braket{a,t|t}|^2da \longleftrightarrow \sum_{k}|\braket{a_{k},t|t}|^2
\end{equation}
cioè si ha il passaggio da una densità di probabilità ad una probabilità. La
distanza fra gli autovalori $a_k$si può rendere piccola aumentando il volume$V$
che racchiude il sistema. Per un $V$ fissato si può definire la densità di
autostato:
\begin{equation}
\rho(a_k)=\frac{\Delta n}{\Delta a_{k}}
\end{equation}
dove $\Delta n$ è il numero di autostati relativo all'intervallo $\Delta a_k$.
Con una opportuna scelta del volume $V$ gli autovalori possono essere trattati
come quantità continue, in quanto, la loro distanza risulta molto piccola. In
questo caso la densità di autostati diventa:
\begin{equation}
\rho(a_k)=\frac{\delta n}{\delta a_k}\quad \textit{con }\delta a_k\ll a_k
\end{equation}

Approssimando la quantità $\delta a_k$ ad un infinitesimo si dovrà parlare
quindi di probabilità di trovare un autovalore nell'intervallo di estremi
$a_k$ e $a_k+da_k$. Questa probabilità è:
\begin{equation}
|\braket{a_k,t|t}|^2\delta n
\end{equation}
dove $\delta n= \rho(a_k)\delta a_k$ è il numero di autostati i cui autovalori
sono compresi nell'intervallo $[a_k,a_k+\delta a_k]$. Con questa approssimazione
si ha:
\begin{equation}
\sum_k|\braket{a_k,t|t}|^2
\end{equation}
da cui:
\begin{equation}
\int|\braket{a_k,t|t}|^2\rho(a_k)\delta a_k
\end{equation}

Questa quindi è la riduzione di uno spettro continuo ad uno quasi continuo. In
pratica quello che si fa è l'approssimazione:

\begin{equation}
|\braket{a,t|t}|^2\simeq |\braket{a_k,t|t}|^2\rho(a_k)\delta a_k
\end{equation}

Ovviamente se si fa tendere $V$ ad infinito si deve avere:
\begin{equation}
\lim_{V\to\infty}|\braket{a_k,t|t}|^2\rho(a_k)\delta a_k = |\braket{a,t|t}|^2da
\end{equation}
L'utilità di questa approssimazione risulterà chiara quando si affronterà la 
teoria di Fermi.

\subsection{Operatore coniugato Hermitiano}
Sia $G$ un operatore lineare che agisce nello spazio di Hilbert. Si definisce 
$G^\dag$, \textit{operatore coniugato Hermitiano}, come:

\begin{gather}
\braket{t,2|G^{\dag}|t,1}=(\braket{t,1|G|t,2})^*\\
\ket{t,3}=G\ket{t,2}\\
\braket{t,2|G^{\dag}|t,1}=(\braket{t,1|G|t,2})^*=(\braket{t,1|t,3})^*=\braket{t,
3|t,1}
\end{gather}
Quindi per $G^{\dag}$ e $G$ vale la proprietà:

\begin{equation}
\ket{t,3}=G\ket{t,2} \Longrightarrow \bra{t,3}=\bra{t,2}G^{\dag}
\end{equation}

Un operatore $G$ si dice \textit{Hermitiano} quando $G=G^{\dag}$. In questo 
caso se $\ket{g}$ è un autostato normalizzato di $G$ con autovalore $g$ si ha 
che:

\begin{equation}
\braket{g|G|g}=g 
\end{equation}

\begin{equation}
\braket{g|G^{\dag}|g}=g^*\rightarrow g=g*
\end{equation}
quindi gli autovalori di un operatore hermitiano sono tutti reali.

Un operatore $G$ si dice \textit{unitario} quando $G=G^{\dag}$\footnote{Il 
testo riporta così: $G=G^{-1}$}. Un operatore unitario conserva i moduli dei 
ket. Infatti:

\begin{equation}
\braket{t|G^{\dag}G|t}=\braket{t|t}
\end{equation}

Nel caso in cui lo spazio di Hilbert sia relativo ad un sistema di $N$ 
particelle i ket si scrivono come:

\begin{equation}
\ket{t}=\int\psi(\vec{r_1},\vec{r_2},\vec{r_3},...,\vec{r_N},t)\ket{\vec{r_1},\v
ec{r_2},\vec{r_3},...,\vec{r_N},t}d^3\vec{r_1}d^3\vec{r_2}...d^3\vec{r_n}
\end{equation}
dove si è posto 
$\ket{\vec{r_1},\vec{r_2},\vec{r_3},...,\vec{r_N},t}=\ket{\vec{r_1},t}\ket{\vec{
r_2},t}...\ket{\vec{r_N},t}$.
Il prodotto scalare si definisce come:

\begin{equation}
\braket{t|t}=\int\psi^N(\vec{r_1},\vec{r_2},...,\vec{r_N},t)\psi(\vec{r_1},\vec{
r_2},...,\vec{r_N},t)d^3\vec{r_1}...d^3\vec{r_N}
\end{equation}


\section{Rappresentazione di Schr\"{o}dinger}
Consideriamo un sistema quantistico descritto dalla funzione d'onda
$\psi(\vec{r},t)=\psi_{p}(t)$. Questa funzione soddisfa l'equazione di
Schr\"{o}dinger:

\begin{equation}
i\hslash\frac{\partial}{\partial t}\psi_{p}(t)=H\psi_{p}(t)
\end{equation}

Questa è equivalente all'equazione vettoriale:

\begin{equation}
i\hslash\frac{d}{dt}\ket{t}=H\ket{t}
\label{eq:rapp_Sch}
\end{equation}
con $\ket{t}=\int\psi_{p}(t)\cdot \ket{\vec{r},t}d^{3}\vec{r}$.
In realtà si è assunto che i vettori di base $\ket{\vec{r},t}$ non varino al
variare di $t$, cioè:
\begin{equation}
\frac{\partial}{\partial t}\ket{\vec{r},t}=0 
\end{equation}
se $\ket{\vec{r},t}=\ket{\vec{r}}$

La formulazione della meccanica quantistica basata sull'equazione
\eqref{eq:rapp_Sch} si dice \textit{rappresentazione di Schr\"{o}dinger}.

Esistono comunque altre rappresentazioni ed in particolare ne esistono un numero
infinito, tutte equivalenti. Questo perché non si possono misurare nè le
funzioni d'onda, nè gli operatori. Quello che si può misurare sono gli
autovalori e le probabilità associate.
Quindi ogni altra rappresentazione che lasci invariate queste due grandezze
andrà bene.
Se applichiamo una trasformazione unitaria a tutti gli elementi delllo spazio di
Hilbert si ottiene ad esempio una rappresentazione equivalente.
Questa trasformazione unitaria si definisce come:
\begin{equation}
  U\ket{t}=\ket{\vec{t}}  \text{ e } \bra{t}U^{\dag}=\bra{\vec{t}} \quad 
\text{con} UU^{\dag}=U^{\dagger}U=1
\end{equation}
Questa è una generica trasformazione unitaria. Definiamo:
\begin{equation}
U(A\ket{t})=\overrightarrow{A\ket{t}}=\vec{A}\ket{\vec{t}}
\end{equation}
dove $\vec{A}$ è l'operatore trasformato. Di questo si può dare una
rappresentazione esplicita:
\begin{equation}
\vec{A}\ket{\vec{t}}=UA\ket{t}=UAU^{\dag}U\ket{t}=UAU^{\dag}\ket{\vec{t}}
\end{equation}
quindi:
\begin{equation}
\vec{A}=UAU^{\dag}
\end{equation}

Se $A$ rappresenta una grandezza fisica e se $\ket{a_{k},t}$ è un suo autostato
con autovalore $a_{k}$ allora si ha che:
\begin{equation}
  
\vec{A}\ket{\overrightarrow{a_{k}t}}=U(A\ket{a_{k}t})=a_{k}\ket{\overrightarrow{
a_{k}t}}
\end{equation}
Quindi la trasformazione $U$ lascia invariati gli autovalori, e l'autostato di
$\vec{A}$ è il trasformato dell'autostato di $A$. Dal momento che:
\begin{equation}
\braket{\overrightarrow{a_{k}t}|\vec{t}}=\braket{a_{k}t|t}
\end{equation}
rimane invariata la probabilità di trovare in una misura l'autovalore $a_{k}$.
Abbiamo dunque dimostrato che questa è una rappresentazione equivalente.

Le rappresentazioni più comunemente usate sono quella di Heisenberg e quella di
interazione.

\section{Rappresentazione di Heisenberg}
Per introdurre questa rappresentazione partiamo dalla rappresentazione di
Schr\"{o}dinger. Consideriamo lo stato all'istante $t$.
Si può porre
\begin{equation}
\ket{t}= T (t,t_{0})\ket{t_{0}} 
\end{equation}
dove $T$ è l'operatore di sviluppo temporale e ovviamente $T(t_{0},t_{0})=1$.
Utilizzando l'equazione di Schr\"{o}dinger si verifica che:
\begin{equation}
i\hslash\frac{d}{dt}T=HT
\end{equation}
$T$ si può formalmente scrivere come:
\begin{equation}
T(t,t_{0})=\exp\left\{-\frac{i}{\hslash}H(t-t_{0})\right\}
\end{equation}
se $H$ è hermitiano allora $T$ risulta unitario. In queste ipotesi si conserva
la probabilità, infatti:
\begin{equation}
\braket{t|t}=\braket{t_{0}|T^{\dag}(t,t_{0})T(t,t_{0})|t_{0}}=\braket{t_{0}|t_{0
}}
\end{equation}
nel caso quasi stazionario di un sitema che decade $H$ non è hermitiano, in
quanto ha autovalori complessi. In questo caso $T$ non è più unitario e non è
garantita la conservazione della probabilità. In ogni caso comunque $T$ commuta
con $H$, cioè:
\begin{equation}
[T,H]=TH-HT=0
\end{equation}
Quindi la rappresentazione di Heisemberg si ottiene considerando la
trasformazione unitaria:

\begin{empheq}[box=\fbox]{equation}
U=T^{\dag}(t,t_{0})=\exp\left\{\frac{i}{\hslash}H(t-t_{0})\right\}
\end{empheq}
cioè si deve applicare questa trasformazione alla rappresentazione di
Schr\"{o}dinger. Questo ovviamente si può fare solo se $H$ è hermitiana.
L'operatore $T^{\dag}$ verifica l'equazione:

\begin{equation}
-i\hslash\frac{d}{dt}T^{\dag}=T^{\dag}H=HT^{\dag}
\end{equation}
Il nuovo stato del sistema sarà:

\begin{equation}
\ket{\vec{t}}=T^{\dag}\ket{t}=\ket{t}_{H}=T^{\dag}T\ket{t_{0}}
\end{equation}
Inoltre si ha che:

\begin{equation}
\ket{t}_{H}=\ket{t_{0}}_{H}=\ket{t_{0}}
\end{equation}
da cui:

\begin{empheq}[box=\fbox]{equation}
\frac{d}{dt}\ket{t}_{H}=0
\end{empheq}

Quindi in questa rappresentazione lo stato rimane costante nel tempo, mentre
invece variano gli operatori. Supponiamo che l'operatore $A$ associato ad una
grandezza fisica non dipenda esplicitamente dal tempo. Dimostriamo che:

\begin{equation}
\frac{d}{dt}A=\frac{\partial}{\partial t}A=0
\end{equation}
Per ipotesi $\frac{\partial A}{\partial t}=0$.
Poniamo (rappresentazione di  Schr\"{o}dinger):

\begin{equation}
\begin{split}
\ket{t}=\int\psi_{p}(t)\ket{\vec{r}}d^{3}\vec{r} \\
A\ket{t}=\int\psi^{I}_{p}(t)\ket{\vec{r}}d^{3}\vec{r}
\end{split}
\end{equation}
dove:

\begin{equation}
\psi ^{I}_{p}(t)=\int A_{pp^{I}} \psi _{p^{I}}(t) d^{3} \vec{r}
\end{equation}
Derivando rispetto al tempo:

\begin{equation}
\begin{split}
\frac{d}{dt}\ket{t}=\int \frac{\partial}{\partial 
t}\psi_{p}(t)\ket{\vec{r}}d^{3}\vec{r}\\
\frac{d}{dt}[A\ket{t}]=\int\frac{\partial}{\partial t} 
\psi^{I}_{p}(t)\ket{\vec{r}}d^{3}\vec{r}
\end{split}
\end{equation}
dove si ha che:

\begin{equation}
\frac{\partial}{\partial t} \psi^{I}_{p}(t)=\int A_{pp^{I}} 
\frac{\partial}{\partial t} \psi_{p^{I}}(t)d^{3} \vec{r}
\end{equation}
Da tutto questo segue che:

\begin{gather}
\frac{d}{dt}[A\ket{t}]=A\frac{d}{dt}\ket{t} \rightarrow \frac{d}{dt}A=0
\end{gather}

Tutto questo è stato valutato nella rappresentazione di Schr\"{o}dinger. 
Vediamo
ora cosa succede nella rappresentazione di Heisenberg:

\begin{gather}
\vec{A}=A_{\mathscr{H}}=T^{\dagger}AT
\end{gather}

Valutiamo ora la derivata totale rispetto al tempo:

\begin{equation}
\begin{split}
\frac{d}{dt}A_{\mathscr{H}} &
=\left(\frac{d}{dt}T^{\dagger}\right)AT+T^{\dagger}A\left(\frac{d}{dt}T\right) 
\\
& = \frac{i}{\hslash}HT^{\dagger}AT-\frac{i}{\hslash}T^{\dagger}ATH \\
& = -\frac{i}{\hslash}(A_{\mathscr{H}}H-HA_{\mathscr{H}}) \\
& = -\frac{i}{\hslash}[A_{\mathscr{H}},H]
\end{split}
\end{equation}
Quindi l'operatore $A_{\mathscr{H}}$ verifica l'equazione:

\begin{empheq}[box=\fbox]{equation}
i\hslash \frac{d}{dt}A_\mathscr{H}=[A_\mathscr{H},H]
\end{empheq}
Questa vale nell'ipotesi che $\frac{\partial A}{\partial t}=0$. Se $A=H$ si 
ottiene che:

\begin{gather}
A_\mathscr{H}=H_\mathscr{H}=T^{\dag}HT=T^{\dag}TH=H\rightarrow H_\mathscr{H}=H
\end{gather}

Quindi anche nel sistema di Heisenberg:

\begin{equation}
\frac{d}{dt}H_\mathscr{H}=0
\end{equation}
se $\frac{\partial}{\partial t}H_\mathscr{H}=0$.

Il fatto che in questa rappresentazione lo stato di un sistema rimane costante 
implica che sono i vettori di base a variare nel tempo:

\begin{equation}
\begin{split}
\ket{\vec{r},t}_\mathscr{H} & =T^{\dag}(t,t_0)\ket{\vec{r}} \rightarrow 
\frac{d}{dt}\ket{\vec{r},t}_\mathscr{H} \\
& = \frac{d}{dt}T^{\dag}\ket{r} = \frac{i}{\hslash}H\ket{\vec{r},t}_\mathscr{H}
\end{split}
\end{equation}

Quindi i vettori $\ket{\vec{r} ,t}_\mathscr{H}$ variano nel tempo in modo 
esattamente opposto a come variano nel tempo i vettori nella rappresentazione 
di Schr\"{o}dinger.

\textit{In una qualsiasi rappresentazione vale sempre che} $\frac{\partial 
A}{\partial t}=0 \leftrightarrow \frac{d}{dt}A$. Nel caso della 
rappresentazione di Heisenberg si
 valuta $\frac{d}{dt}A_\mathscr{H}$ con l'ipotesi che sia $\frac{\partial 
A}{\partial t}=0 $ e non $\frac{\partial A_\mathscr{H}}{\partial t}=0 $.
 
\section{Rappresentazione di interazione}
Questa rappresentazione è molto usata nella fisica delle particelle elementari.
Consideriamo un generico sistema la cui Hamiltoniana viene posta nella forma:

\begin{equation}
H=H_{0}+H_\text{int}
\end{equation}
dove $H_0$ è il termine non perturbato e $H_\text{int}$ il termine di 
interazione. La
rappresentazione di interazione si ricava da quella di Schr\"{o}dinger
considerando la trasformazione unitaria:

\begin{empheq}[box=\fbox]{equation}
U = T_{0}^{\dag}(t,t_{0}) = e^{\frac{i}{\hslash}H_{0}(t-t_{0})}
\end{empheq}

Lo stato del sistema in questa rappresentazione diventa:

\begin{equation}
\ket{t}_I=T_{0}^{\dag}(t,t_{0})\ket{t}
\end{equation}
mentre un generico operatore si può scrivere come:

\begin{equation}
A_{I}=T_{0}^{\dag}(t,t_{0})AT_{0}(t,t_{0})
\end{equation}

Da questa definizione si nota subito che la rappresentazione di interazione è
una via di mezzo fra quella di Schr\"{o}dinger e quella di Heisemberg. In
particolare questa \textit{coincide con quella di Schr\"{o}dinger se $H_0=0$,
mentre coincide con quella di Heisenberg se $H_\text{int}=0$.}

Partendo dall'equazione di Schr\"{o}dinger e sostituendo $\ket{t}=T_0\ket{t}_I$
si ha:
\begin{equation}
\begin{split}
                      i\hslash \frac{\partial}{\partial t}\ket{t} &= H\ket{t} \\
               i\hslash \frac{\partial}{\partial t}(T_0\ket{t}_I) &= 
HT_0\ket{t}_I \\
HT_0\ket{t}_I + i\hslash T_0 \frac{\partial}{\partial t}\ket{t}_I &= 
(H_0+H_\text{int})T_0\ket{t}_I \\
                i\hslash T_0 \frac{\partial}{\partial t}\ket{t}_I &= 
H_\text{int}T_0\ket{t}_I \\
\end{split}
\end{equation}
e infine:

\begin{empheq}[box=\fbox]{equation}
i\hslash \frac{\partial}{\partial t}\ket{t}_I=(H_\text{int})_I\ket{t}_I
\end{empheq}

Questo significa che nella rappresentazione di interazione \textit{lo stato del
sistema varia nel tempo solo per effetto dell' Hamiltoniana di interazione.}

L'utilità di questa rappresentazione è evidente nello studio dei processi di
\textit{scattering}. Infatti con l'uso della rappresentazione di interazione è
possibile isolare la semplice evoluzione dinamica dallo stato iniziale allo
stato finale dopo lo scattering:

\begin{equation}
\ket{i}_I \rightarrow \ket{f}_I
\end{equation}

Questo discorso conserva la sua validità anche se il sistema è un sistema di
particelle.
Consideriamo ora un generico operatore $A$ che, nella rappresentazione di
Schr\"{o}dinger non dipende esplicitamente dal tempo, nella rappresentazione di
interazione invece varierà nel tempo solo per effetto di $H_0$:

\begin{equation}
\begin{split}
i\hslash \frac{\partial}{\partial t} A_I & = i\hslash \frac{\partial}{\partial
t}\left(e^{\frac{i}{\hslash}H_0(t-t_0)}Ae^{-\frac{i}{\hslash}H_0(t-t_0)}\right)=
\\
& =
i\hslash\left(\frac{i}{\hslash}H_0T_{0}^{\dag}AT_0-T_{0}^{\dag}A\frac{i}{\hslash
}H_0T_0
\right)= \\
& = -H_0A_I+A_IH_0= \\
& = [A_I,H_0]
\end{split}
\end{equation}

Se consideriamo il caso in cui $A\equiv H_0$ si vede dall'espressione di $T_0$
che $H_{0_I}\equiv H_0$. Quindi $H_0$ è la stessa per la rappresentazione di
Schr\"{o}dinger  e di interazione. Lo stesso non vale in generale per
$H_\text{int}$ in quanto:
\begin{equation}
\begin{split}
[H_\text{int},H_0] &\neq 0 \Rightarrow \\
\Rightarrow H_{\text{int}_{I}} &\neq H_\text{int} \\
\Rightarrow \frac{\partial}{\partial t}H_{\text{int}_{I}} &\neq 0
\end{split}
\end{equation}
Tutto questo è evidente nei processi di scattering. Nella rappresentazione di
interazione lo stato iniziale $\ket{i}_I$ e quello finale $\ket{f}_I$ sono
indipendenti dal tempo in quanto si possono considerare imperturbati, cioè sono
autostati di $H_0$. Lo stato del sistema risulterà dipendente dal tempo solo
nell' intervallo di tempo in cui avverrà la transizione da $\ket{i}_I$  a
$\ket{f}_I$, ma questa transizione avviene ad opera di $H_\text{int}$, dunque
soltanto nel breve periodo della transizione $H_{int_{I}}$ può risultare 
diversa
da zero.

\section{Momento angolare}
Il momento angolare in meccanica quantistica è un operatore vettoriale 
$\vec{J}=
(J_x, J_y, J_z)$ le cui componenti verificano le seguenti regole di
commutazione:

\begin{gather}
[J_x, J_y]= i\hslash J_z\\
[J_y, J_z]= i\hslash J_x\\
[J_z, J_x]= i\hslash J_y
\end{gather}

Da queste regole di commutazione segue immediatamente che non può esistere un
autostato comune a $J_x, J_y$ e $J_z$ con autovalori non nulli.
In particolare non può esistere un autostato comune a due sole componenti con
autovalori non nulli. Infatti se esistesse uno stato che è autostato di $J_x$ e
$J_y$ allora dalla
prima regola di commutazione si vede che questo deve essere autostato di $J_z$
con autovalore nullo, ma dalle altre due regole di commutazione si deduce che
anche gli autovalori di $J_x$ e $J_y$ devono essere nulli.\footnote{l' unico
caso in cui l'autostato è autostato di $J_x, J_y, J_z$ è quello in cui $J=0$}

Si definisce l'operatore:

\begin{equation}
\vec{J}^2=J_{x}^2+J_{y}^2+J_{z}^2
\end{equation}
che rappresenta il modulo quadro di $\vec{J}$. Si possono facilmente verificare
le seguenti regole di commutazione:

\begin{gather}
[\vec{J}^2,J_x]= 0\\
[\vec{J}^2,J_y]= 0\\
[\vec{J}^2,J_z]= 0\\
\end{gather}

Risulta quindi che $\vec{J}^2$ commuta con ciascuna componente.
L'insieme di tutte queste regole di commutazione implica che lo stato del
sistema può essere contemporaneamente autostato di $\vec{J}^2$ e di una sola
delle componenti di
$\vec{J}$. L'asse individuato da tale componente si dice asse di
quantizzazione.\footnote{Tipo asse z}

Si può dimostrare che gli autovalori dell'operatore $\vec{J}^2$ sono:

\begin{equation}
j(j+1)\hslash^2
\end{equation}
dove: $j=0,\frac{1}{2},1,\frac{3}{2},2,\frac{5}{2}\dots$ Fissato un sottospazio
$j$ di $\vec{J}^2$ si ha che in questo sottospazio gli autovalori di $J_z$ sono:

\begin{equation}
m_j\hslash
\end{equation}
con $m_j=-j,-j+1,\dots,j-1,j$. Quindi il valore massimo che può avere $J_z$ è
$j\hslash$ e si  vede che il modulo quadro di $\vec{J}$ è più grande del
quadrato di $J_Z=j^2\hslash^2<j(j+1)\hslash^2$. Questo è conseguenza del fatto 
che
non può esistere uno stato del sistema in cui una delle componenti assume 
valore
massimo mentre le altre due assumono valore nullo.

Quindi supponendo che lo stato sia un autostato di $J_z$ quello con valore
massimo $j\hslash$
\begin{gather}
\braket{J_z}=j\hslash\\
\braket{J}=\braket{J_x}\vec{i}+\braket{J_y}\vec{j}+\braket{J_z}\vec{k}
\end{gather}
deve corrispondere al vettore classico:
\begin{equation}
\braket{J}=(0,0,j\hslash)
\end{equation}

Il modulo del vettore $\vec{J}$ è quindi pari a $j^2\hslash^2$. Se fosse
$\braket{J}\cdot \braket{J}=\braket{j^2}$ avremmo:
\begin{equation}
\braket{J^2}=\braket{J_x^2+J_y^2+J_z^2}=\braket{J_z}=j^2\hslash
\end{equation}

Ma quindi $\braket{J_x^2}=\braket{J_y^2}=0$ e lo stato sarebbe anche autostato
di $J_x$ e di $J_y$ con autovalore nullo.
\breaknote
Sia\marginnote{17/12/1997} $\ket{m_j}$ il generico autostato di $J_z$ e $J^2$,
quindi deve essere che:
\begin{equation}
J_z\ket{m_j}=m_j\hslash \ket{m_j}
\end{equation}
con 
\begin{equation}
m_j=-j,-j+1,\dots,j
\end{equation}

Quindi fissato $j$ il numero di autovalori corrispondenti di $J_z$ è $2j+1$
(questo sia per $j$ intero che semi-intero), ovviamente $2j+1$ è anche il 
numero
degli autostati
$\ket{m_j}$. Questi costituiscono una base degli stati con momento angolare $J$
fissato.

Apartire da $J_x$ e $J_y$ si definiscono gli operatori:

\begin{gather}
J_+ =J_x+iJ_y\\
J_- =J_x-iJ_y
\end{gather}
$J_+$ e $J_-$ sono uno il complesso coniugato dell'altro. Si verifica facilmente
che:

\begin{gather}
[J_z,J_+]=\hslash J_+ \\
[J_z,J_-]=-\hslash J_-
\end{gather}

Da queste regole di commutazione si può risalire al significato fisico dei due
operatori $J_+$ e $J_-$. Infatti si ha che:

\begin{equation}
\begin{split}
J_z(J_+\ket{m_j}) &= (\hslash J_++J_+J_z)\ket{m_j}=\\
& = \hslash J_+\ket{m_j}+\hslash m_j J_+\ket{m_j}=\\
& = \hslash(1+m_j)(J_+\ket{m_j})
\end{split}
\end{equation}

Quindi lo stato $J_+\ket{m_j}$ è uno autostato di $J_z$ con autovalore $\hslash
(m_j+1)$. Analogamente si dimostra che lo stato $J_-\ket{m_j}$ è un autostato 
di
$J_z$ con autovalore $\hslash(m_j-1)$. Quindi si può scrivere che:

\begin{gather}
J_+\ket{m_j}= \alpha \ket{m_j+1} \\
J_-\ket{m_j}= \beta \ket{m_j-1}
\end{gather}
con $-j<m_j<j$

Se consideriamo il caso in cui il nostro sistema è costituito da una singola
particella a riposo allora il momento angolare $\vec{J}$ si riduce all'
operatore $\vec{S}$, il momento
angolare di spin\footnote{Momento angolare intrinseco. }. In questo caso si
pone:
\begin{gather}
j=s \\
m_j=m_s=s,s-1,\dots,-s
\end{gather}
dove $s$ è il numero quantico di spin. Se la particella non è a riposo allora
$\vec{J}$ è la somma vettoriale del momento angolare orbitale e del momento di
spin.
