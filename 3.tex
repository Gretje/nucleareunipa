%!TEX root = nucleare.tex
\chapter{Proprietà dei nuclei atomici}
\section{Distribuzione di carica del nucleo}
\subsection{Formula di Saxon}
Supponiamo \marginnote{19-01-1998} che la distribuzione $\rho{(\vec{r})}$ sia diversa da 0 solo nei punti in cui è verificata la relazione:
\begin{equation}
\vec{q} \cdot \vec{r} \ll 1
\end{equation}
in questo caso si può espandere la funzione $e^{i\vec{q} \cdot \vec{r}}$ in
serie di potenze. L'integrale, da $F{(\vec{q})}$, diventa quindi:
\begin{equation}
F{(\vec{q})} = \int \rho{(\vec{r})} \left[ 1 + i\vec{q} \cdot \vec{r} - \frac{1}{2!}(\vec{q} \cdot \vec{r})^2 + ... \right] d^3\vec{r}
\end{equation}
Se ci si ferma fino al termine del secondo ordine si ha:
\begin{equation}
F{(\vec{q})} = 1 + i\int \rho{(\vec{r})}(\vec{q} \cdot \vec{r}) d^3\vec{r} - \frac{1}{2} \int \rho{(\vec{r})}(\vec{q} \cdot \vec{r})^2 d^3\vec{r}
\end{equation}
e nell'ipotesi che la distribuzione $\rho$ abbia la proprietà:
\begin{equation}
\rho{(\vec{r})} = \rho{(-\vec{r})}
\end{equation}
che è verificata ad esempio se $\rho$ ha simmetria sferica, allora il primo
integrale si annulla. Indichiamo con $r_q$ la componente di $\vec{r}$ nella
direzione di $\vec{q}$, cioè $r_q q = \vec{r} \cdot \vec{q}$. Si ha:
\begin{equation}
\int \rho{(\vec{r})}(\vec{q} \cdot \vec{r})^2 d^3\vec{r} = q^2 \int \rho{(r)} r_q^2  d^3\vec{r} = \frac{1}{3} q^2 \int \rho{(r)} \vec{r}^2 d^3\vec{r}
\end{equation}
L'ultimo passaggio è giustificato dal fatto che, data la simmetria sferica, si
ha:
\begin{equation}
\int \rho{(r)} r_i^2 d^3\vec{r} = \int \rho{(r)} r_q^2 d^3\vec{r} \qquad [r_i (i = x,y,z)]
\end{equation}
\begin{equation}
\int \rho{(r)} \vec{r}^2 d^3\vec{r} = \sum_{i=1}^3 \int \rho{(r)} r_i^2 d^3\vec{r} = 3 \int \rho{(r)} r_q^2 d^3\vec{r}
\end{equation}
Quindi, sotto l'ipotesi di una $\rho{(\vec{r})}$ con simmetria sferica si ha:
\begin{equation}
F{(\vec{q})} = 1 - \frac{1}{6} q^2 \int \vec{r}^2 \rho{(r)}  d^3\vec{r} + ... = 1 - \frac{1}{6} q^2 \left \langle \vec{r}^2 \right \rangle + ...
\end{equation}
Da questa si vede che da $F{(q)}$ si può risalire ad una stima di $\left \langle
\vec{r}^2 \right \rangle$. In realtà non è così semplice determinare la forma
della funzione $F{(q)}$. Per determinare poi $\rho$ si procede per tentativi,
cioè si ipotizza una particolare $\rho{(r)}$ e si confrontano poi i fattori di
forma teorico e sperimentale. Con questo metodo si è trovata una distribuzione
che risulta in accordo con i dati sperimentali. Questa è chiamata
\textit{Formula di Saxon}:
\begin{empheq}[box=\fbox]{equation}
\rho{(r)} = \frac{\rho_1}{e^{(r-c)/Z_1} + 1}
\end{empheq}

dove si ha:
\begin{itemize}
\item[$c$]$= 1.07 \times A^{\frac{1}{3}} \times 10^{-13}$ cm;
\item[$Z_1$]$= 0.545 \times 10^{-13}$ cm;
\item[$\rho_1$]= costante di normalizzazione;
\end{itemize}

Questa $\rho{(r)}$ risulta valida per tutti i nuclei con $A > 30$ con piccole
variazioni del parametro c. La distribuzione di carica ha quindi un andamento
del tipo (\autoref{fig:saxon}):
\begin{figure}[hbtp]
\centering
\caption{Formula di Saxon.}
\label{fig:saxon}
\begin{tikzpicture}[line cap=round,line join=round,>=stealth
  ,x=1.0cm,y=1.0cm]
  \draw[->,color=black] (-0.72,0) -- (8.62,0);
  \foreach \x in {,1,2,3,4,5,6,7,8}
  \draw[shift={(\x,0)},color=black] (0pt,2pt) -- (0pt,-2pt);
  \draw[->,color=black] (0,-0.54) -- (0,6.14);
  \foreach \y in {,1,2,3,4,5,6}
  \draw[shift={(0,\y)},color=black] (2pt,0pt) -- (-2pt,0pt);
  \clip(-0.72,-0.54) rectangle (8.62,6.14);
  \draw (-0.24,6.28) node[anchor=north] {$\rho$};
  \draw (6.89,-0.02) node[anchor=north west] {$r$};
  \draw[smooth,samples=100,domain=0.0:8.623178016726417]
  plot(\x,{5/(2.718281828^(((\x)-3.32)/0.55)+1)});
\end{tikzpicture}

\end{figure}
Il massimo di $\rho$ si ha per $r = 0$, cioè al centro del nucleo.

\section{Momenti magnetici dell'atomo e struttura iperfine}
\subsection{Hamiltoniana dell'elettrone}
Un gran numero di esperimenti hanno messo in evidenza che i nuclei atomici hanno
un momento angolare intrinseco al quale è associato un momento magnetico. Dal
momento che i nuclei sono costituiti da particelle (protoni e neutroni) di spin
$\frac{1}{2}$ si hanno:
\begin{equation} \begin{split}
\text{Bosoni}& \Rightarrow \text{Numero di massa pari \qquad(Spin} = n\hbar);\\
\text{Fermioni}& \Rightarrow \text{Numero di massa dispari \qquad(Spin} = \frac{2n+1}{2}\hbar);
\end{split}\end{equation}
L'interazione elettromagnetica dovuta allo spin del nucleo genera nell'atomo la struttura iperfine. Si ha che:
\begin{equation}
\vec{\mu}_I = \frac{g_I \mu_N \vec{I}} {\hbar}
\end{equation}
dove:
\begin{itemize}
\item[$g_I$]= Fattore $g$ nucleare;
\item[$\mu _N$]= Magnetone $(?)$ nucleare = $\frac{e\hbar}{2m_pc} \simeq 0.5050823 \times 10^{-23}$ erg/gauss;
\item[$\vec{I}$]= Spin del nucleo;
\end{itemize}
Si usa anche la scrittura:
\begin{equation}
\vec{\mu}_I =  \gamma_I \vec{I} \qquad \text{con}\ \gamma_I = \frac{g_I \mu_N}{\hbar}\ \text{(Rapporto giromagnetico nucleare)}
\end{equation}
ovviamente $g_I$ e $\gamma_I$ sono grandezze equivalenti.

Analizziamo ora il meccanismo dinamico che genera la struttura iperfine
dell'atomo. La presenza di $\mu_I$ implica che se c'è un campo magnetico esterno
$H$, si ha un'energia potenziale:
\begin{equation}
W_m = -\vec{\mu}_I \cdot \vec{H} \qquad \Rightarrow \qquad \left \langle W_m \right \rangle = -\left \langle \vec{\mu}_I \cdot \vec{H} \right \rangle
\end{equation}
Supponiamo che si abbia un nucleo in un atomo con un singolo elettrone di
valenza. Quindi $H$ è solo il campo generato da questo elettrone di valenza in
quanto gli altri elettroni per il principio di Pauli hanno un momento magnetico
totale nullo. Sia $\vec{J}$ il momento angolare totale dell'elettrone di
valenza, cioè:
\begin{equation}
\vec{J} = \vec{L} + \vec{S}
\end{equation}
 $\vec{H}{(0)}$ (il campo generato dall'elettrone nel punto dove si trova il
 nucleo) sarà ovviamente nella stessa direzione di $\vec{J}$, ma verso opposto
 (in quanto l'elettrone ha carica negativa). Possiamo scrivere:
\begin{equation}
\vec{H}{(0)} = H{(0)}\frac{\vec{J}}{j\hbar} \qquad \text{con}\ H{(0)}<0
\end{equation}
Si avrà quindi uno splitting di energia pari a:
\begin{equation}
\Delta W = \left \langle W_m \right \rangle = -\left \langle \vec{\mu}_I \vec{H}{(0)} \right \rangle = -\gamma_I \frac{H{(0)}}{j \hbar} \left \langle \vec{I} \vec{J} \right \rangle
\end{equation}
L'Hamiltoniana totale $\mathcal{H}$ commuta con $W_m$, ed anche $I^2$ e $J^2$
commutano con $W_m$, cioè:
\begin{equation}\begin{split}
\left [ \mathcal{H}, W_m \right ] = 0\\
\left [ I^2, W_m \right ] = 0\\
\left [ J^2, W_m \right ] = 0
\end{split}\end{equation}
(Se consideriamo un atomo idrogenoide nell'Hamiltoniana totale ci sono i termini
$P^{2}/2m$\footnote{È indecifrabile questa parte}, $L^2$, $\vec{L} \cdot \vec{S}$
dove $L$ e $S$ sono relativi all'elettrone. Si può verificare che ciascuno di
questi termini commuta con $\vec{I} \cdot \vec{J}$, $I^2$ e $J^2$).

Mentre invece:
\begin{equation}\begin{split}
\left [ I_Z, W_m \right ] \ne 0\\
\left [ J_Z, W_m \right ] \ne 0
\end{split}\end{equation}
Quindi invece lo spin del nucleo e il momento angolare totale dell'elettrone si conserveranno in modulo ma non in direzione. Invece si conserverà il momento angolare totale dell'atomo:
\begin{equation}
\vec{F} = \vec{I} + \vec{J}
\end{equation}
Si possono quindi immaginare questi tre vettori come in
\autoref{fig:precessione}:
\begin{figure}[hbtp]
\centering
\caption{Precessione dei momenti.}
\label{fig:precessione}
\begin{tikzpicture}[line cap=round,line join=round,>=triangle
  45,x=1.0cm,y=1.0cm]
  \clip(-1.14,-0.7) rectangle (2.54,4.46);
  \draw [->,color=\MainColor] (0,0) -- (0,4);
  \draw [-> ] (0,0) -- (2.02,1.16);
  \draw [-> ] (2.02,1.16) -- (0,4);
  \draw (-0.62,2.5) node {$ \langle F \rangle $};
  \draw (1.32,3.12) node[anchor=north west] {$ \langle J\rangle $};
  \draw (1.18,0.74) node[anchor=north west] {$ \langle I\rangle $};
\end{tikzpicture}

\end{figure}
dove $\left \langle F \right \rangle$ rimane fisso mentre $\left \langle J
\right \rangle$ e $\left \langle I \right \rangle$ hanno lo stesso moto di
precessione attorno alla direzione di $\left \langle F \right \rangle$

\subsection{Livelli di struttura iperfine e regola degli intervalli}
Abbiamo \marginnote{21-01-1998} visto che lo spostamento di energia in presenza
di un campo magnetico è:
\begin{equation}
\Delta W = \left \langle W_m \right \rangle = -\left \langle \vec{\mu}_I \cdot \vec{H}{(0)} \right \rangle = -\gamma_I \frac{H{(0)}}{j \hbar} \left \langle \vec{I} \cdot \vec{J} \right \rangle
\end{equation}
Calcoliamo quindi questo valore medio:
\begin{equation}
W_m =  -\gamma_I \frac{H{(0)}}{j \hbar} \vec{I} \cdot \vec{J}
\end{equation}
Viste le regole di commutazione di $W_m$ gli stati stazionari del sistema
saranno autostati anche di $\vec{I}\cdot\vec{J}$ e quindi di $F^2$.
Si può dunque scrivere:
\begin{equation}\begin{split}
\left \langle F^2 \right \rangle &= f (f+1) \hbar\\
\vec{I}\vec{J} &= \frac{1}{2}\left [ F^2 - I^2 - J^2 \right ]\\
\left \langle \vec{I} \vec{J} \right \rangle &= \frac{1}{2} \left [ \left
\langle F^2 \right \rangle - \left \langle I^2 \right \rangle - \left \langle
J^2 \right \rangle \right ] = \frac{\hbar^2}{2} \left [ f(f+1) - i(i+1) - j(j+1)
\right ]\\
\Delta W &= -\frac{1}{2} \gamma_I \hbar \frac{H{(0)}}{j} \left [ f(f+1) - i(i+1)
- j(j+1) \right ]
\end{split}\end{equation}
e per ogni valore di $f$ si avrà un diverso valore di $\Delta W$. Quindi se
all'interazione elettrostatica si aggiunge l'interazione magnetica si avrà un
numero di sottolivelli determinati dal valore di $f$. Il numero di questi
sottolivelli è:
\begin{equation}\begin{split}
2i + 1 \qquad \text{se}\ i<j\\
2j + 1 \qquad \text{se}\ i>j
\end{split}\end{equation}
Questi livelli sono quelli che generano la \textit{struttura iperfine} che si
osserva sperimentalmente. Se si conoscono $j$ e $f$ si può risalire al valore
del numero quantico di spin nucleare $i$. Per esempio, se si osservano un numero
di righe spettrali $N < 2j + 1$
allora deve essere $i < j$ e $N = 2i + 1$. Quindi dalla semplice conoscenza di
$N$ si ricava $i = \frac{1}{2} (N-1)$. Quando invece risulta $N = 2j + 1$ le
cose risultano un pò più complicate. In questo caso si può avere per $i:\ i \ge
j$. Il valore di $i$ si può ricavare utilizzando la regola degli intervalli:
\begin{equation}
f: \begin{cases}
f_1 = i + j &\rightarrow \qquad W_1 = W + \Delta W_1\\
f_2 = i + j-1 &\rightarrow \qquad W_2 = W + \Delta W_2\\
f_3 = i + j-2\qquad & \rightarrow \qquad W_3 = W + \Delta W_3\\
...
\end{cases}
\end{equation}
Si ha che:
\begin{equation}\begin{split}
\frac{W_1-W_2}{W_2-W_3} = &\frac{\Delta W_1 - \Delta W_2}{\Delta W_2 - \Delta W_3} =\frac{f_1(f_1+1) - f_2(f_2+1)}{f_2(f_2+1) - f_3(f_3+1)} =\\\\
= &\frac{f_1(f_1+1) - f_2f_1}{f_2(f_2+1)-f_3f_2} = \frac{f_1 \left [f_1 + 1 - f_2 \right ]}{f_2 \left [f_2 + 1 - f_3 \right ]} = \frac{f_1}{f_2}
\end{split}\end{equation}
Questo risultato può essere generalizzato scrivendo per esteso la \textit{Regola degli intervalli}:
\begin{empheq}[box=\fbox]{equation}
\frac{W_{\alpha} - W_{\alpha + 1}}{W_{\alpha + 1} - W_{\alpha + 2}} = \frac{f_{\alpha}}{f_{\alpha + 1}}
\end{empheq}
Questo rapporto può essere valutato sperimentalmente, dalla conoscenza di $\frac{f_{\alpha}}{f_{\alpha + 1}}$ e $j$ si può arrivare alla determinazione di $i$. La regola degli intervalli è utile anche per determinare l'aderenza del modello teorico con la realtà. La deviazione di piccoli scarti dalla regola degli intervalli portò alla scoperta del momento di quadrupolo del nucleo.

Quanto detto fin'ora può essere generalizzato ad atomi con più elettroni di valenza, basta porre $\vec{J}$ uguale al momento totale degli elettroni e considerare in $\vec{H}$ anche il contributo degli altri elettroni.
Le misure di $i$ non portano ad una determinazione univoca di $\vec{\mu}_I$ in quanto resta da misurare il rapporto giromagnetico nucleare $\gamma_I = g_I \frac{\mu_N}{\hbar}$. 

\subsection{Rapporto giromagnetico nucleare}
Il metodo più semplice per determinare $g_I$ è quello di considerare un campo magnetico esterno uniforme che sia tanto grande da poter trascurare l'interazione con gli elettroni.
Se $\vec{H}$ è diretto lungo l'asse $\hat{z}$, l'Hamiltoniana di interazione è:
\begin{equation}
\mathcal{H}_{int} = - \vec{\mu}_I \vec{H}_\text{est} = - \gamma_I I_z H_z
\end{equation}
e l'Hamiltoniana totale del nucleo, si può scrivere dunque come:
\begin{equation}
\mathcal{H} = \mathcal{H}_0 + \mathcal{H}_{int}
\end{equation}
Ovviamente si avrà che:
\begin{equation}
\left [ \mathcal{H}_0, \vec{I} \right ] = 0; \qquad \left [ \mathcal{H}, I_z \right ] = 0
\end{equation}
Il nucleo, in questa situazione subirà l'\textit{effetto Zeeman}, cioè:
\begin{equation}
\Delta W_\text{nucl} = - \gamma_I m_I H_\text{est} \hbar \qquad \text{con}\ m_I = \text{$i$, $i-1$, ..., $-i$}
\end{equation}

Si formeranno $2i+1$ sottolivelli di energia del nucleo e l'energia di ciascun
sottolivello sarà caratterizzata solo da $m_I$.

Se si fa passare il nucleo da un sottolivello ad un altro adiacente si può
misurare $\gamma_I$. Infatti nella transazione il nucleo emetterà (o assorbirà)
un quanto di energia pari a:
\begin{equation}\begin{split}
h \nu &= \left| - \gamma_I(m_I + 1) H_\text{est} \hbar + \gamma_I m_I H_\text{est} \hbar \right| = \gamma_I H_\text{est} \hbar \\
\nu &= \frac{1}{2\pi} \left| \gamma_I \right| H_\text{est}
\end{split}\end{equation}
Dalla frequenza $\nu$ della radiazione emessa (o assorbita) e da $H_\text{est}$ si può quindi ricavare $\left| \gamma_I \right|$. La frequenza $\nu$ coincide con la \textit{frequenza di Larmor} con cui precessa $\left\langle \vec{I} \right\rangle$ attorno alla direzione di $\vec{H}_\text{est}$. Per ricavare questo risultato basta usare un modello classico.

Supponiamo di avere un campo magnetico $H_\text{est}$ in cui è presente un giroscopio caratterizzato da:
\begin{equation}
\vec{I}^{ce} = \left\langle \vec{I} \right\rangle; \qquad \vec{\mu}_I^{cl} = \left\langle \vec{\mu}_I \right\rangle = \gamma_I \vec{I}^{cl}
\end{equation}
Questo giroscopio sarà soggetto ad un momento meccanico:
\begin{equation}
\vec{\tau} = \frac{d}{dt}\vec{I}^{cl} = \vec{\mu}_I^{cl} \wedge \vec{H}_\text{est} = \gamma_I \vec{I}^{cl} \wedge \vec{H}_\text{est}
\end{equation}
si avrà quindi:
\begin{equation}\begin{split}
\frac{d}{dt} (\vec{I}^{cl})^{2} = 2 \vec{I}^{cl} \cdot \frac{d}{dt} \vec{I}^{cl} = 0 \qquad &\Rightarrow \qquad (\vec{I}^{cl})^{2} = \text{costante}\\
\frac{d}{dt} (\vec{H}_\text{est} \cdot \vec{I}^{cl}) = \vec{H}_\text{est} \cdot \frac{d}{dt} \vec{I}^{cl} = 0 \qquad \Rightarrow \qquad \vec{I}^{cl} \cdot \vec{H}_\text{est} = \text{costante} \qquad &\Rightarrow \qquad I_z^{cl} = \text{costante}
\end{split}\end{equation}
Quindi il vettore $\vec{I}^{cl}$ sarà soggetto ad un moto di precessione attorno all'asse $\hat{z}$. Se $\gamma_I$ è positivo la rotazione sarà in senso orario. Quindi:
\begin{equation}
\begin{split}
\left| \frac{d}{dt} \vec{I}^{cl} \right| &= - \gamma_I I^{cl}H_\text{est} \sin \theta\\
\left| d \vec{I}^{cl} \right| &= I^{cl} \sin \theta d\varphi
\end{split}
\end{equation}
Da queste si può dedurre la velocità angolare di precessione:
\begin{equation}
\begin{split}
\omega_L = \frac{d \varphi}{dt} = - \gamma_I H_\text{est} \qquad \Rightarrow \qquad \nu = \frac{1}{2 \pi} \left|\omega_L \right| = \frac{1}{2 \pi} \left| \gamma_I \right| H_\text{est}
\end{split}
\end{equation}
A \marginnote{23-01-1998} questo punto si può applicare il principio di
corrispondenza e si può concludere che $\left\langle \vec{I} \right\rangle$ sarà
soggetto a questo stesso moto di precessione con la stessa frequenza. Il moto di
$\left\langle \vec{I} \right\rangle$ si può equivalentemente dedurre dal fatto
che:
\begin{equation}
\left[ I^2, \mathcal{H} \right] = \left[ I_z, \mathcal{H} \right] = 0 \qquad \text{e}\ \qquad \left[ I_x, \mathcal{H} \right] \ne 0;\ \left[ I_y, \mathcal{H} \right] \ne 0
\end{equation}
Vediamo come si può sfruttare questa proprietà per far passare il nucleo da uno
stato a quello adiacente in modo da misurare $\gamma_I$. A due livelli di
energia adiacenti saranno associati due diversi valori di $\theta$. Per
stimolare la transizione si può usare un campo magnetico $H'$ ortogonale ad
$H_\text{est}$ e molto più piccolo in modulo. Questo campo $H'$ deve essere
fatto ruotare attorno all'asse $\hat{z}$ con una velocità angolare risonante
$\omega_r = \omega_L$. Se ci si pone nel sistema di riferimento solidale a
$\left\langle \vec{I} \right\rangle$ il campo $H'$ sarà visto come un campo
costante, il suo effetto sarà quindi quello di far variare l'angolo $\theta$. Il
fatto che $H' \ll H_\text{est}$ garantisce che i livelli energetici non saranno
modificati. La transizione voluta si avrà solo quando la frequenza di $H'$ sarà
uguale a $\omega_r$.

Il rapporto giromagnetico nucleare  si può quindi determinare dal valore
sperimentale di $\omega_r$:
\begin{equation}
\gamma_I = \frac{\omega_r}{H_\text{est}} \qquad \Leftrightarrow \qquad g_I = - \frac{\hbar}{\mu_N} \frac{\omega_r}{H_\text{est}}
\end{equation}
Nella pratica invece di usare un campo magnetico $H'$ ruotante si usa un campo
magnetico alternato, ad esempio, nella direzione $\hat{y}$. Questo campo può
essere pensato come la sovrapposizione di due campi uguali e ruotanti con
frequenze opposte, cioè:
\begin{equation}
\vec{H}' = H' \cos (\omega t) \hat{y} = \frac{1}{2} H' (\cos (\omega t) \hat{y} + \sin (\omega t) \hat{x} ) + \frac{1}{2} H' (\cos (\omega t) \hat{y} - \sin (\omega t) \hat{x} )
\end{equation}
di questi due campi sarà efficace solo quello che ruota nel verso giusto. In
realtà ciò che si misura è la potenza assorbita dai nuclei in funzione di
$\omega$. Questa potenza avrà un massimo quando $\omega = \omega_r$. Per molti
nuclei il valore di $g_I$ è attorno all'unità e può essere sia positivo che
negativo. Il fatto che possa anche essere negativo si spiega con il fatto che i
neutroni anche se hanno carica nulla hanno un momento magnetico diverso da zero
con un fattore $g$ negativo. Per $H_\text{est}$ si usano valori dell'ordine di
$1000$ gauss. In questo caso, per la frequenza di risonanza si ha $\omega_r
\approx 10^6$ cicli/sec.

\section{Momenti elettrici dell'atomo e tensore $\mathcal{Q}$}
\subsection{Sviluppo in serie del potenziale del nucleo}
In un atomo il campo generato dagli elettroni si può considerare simmetrico
attorno alla direzione di $\left\langle \vec{J} \right\rangle$, questa
assunzione è buona viste le velocità con cui ruotano gli elettroni. Analogamente
la carica nucleare può essere assunta simmetrica attorno alla direzione di
$\left\langle \vec{I} \right\rangle$.

Introduciamo due sistemi di riferimento con la stessa origine posta nel centro
del nucleo. Il sistema di riferimento atomico sia preso con l'asse $\hat{z}$
parallelo a $\left\langle \vec{J} \right\rangle$. Il sistema di riferimento
nucleare con l'asse $\hat{z}'$ parallelo a $\left\langle \vec{I} \right\rangle$.
L'energia elettrostatica del nucleo dovuta alla carica degli elettroni sarà data
in termini classici da:
\begin{equation}
W_E = \int \rho^N_c{(\vec{r})} \varphi{(\vec{r})} dv
\end{equation}
dove $\varphi{(\vec{r})}$ è il potenziale dovuto agli elettroni. Si ha che:
\begin{equation}
\int \rho^{N}_c{(\vec{r})} dv = Ze \qquad \qquad Z = \text{numero atomico}
\end{equation}
Dal momento che le dimensioni del nucleo sono molto piccole rispetto a quelle
atomiche di può sviluppare $\varphi$ in serie di Taylor. Lavoriamo nel sistema
atomico ponendo:
\begin{equation}
\vec{r} = (x, y,z) = (x_1, x_2, x_3)
\end{equation}
Quindi si ha che:
\begin{equation}
\begin{split}
&\varphi{(\vec{r})} = \varphi{(0)} + \vec{\nabla} \varphi \vec{r} + \frac{1}{2} \sum_{\alpha, \beta = 1}^3 \frac{\partial^2 \varphi}{\partial x_{\alpha} \partial x_{\beta}} x_{\alpha} x_{\beta} + ...\\
W_E &= \int \rho^{N}_{c} (\vec{r}) \left[ \varphi{(0)} + \vec{\nabla} \varphi \vec{r} + \frac{1}{2} \sum_{\alpha, \beta = 1}^3 \frac{\partial^2 \varphi}{\partial x_{\alpha} \partial x_{\beta}} x_{\alpha} x_{\beta} + ... \right] dv
\end{split}
\end{equation}
dove le derivate sono calcolate nell'origine. Se trascuriamo tutti i termini
della serie tranne il primo si ottiene in questa approsimazione:
\begin{equation}
W_E = \int \rho^{N}_c{(\vec{r})} \varphi{(0)} dv = \varphi{(0)} z e
\end{equation}
e questo equivale ad assumere un nucleo puntiforme, caso in cui il nucleo tende
ovviamente a comportarsi come un semplice monopolo elettrico. Si avrà dunque
sempre la stessa energia indipendentemente dall'orientazione relativa di
$\left\langle \vec{I} \right\rangle$ e $\left\langle \vec{J} \right\rangle$.

Se consideriamo l'approssimazione successiva si ottiene il termine aggiuntivo:
\begin{equation}
\int \rho^{N}_c{(\vec{r})} \vec{\nabla} \varphi \vec{r} dv = \vec{\nabla} \varphi \int \rho^{N}_c{(\vec{r})} \vec{r} dv
\end{equation}
questo termine è non nullo solo se il nucleo ha un momento di dipolo nucleare definito come:
\begin{equation}
\vec{d}_N = \int \rho^{N}_c{(\vec{r})} \vec{r} dv
\end{equation}
per simmetria $\vec{d}_N$ deve essere orientato come $\left\langle \vec{I} \right\rangle$.

La presenza di questo momento implicherebbe che l'energia sia funzione
dell'orientazione tra $\left\langle \vec{I} \right\rangle$ e $\left\langle
\vec{J} \right\rangle$. Tuttavia non si è mai osservata alcuna dipendenza
dell'energia da tali parametri, quindi è lecito porre:
\begin{equation}
\vec{d}_N = 0
\end{equation}
questo risultato può essere anche giustificato teoricamente mediante l'uso della
meccanica quantistica.
\\
\\
Consideriamo l'espressione quantistica della densità di carica nucleare:
\begin{equation}
\rho_c{(x,y,z)}^N = \sum_{i=1}^A e_i P_i{(x,y,z)}
\end{equation}
dove $e_i$ è la carica del nucleone i-esimo e $P_i$ è la densità di probabilità
di trovare il nucleone nel punto $\vec{r} = (x,y,z)$.
Dimostreremo in seguito che l'interazione che tiene uniti i nucleoni è
invariante rispetto all'inversione degli assi, questo implica che:
\begin{equation}
P_i{(-x,-y,-z)} = P_i{(x,y,z)} \qquad \Rightarrow \qquad \rho_c{(-x,-y,-z)}^N = \rho_c{(x,y,z)}^N
\end{equation}
Questa proprietà spiega la mancanza di un momento di dipolo elettrico. Infatti
ogni contributo al momento di dipolo $\vec{r} \rho_c^N dv$ viene annullato dal
corrispondente contributo speculare $- \vec{r} \rho_c^N dv$.

Consideriamo ora il contributo del secondo ordine nello sviluppo di $\varphi$
(potenziale elettrico generato dagli elettroni). I termini associati alle
derivate seconde $\frac{\partial^2 \varphi}{\partial x_{\alpha} \partial
x_{\beta}}$ con $\alpha \ne \beta$ non danno alcun contributo all'energia in
quanto risultano nulli per la simmetria assiale di $\varphi$ (le derivate sono
calcolate nell'origine).

Dimostriamo questa affermazione considerando prima le derivate rispetto a $x$ e
$y$ (l'asse di simmetria in questo caso è l'asse $\hat{z}$)
\begin{equation}
\frac{\partial}{\partial x} \left( \frac{\partial \varphi}{\partial y} \right) = \frac{\partial}{\partial y} \left( \frac{\partial \varphi}{\partial x} \right) \qquad \text{e} \qquad \frac{\partial \varphi}{\partial x} = \frac{\partial \varphi}{\partial y} = \frac{\partial \varphi}{\partial r} \qquad \qquad r = \sqrt{x^2 + y^2}
\end{equation}
la prima uguaglianza vale sempre mentre la seconda soltanto nell'origine. Quindi sempre nell'origine si ha:
\begin{equation}
\frac{\partial}{\partial x} \left( \frac{\partial \varphi}{\partial y} \right) = \frac{\partial}{\partial y} \left( \frac{\partial \varphi}{\partial x} \right) = \frac{\partial}{\partial r_{\perp}} \left( \frac{\partial \varphi}{\partial r} \right) = 0
\end{equation}
dove $\frac{\partial}{\partial r_{\perp}}$ è l'operatore derivata nella
direzione ortogonale a quella di $r$. Consideriamo ora le derivate miste che
comprendono $z$. Per dimostrare che anche queste sono nulle basta considerare il
fatto che $\varphi$ è simmetrico rispetto all'inversione dell'asse $\hat{z}$ e
quindi nell'origine si ha $\frac{\partial \varphi}{\partial z} = 0$. Anche le
funzioni $\frac{\partial \varphi}{\partial x}$ e $\frac{\partial
\varphi}{\partial y}$ sono simmetriche per inversione dell'asse $\hat{z}$ e
quindi è dimostrato che:
\begin{equation}
\frac{\partial^2 \varphi}{\partial z \partial x} = \frac{\partial^2 \varphi}{\partial z \partial y} = 0
\end{equation}
Possiamo quindi concludere che l'energia del nucleo nel campo elettrostatico
elettronico (cioè, dell'elettrone) si può scrivere come:
\begin{equation}
\begin{split}
W_E &= Z e \varphi{(0)} + \Delta W_{\mathcal{Q}}\\
\Delta W_{\mathcal{Q}} &= \frac{1}{2} \int \rho_c^N \sum_{\alpha = 1}^3 \left| \frac{\partial^2 \varphi}{\partial x_{\alpha}^2} \right|{(x=y=z=0)}  x_{\alpha}^2 dv \qquad = \qquad \frac{1}{2} \sum_{\alpha=1}^3 \left| \frac{\partial^2 \varphi}{\partial x_{\alpha}^2} \right|{(0)} \int \rho_c^N  x_{\alpha}^2 dv
\end{split}
\end{equation}
Possiamo dunque a questo punto sfruttare il fatto che nell'origine $\varphi$
deve soddisfare l'equazione di Laplace:
\begin{equation}
\frac{\partial^2 \varphi}{\partial x^2} + \frac{\partial^2 \varphi}{\partial y^2} + \frac{\partial^2 \varphi}{\partial z^2} = 0 \qquad \Rightarrow \qquad \frac{\partial^2 \varphi}{\partial x^2} + \frac{\partial^2 \varphi}{\partial y^2} = -\frac{1}{2} \frac{\partial^2 \varphi}{\partial z^2}
\end{equation}
dove l'ultima implicazione sfrutta le proprietà di simmetria di $\varphi$.

Quindi per $\Delta W_{\mathcal{Q}}$ si può scrivere:
\begin{equation}
\begin{split}
\Delta W_{\mathcal{Q}} &= \frac{1}{2} \frac{\partial^2 \varphi}{\partial z^2} \int \rho_c^N \left[z^2 - \frac{1}{2}\left( x^2 + y^2 \right) \right] dv\\
&= \frac{1}{4} \frac{\partial^2 \varphi}{\partial z^2} \int \rho_c^N \left[2z^2 - \left( x^2 + y^2 \right) \right] dv = \frac{1}{4} \frac{\partial^2 \varphi}{\partial z^2} \int \rho_c^N \left( 3z^2 - r^2 \right) dv
\end{split}
\end{equation}
dove si è posto $r^2 = x^2 + y^2 + z^2$. Tutti questi conti sono stati svolti
\footnotetext{Il carattere in questione è una Q calligrafica [NdT].}
nel sistema di riferimento atomico, cioè sul sistema con l'asse $\hat{z}$
parallelo a $\left\langle \vec{J} \right\rangle$. Per evidenziare il significato
fisico del termine $\Delta W_{\mathcal{Q}}$ conviene ora passare al sistema di
riferimento nucleare, in cui l'asse $\hat{z}$ è parallelo a $\left\langle
\vec{I} \right\rangle$. Poniamo per semplicità:
\begin{equation}
x'_1 = x' \qquad x'_2 = y' \qquad x'_3 = z'
\end{equation}
e introduciamo il tensore\footnote{$\mathcal{Q}$ è una Q calligrafica. [NdT]}:
\begin{empheq}[box=\fbox]{equation}
\mathcal{Q}_{\alpha \beta} = \int \rho_c^N \left( 3 x'_{\alpha} x'_{\beta} -
\delta_{\alpha \beta} r^2 \right) dv \qquad \qquad \alpha, \beta = \left\lbrace
1, 2, 3 \right\rbrace
\end{empheq}
dove $\delta_{\alpha \beta}$ indica la delta di Kronecker. Questo tensore
definisce il momento di quadrupolo elettrico nucleare, dal momento che l'asse
$\hat{z}'$ coincide con l'asse di rotazione medio nucleare possiamo assumere che
questo sia l'asse di simmetria per la distribuzione di carica $\rho_c^N$. Quindi
per $\rho_c^N$ vale che:
\begin{equation}
\rho_c{(x',y',z')}^N = \rho_c{(-x',y',z')}^N = \rho_c{(x',-y',z')}^N
\end{equation}
da questa proprietà segue che:
\begin{equation}
\text{per}\ \alpha \ne \beta \qquad \mathcal{Q}_{\alpha \beta} = \int \rho_c^N 3 x'_{\alpha} x'_{\beta} dv = 0
\end{equation}
Un'altra proprietà del tensore $\mathcal{Q}_{\alpha \beta}$ è quella di avere
traccia nulla, infatti:
\begin{equation}
\mathcal{Q}_{x' x'} = \mathcal{Q}_{y' y'} = -\frac{1}{2} \mathcal{Q}_{z' z'}
\end{equation}
(dalla definizione di $\mathcal{Q}_{\alpha \beta}$ si ricava che
$\mathcal{Q}_{x' x'} = \mathcal{Q}_{y' y'}$ e $\mathcal{Q}_{x' x'} +
\mathcal{Q}_{y' y'} + \mathcal{Q}_{z' z'} = 0$. Da queste due si ricava
l'equazione sopra).

Possiamo dunque concludere che il tensore $\mathcal{Q}_{\alpha \beta}$ è un
tensore diagonale con una sola componente indipendente, che possiamo
identificare con $\mathcal{Q}_{z' z'}$. Poniamo quindi:
\begin{empheq}[box=\fbox]{equation}
\mathcal{Q} = \mathcal{Q}_{z' z'} = \int \rho_c^N \left( 3 z^2 - r^2 \right) dv
\end{empheq}

Abitualmente, è proprio questa quantità che viene intesa come \textit{momento di
quadrupolo elettrico del nucleo}.

\subsection{Momento di quadrupolo elettrico del nucleo}
Vediamo ora di esprimere $\Delta W_{\mathcal{Q}}$ in funzione di $\mathcal{Q}$.
Vista la simmetria assiale della distribuzione di carica nucleare gli assi
$\hat{x}'$ e $\hat{y}'$ possono essere scelti arbitrariamente, possiamo quindi
sceglierli in maniera tale che l'asse $\hat{z}$ cada sul piano $\hat{x}'
\hat{z}'$. Si può dunque scrivere:
\begin{figure}[hbtp]
\centering
\caption{Sistema di riferimento.}
\label{fig:sisrif}
\begin{tikzpicture}[line cap=round,line join=round,>=stealth
  ,x=1.0cm,y=1.0cm]
  \clip(-0.4,-0.4) rectangle (8.2,4.14);
  \draw [shift={(0,0)},color=\MinorColor,fill=\MinorColor,fill opacity=0.1] (0,0) --
  (50.28:0.6) arc (50.28:90:0.6) -- cycle;
  \draw [->] (0,0) -- (0,4);
  \draw [->] (0,0) -- (3.14,3.78);
  \draw [->] (0,0) -- (8,0);
  \draw (0.26,1.5) node[anchor=north west] {$ \theta $};
  \draw (-0.28,3.68) node[anchor=north ] {$ z' $};
  \draw (2.1,3.4) node[anchor=north west] {$z$};
  \draw (6.7,0.72) node[anchor=north west] {$ x' $};
\end{tikzpicture}

\end{figure}
\begin{equation}
\begin{split}
z = &z' \cos \theta + x' \sin \theta
\end{split}
\end{equation}
dove $\theta$ è l'angolo formato dagli assi $\hat{z}$ e $\hat{z}'$, quindi
coincide con l'angolo formato dai vettori $\left\langle \vec{J} \right\rangle $
e $\left\langle \vec{I} \right\rangle $. In base a questa scelta si ha che:
\begin{equation}
z^2 = z'^2 \cos^2 \theta + x'^2 \sin^2 \theta + 2 x' z' \sin \theta \cos \theta
\end{equation}
Sostituendo queste espressioni in $\Delta W_{\mathcal{Q}}$ si ottiene:
\begin{equation}
\begin{split}
\Delta W_{\mathcal{Q}} &= \frac{1}{4} \frac{\partial^2 \varphi}{\partial z^2} \int \rho_c^N \left( 3 z^2 - r^2 \cos^2 \theta - r^2 \sin^2 \theta \right) dv =\\
&= \frac{1}{4} \frac{\partial^2 \varphi}{\partial z^2} \left[ \int \rho_c^N \left( 3 z'^2 - r^2 \right) \cos^2 \theta dv + \int \rho_c^N \left( 3 x'^2 - r^2 \right) \sin^2 \theta dv + \int \rho_c^N 6 x'z' \sin \theta \cos \theta dv \right] =\\
&= \frac{1}{4} \frac{\partial^2 \varphi}{\partial z^2} \left[ \int \rho_c^N \left( 3 z'^2 - r^2 \right) \cos^2 \theta dv + \int \rho_c^N \left( 3 x'^2 - r^2 \right) \sin^2 \theta dv \right] =\\
&= \frac{1}{4} \frac{\partial^2 \varphi}{\partial z^2} \left[ \mathcal{Q}_{z' z'} \cos^2 \theta + \mathcal{Q}_{x' x'} \sin^2 \theta \right] = \frac{1}{4} \frac{\partial^2 \varphi}{\partial z^2} \left[ \mathcal{Q} \cos^2 \theta - \frac{1}{2} \mathcal{Q}(1 - \cos^2 \theta) \right]\\
\\
\Delta W_{\mathcal{Q}} &= \frac{1}{4} \frac{\partial^2 \varphi}{\partial z^2} \mathcal{Q} \left( \frac{3}{2} \cos^2 \theta - \frac{1}{2} \right) 
\end{split}
\end{equation}
(nella seconda equazione, l'integrale $\int \rho_c^N 6 x'z' \sin \theta \cos
\theta dv$ si annulla per la simmetria di $\rho_c^N$)\footnote{Negli appunti
  originali, o almeno la copia che la copisteria ha, manca una pagina in questo
punto. Si passa da pagina 73 a pagina 75. [NdT]}.

\subsection{Espressione quantistica del momento di quadrupolo elettrico}
Abbiamo \marginnote{28-01-1998} visto che il contributo di energia
elettrostatica dovuto al quadrupolo elettrico nucleare è dato dalle equazioni:
\begin{empheq}[box=\fbox]{equation}
\begin{split}
\mathcal{Q} &= \mathcal{Q}_{z' z'} = \int \rho_c^N \left( 3 z^2 - r^2 \right) dv\\
\Delta W_{\mathcal{Q}} &= \frac{1}{4} \frac{\partial^2 \varphi}{\partial z^2} \mathcal{Q} \left( \frac{3}{2} \cos^2 \theta - \frac{1}{2} \right)
\end{split}
\end{empheq}
Queste sono state ricavate da considerazioni classiche. In termini quantistici si ha:
\begin{equation}
\Delta W_{\mathcal{Q}} = \left\langle \frac{1}{4} \frac{\partial^2 \varphi}{\partial z^2} \mathcal{Q} \left( \frac{3}{2} \cos^2 \theta - \frac{1}{2} \right) \right\rangle  = \frac{1}{4} \frac{\partial^2 \varphi}{\partial z^2} \mathcal{Q} \left( \frac{3}{2} \left\langle \cos^2 \theta \right\rangle  - \frac{1}{2} \right)
\end{equation}
dove $\cos \theta$ è un operatore definito in maniera tale che:
\begin{equation}
\begin{split}
\left\langle \cos \theta \right\rangle &= \frac{1}{\hbar^2 i j} \left\langle \vec{I} \cdot \vec{J} \right\rangle\\
\left\langle \vec{I} \right\rangle  &= \left( \left\langle \vec{I}_x \right\rangle, \left\langle \vec{I}_y \right\rangle, \left\langle \vec{I}_z \right\rangle \right) = \left( 0, 0, \hbar i \right) \qquad \rightarrow \qquad \text{S.R. nucleare}\\
\left\langle \vec{J} \right\rangle  &= \left( \left\langle \vec{J}_x \right\rangle, \left\langle \vec{J}_y \right\rangle, \left\langle \vec{J}_z \right\rangle \right) = \left( 0, 0, \hbar j \right) \qquad \rightarrow \qquad \text{S.R. atomico}\\
\end{split}
\end{equation}
L'operatore $\vec{I} \cdot \vec{J}$ si può scrivere come:
\begin{equation}
\vec{I} \cdot \vec{J} = \frac{1}{2} \left[ \vec{F}^2 - \vec{I}^2 - \vec{J}^2 \right] \qquad \qquad (\vec{F} = \vec{I} + \vec{J})
\end{equation}
quindi il suo valore di aspettazione è:
\begin{equation}
\left\langle \vec{I} \cdot \vec{J} \right\rangle = \frac{\hbar}{2} K_f \qquad \qquad K_f = f(f+1) - i(i+1) - j(j+1)
\end{equation}
Per il valore di aspettazione di $\cos \theta$ si ottiene dunque:
\begin{equation}
\left\langle \cos \theta \right\rangle = \frac{K_f}{2ij}
\end{equation}
questo risultato però non ci consente di valutare in modo rigoroso la quantità
$\left\langle \cos^2 \theta \right\rangle $ in quanto:
\begin{equation}
\left\langle \left( \vec{I} \cdot \vec{J} \right)^2  \right\rangle  \ne \left( \left\langle \vec{I} \cdot \vec{J} \right\rangle^2 \right) \qquad \Rightarrow \qquad \left\langle \cos^2 \theta \right\rangle \ne \left( \left\langle \cos \theta \right\rangle \right)^2
\end{equation}
questo è analogo al fatto che $\left\langle \vec{J}^2 \right\rangle \ne \left(
\left\langle \vec{J} \right\rangle \right)^2$. Però si sa\footnote{Si sa? Tu lo
sai? E chi lo sa? [cit.]} che:
\begin{equation}
\left\langle \vec{J}^2 \right\rangle \simeq \left( \left\langle \vec{J} \right\rangle \right)^2 \qquad \text{quando}\ j \gg 1
\end{equation}
analogamente avviene che:
\begin{equation}
\left\langle \cos^2 \theta \right\rangle \simeq \left( \left\langle \cos \theta \right\rangle \right)^2 = \frac{K_f^2}{4(ij)^2}
\end{equation}
Questa approssimazione ci consente di scrivere la seguente formula
semi-classica:
\begin{empheq}[box=\fbox]{equation}
\Delta W_{\mathcal{Q}} = \frac{1}{4} \frac{\partial^2 \varphi}{\partial z^2} \mathcal{Q} \left( \frac{3}{2} K_f^2 - 2i^2j^2 \right) \frac{1}{4i^2j^2}
\end{empheq}
La corrispondente espressione quantistica che non verrà però dimostrata è:
\begin{empheq}[box=\fbox]{equation}
\Delta W_{\mathcal{Q}} = \frac{1}{4} \frac{\partial^2 \varphi}{\partial z^2} \mathcal{Q} \left( \frac{3}{2} K_f(K_f+1) - 2i(i+1)j(j+1) \right) \frac{1}{i(2i-1)j(2j-1)}
\end{empheq}
si vede subito che questa tende a quella semi-classica nei limiti suddetti. In
entrambe le formule compare il numero quantico $f$ che può assumere valori:
\begin{equation}
f = i+j, i+j-1, i+j-2, ..., \left| i-j \right| 
\end{equation}
e a diversi valori di $f$ corrispondono diversi valori di $\Delta W_{\mathcal{Q}}$:
\begin{equation}
\Delta W_{\mathcal{Q}} = \left( \Delta W_{\mathcal{Q}} \right)_f 
\end{equation}
Quando il nucleo possiede un $\mathcal{Q} \ne 0$ ciascun livello della struttura
iperfine risulterà spostato della quantità $\left( \Delta W_{\mathcal{Q}}
\right)_f$. Cioè:
\begin{equation}
W_f \rightarrow W_f + \left( \Delta W_{\mathcal{Q}} \right)_f
\end{equation}
Questo provoca una piccola deviazione dalla regola degli intervalli. La misura
di questa deviazione permette di determinare il valore di $\mathcal{Q}$ una
volta che è noto il valore di $\frac{\partial^2 \varphi}{\partial z^2}$
nell'origine.

\subsection{Relazione tra $\mathcal{Q}$ e forma del nucleo}
L'esistenza di $\mathcal{Q}$ ha una stretta connessione con la forma del nucleo.
In termini classici si può scrivere:
\begin{equation}
\begin{split}
\rho_c{(\vec{r})}^N &= Z e \rho{(\vec{r})}^N\\
\mathcal{Q} &= Z e \int \rho{(\vec{r})}^N (3z'^2 - r^2) dv = Z e (3 \left\langle z'^2 \right\rangle - \left\langle r^2 \right\rangle )
\end{split}
\end{equation}
dove si ha:
\begin{equation}
\left\langle r^2 \right\rangle  = \left\langle x'^2 \right\rangle + \left\langle y'^2 \right\rangle + \left\langle z'^2 \right\rangle \qquad \text{e} \qquad \left\langle x'^2 \right\rangle = \left\langle y'^2 \right\rangle
\end{equation}
La stessa connessione tra $\mathcal{Q}$ e la forma del nucleo può essere
ricavata in termini quantistici:
\begin{equation}
\rho_c{(\vec{r})}^N = \sum_{k=1}^A e_k P_k{(\vec{r})}
\end{equation}
dove $P_k{(\vec{r})}$ è la densità di probabilità di trovare il nucleone k-esimo nel punto $\vec{r}$.

Indichiamo con $\psi{(\vec{r}_1, \vec{r}_2, ..., \vec{r}_A)}$ la funzione che
descrive lo stato quantico del nucleo. Questo stato è caratterizzato dal fatto
che $I_z$ è massima $(=i\hbar)$. Si avrà quindi che:
\begin{equation}
P_k{(\vec{r})} = P_k{(\vec{r}_k)} = \int \left| \psi{(\vec{r}_1, \vec{r}_2, ..., \vec{r}_A)} \right| ^2 d^3 \vec{r}_1, ..., d^3 \vec{r}_{k-1}, d^3 \vec{r}_{k+1}, ..., d^3 \vec{r}_A
\end{equation}
Se utilizziamo questa equazione per scrivere $\rho_c$, per $\mathcal{Q}$ si
avrà:
\begin{equation}
\begin{split}
\mathcal{Q} &= \int \rho_c^N \left( 3 z'^2 - r^2 \right) dv\\
&= \sum_{k=1}^A e_k \int P_k{(\vec{r})} \left( 3 z'^2 - r^2 \right) dv\\
&= e \sum_{k=1}^Z \int P_k{(\vec{r})} \left( 3 z'^2_k - r_k^2 \right) d^3\vec{r}_k\\
&= e \sum_{k=1}^Z \int \left| \psi{(\vec{r}_1, \vec{r}_2, ..., \vec{r}_A)} \right| ^2 \left( 3 z'^2_k - r_k^2 \right) d^3 \vec{r}_1, ..., d^3 \vec{r}_A
\end{split}
\end{equation}
questi contributi sono tutti uguali in quanto trattandosi di particelle
identiche, il modulo quadro di $\psi$ non cambia per lo scambio di due indici,
dunque si può scrivere:
\begin{empheq}[box=\fbox]{equation}
\mathcal{Q} = Z e \int \left| \psi{(\vec{r}_1, \vec{r}_2, ..., \vec{r}_A)} \right| ^2 \left( 3 z'^2_k - r_k^2 \right) d^3 \vec{r}_1, ..., d^3 \vec{r}_A
\end{empheq}
il momento di quadrupolo $\mathcal{Q}$ sarà nullo per nuclei con spin zero, in
quanto la $\rho$ avrà simmetria sferica. Esiste una regola empirica secondo cui
hanno spin nullo (e dunque nulli sia il momento di dipolo magnetico che il
momento di quadrupolo elettrico) tutti i nuclei con un numero pari di protoni e
di neutroni.

Si \marginnote{30-01-1998} può dimostrare che $\mathcal{Q} = 0$ anche per tutti
i nuclei che hanno spin $\frac{1}{2}$. Per dimostrarlo basta porre:
\begin{equation}
\begin{split}
z'^2 &= r^2 \cos^2 \theta = r^2 \frac{I_r^2}{\vec{I}^2}\\
I_r &= \vec{r} \cdot \vec{I} = I r \cos \theta
\end{split}
\end{equation}
dove si è preso $\vec{I}$ lungo l'asse $\hat{z}'$. Nel caso di spin $\frac{1}{2}$ si ha:
\begin{equation}
\begin{split}
\left\langle \vec{I}^2 \right\rangle &= \frac{1}{2} \left( \frac{1}{2} + 1 \right) \hbar^2 = \frac{3}{4} \hbar^2\\
\left\langle I_r^2 \right\rangle &= \frac{1}{4} \hbar P_+ + \frac{1}{4} \hbar P_- = \frac{1}{4} \hbar ^2
\end{split}
\end{equation}
dove $P_+$ e $P_-$ sono le probabilità di avere spin $\frac{1}{2}$ e
$-\frac{1}{2}$ lungo la direzione $\vec{r}$. Calcoliamo adesso $\left\langle
\hat{z}'^2 \right\rangle$:
\begin{equation}
\left\langle \hat{z}'^2 \right\rangle  = \frac{4}{3} \left\langle r^2 \right\rangle \frac{1}{4} = \frac{1}{3} \left\langle r^2 \right\rangle = \left\langle r^2 \right\rangle \frac{\left\langle I_r^2 \right\rangle}{\left\langle \vec{I}^2 \right\rangle}
\end{equation}
Il nucleo di spin $\frac{1}{2}$ si può immaginare come un nucleo di spin zero
più un nucleone. Se si ha un nucleo con $\mathcal{Q} \ne 0$ allora certamente
non si ha simmetria sferica. Si possono allora distinguere due casi:
\begin{equation}
\begin{split}
\mathcal{Q} &> 0 \qquad \Rightarrow \qquad \left\langle \hat{z}'^2 \right\rangle > \frac{1}{3} \left\langle r^2 \right\rangle\\
\mathcal{Q} &< 0 \qquad \Rightarrow \qquad \left\langle \hat{z}'^2 \right\rangle < \frac{1}{3} \left\langle r^2 \right\rangle
\end{split}
\end{equation}
Nel primo caso la forma del nucleo è uno sferoide allungato nella direzione di
$\hat{z}'$. Nel secondo caso si ha uno sferoide schiacciato nella direzione di
$\hat{z}'$. $\mathcal{Q}$ si misura in unità di carica protonica e ha le
dimensioni di un'area. L'ordine di grandezza che si trova è:
\begin{equation}
\left| \mathcal{Q} \right| \sim 10^{-24}\ \text{cm}^2
\end{equation}
Si può dimostrare che valori di questo tipo sono associati a rapporti fra
semiasse maggiore e minore del tipo:
\begin{equation}
\frac{d_\text{max}}{d_\text{min}} \sim 1 + 0.1 
\end{equation}
quindi in realtà si ha una distorsione della simmetria sferica estremamente piccola.

\chapter{Modelli Nucleari}
\section{Modello a goccia di liquido}
\subsection{Energia di legame di volume}
Lo studio teorico di nuclei con $A > 2$ è molto complicato per due motivi:
Il primo è dovuto al fatto che non si hanno sufficienti conoscenze sulle forze
che i nucleoni esercitano tra loro, il secondo è dovuto al fatto che dal punto
di vista matematico non si è in grado di risolvere l'equazione di
Schr\"{o}dinger neppure ricorrendo ad approssimazioni.

L'unica strada percorribile è quindi quella di introdurre due modelli
semi-empirici, in più si considereranno diversi modelli a seconda del numero di
massa e delle proprietà che si vogliono evidenziare. Questi diversi modelli non
sono indipendenti l'uno dall'altro.

Considereremo adesso il \textit{modello a goccia di liquido}.
Questo si basa su aspetti prettamente classici e si applica a nuclei con un
valore di $A$ sufficientemente grande in modo da poter trascurare l'identità di
ogni singolo nucleone. L'ipotesi fondamentale di questo modello è
l'incompressibilità del nucleo. Questa ipotesi è suggerita dal fatto che la
densità non dipende da $A$.

Un nucleo complesso si può immaginare come un insieme di sferette rigide tenute
insieme da una forza di contatto, che si esercita solo a distanze estremamente
piccole. All'interno del nucleo, per ragioni prettamente geometriche, ogni
nucleone è a contatto con altri $12$ nucleoni. Se si vuole esprimere la massa
$M$ del nucleo in funzione di $A$ e di $Z$ si utilizza la seguente
considerazione geometrica.

Al primo livello di approssimazione:
\begin{equation}
M_0{(A,Z)} = Z m_p + (A-Z) m_n
\end{equation}
A questo termine bisogna sottrarre la quantità di massa equivalente all'energia
di legame dei nucleoni. Sia $U$ l'energia di legame fra due nucleoni, il difetto
di massa di un nucleone dovuto ad un solo legame è:
\begin{equation}
\delta m_i{(j)} = \delta m_j{(i)} = \frac{1}{c^2} \frac{U}{2} \qquad \qquad \text{i, j = nucleoni}
\end{equation}
quindi ogni nucleone avrà un difetto di massa totale pari a:
\begin{equation}
\delta m_i = \sum_{j = 1}^{12} \delta m_i{(j)} = \frac{12}{c^2} \frac{U}{2} = 6 \frac{U}{c^2}
\end{equation}
questo è il difetto di massa di un nucleone circondato da $12$ nucleoni. Se si
trascura il fatto che non tutti i nucleoni sono interni si può sottrarre a $M_0$
il termine:
\begin{equation}
M_1 = \sum_{i = 1}^A \delta m_i = A \frac{6U}{c^2}
\end{equation}
e a questo grado di approssimazione si ha:
\begin{equation}
M{(A, Z)} = M_0{(A, Z)} - M_1{(A)}
\end{equation}
La quantità $M_{1}c^2$ è chiamata \textit{energia di legame di volume}.
\begin{empheq}[box=\fbox]{equation}
E_\text{vol} = M_1 c^2 = 6 U A
\end{empheq}
Vedere \pageref{allegato_1}\footnote{Questi allegati erano presenti negli
  appunti originali nello stesso ordine in cui sono presenti qui. Venivano dati
direttamente dal professore durante le lezioni.}.
\subsection{Energia di superficie}
Per migliorare le approssimazioni si possono considerare i nucleoni
superficiali, il cui difetto di massa deve essere ridotto di:
\begin{equation}
\delta m_\text{sup} = \frac{3}{c^2} \frac{U}{2}
\end{equation}
(poichè i nucleoni di superficie sono a contatto con $9$ nucleoni invece di $12$).

Per calcolare il numero di nucleoni superficiali si deve considerare la densità
di nucleoni. Si ha dunque che:
\begin{equation}
\frac{A}{V_N} = \frac{3A}{4 \pi R^3} = \frac{A}{\frac{4}{3} \pi r_0^3 A} \qquad \qquad R = r_0 A^{\frac{1}{3}}
\end{equation}
Consideriamo uno strato superficiale di spessore $d$, dove $d$ è il diametro di
un nucleone. Il volume di questo strato è:
\begin{equation}
\delta V \simeq 4 \pi R^2 d
\end{equation}
quindi il numero di nucleoni superficiali è:
\begin{equation}
\delta A = \frac{A}{V_N} \delta V = \frac{A}{V_N} 4 \pi R^2 d = \frac{3 d A}{R} = \frac{3 d A^{\frac{2}{3}}}{r_0}
\end{equation}
L'incremento di massa dovuto ai nucleoni superficiali dunque è:
\begin{equation}
M_2 = \frac{3}{c^2} \frac{U}{2} \delta A = \frac{9}{2}
\frac{UdA^{\frac{2}{3}}}{c^2r_0} 
\end{equation}

La quantità $M_2 c^2$ è chiamata \textit{energia di superficie}:
\begin{empheq}[box=\fbox]{equation}
E_\text{sup} = M_2 c^2 = \frac{9}{2} \frac{UdA^{\frac{2}{3}}}{r_0}
\end{empheq}

La massa $M{(A, Z)}$ risulta quindi:
\begin{equation}
M{(A, Z)} = M_0{(A, Z)} - M_1{(A)} + M_2{(A)}
\end{equation}

La corrispondente energia di legame del nucleo è:
\begin{equation}
E_L = E_\text{vol} - E_\text{sup}
\end{equation}

\subsection{Energia coulombiana del nucleo}
$E_L$ \marginnote{02-02-1998} è in realtà più piccola a causa della forza
repulsiva che esercitano tra loto i $Z$ protoni. Supponiamo che il nucleo abbia
forma sferica con raggio $R$ e che i protoni siano distribuiti in maniera
uniforme. Sotto queste ipotesi si ha:
\begin{equation}
\rho_c = Z e \frac{3}{4 \pi R^3} \qquad \qquad R = r_0 A^{\frac{1}{3}}
\end{equation}
Consideriamo una sfera concentrica al nucleo ma con raggio $r < R$. La carica
contenuta in questa sfera è:
\begin{equation}
Q{(r)} = \frac{4}{3} \pi r^3 \rho_c
\end{equation}
con il teorema di Gauss è possibile calcolare il campo elettrico ed il
potenziale presenti sulla superficie della sfera di raggio $r$. Se aggiungiamo
alla sfera di raggio $r$ uno strato di spessore $dr$ si ha un incremento di
energia pari a:
\begin{equation}
dE_c{(r)} = Q{(r)} \frac{dQ}{r} = \frac{16}{3} \pi^2 \rho_c^2 r^4 dr
\end{equation}
ed integrando tra $0$ e $R$ si ottiene l'\textit{energia totale coulombiana}:
\begin{empheq}[box=\fbox]{equation}
E_c = \int_0^R dE_c{(r)} = \frac{16}{15} \pi^2 \rho_c^2 R^5 = \frac{3}{5}
\frac{Z^2 e^2}{r_0 A^{\frac{1}{3}}}
\end{empheq}
A questo termine di energia corrisponde un incremento di massa:
\begin{equation}
M_3 = \frac{3}{5} \frac{Z^2 e^2}{c^2 r_0 A^{\frac{1}{3}}}
\end{equation}
e a questo successivo livello di approssimazione la massa del nucleo diventa:
\begin{equation}
M{(A, Z)} = M_0{(A, Z)} - M_1{(A)} + M_2{(A)} + M_3{(A, Z)}
\end{equation}
e l'energia di legame si può scrivere come:
\begin{equation}
E_L = E_\text{vol} - E_\text{sup} - E_c
\end{equation}

\subsection{Termine energetico di simmetria}
$E_c$ diminuisce rapidamente al diminuire di $Z$. Dal punto di vista
sperimentale si nota che i nuclei stabili contengono un numero di neutroni $N$
maggiore di $Z$. Al diminuire di $Z$ si osserva che $N$ tende a $Z$. Più
precisamente\footnote{Eh... più preciso di così si muore. [NdT]}:
\begin{equation}
A < 36\ \Rightarrow\ N \simeq Z \qquad \qquad A > 36\ \Rightarrow\ N > Z
\end{equation}
Questa tendenza dei nuclei stabili porta a dover fare due considerazioni:

L'eccesso di neutroni nei nuclei con $A > 36$ si può ricollegare alla repulsione
elettrostatica. L'aumento del numero di neutroni riduce infatti il valore di
$E_L$ a parità di $A$. La maggiore stabilità di un nucleo implica un minimo di
energia, motivo per cui i nuclei tendono ad aumentare il numero di neutroni al
crescere di $A$.

In assenza di repulsione elettrostatica, l'energia di un nucleo risulterebbe
minima per $N = Z = \frac{A}{2}$. Questo spiega il motivo per cui l'eccesso di
neutroni nei nuclei stabili è soggetto a saturazione. Se in un nucleo stabile
con un eccesso di neutroni si sostituissero alcuni protoni con ulteriori
neutroni (mantenendo $A$ fisso) l'energia non diminuirebbe ma tenderebbe
piuttosto ad aumentare (al crescere di $N$, $E_c$ diminuisce ma $E_L$ aumenta).
Si può dimostrare che se si ha uno stato legato con due tipi di fermioni,
l'energia è minima quando viene minimizzato il numero complessivo di coppie di
fermioni identici. Questo perchè i fermioni identici interagiscono tramite forze
di scambio che tendono ad aumentarne l'energia.

Per tenere conto di questa ulteriore diminuizione di $E_L$, a seconda della
quantità di neutroni presenti bisogna considerare il \textit{termine di
simmetria} aggiuntivo:
\begin{empheq}[box=\fbox]{equation}
E_\text{simm} = a_\text{simm}\ A \left[ \frac{\left( \frac{A}{2} \right) - Z}{A} \right]^2 c^2
\end{empheq}
è questa l'energia che aumenta più di quanto diminuisca $E_c$ (al crescere di
$N$). Il termine raccolto entro le parentesi rappresenta la percentuale di
neutroni in eccesso. Questo termine può essere considerato classicamente (a
parità di percentuale deve essere proporzionale ad $A$, mentre a parità di $A$
deve aumentare con il crescere della percentuale di neutroni in eccesso). La
proporzionalità ad $A$ di questo termine è legata all'incomprimibilità del
nucleo in quanto la densità di energia $\frac{E_\text{simm}}{V}$ deve risultare
costante. Il corrispondente incremento di massa è:
\begin{equation}
M_4{(A, Z)} = a_\text{simm}\ A \left[ \frac{\left( \frac{A}{2} \right) - Z}{A} \right]^2 = a_\text{simm}\ \frac{\left[ \left( \frac{A}{2} - Z \right)  \right]^2}{A}
\end{equation}
e le correzioni:
\begin{equation}
\begin{split}
M{(A, Z)} &= M_0{(A, Z)} - M_1{(A)} + M_2{(A)} + M_3{(A, Z)} + M_4{(A, Z)}\\
E_L &= E_\text{vol} - E_\text{sup} - E_c - E_\text{simm}
\end{split}
\end{equation}

\subsection{Termini correttivi sperimentali e formula di Weizs\"{a}cker}
Vi è un'ultima correzione da fare legata alla verifica sperimentale che vi è una
maggiore abbondanza di nuclei stabili con un numero di protoni e neutroni
entrambi pari piuttosto che quando sono entrambi dispari (per valori di $A$
fissati). Questa condizione riguarda soltanto i nuclei con valori di $A$ pari.

Il termine di massa correttivo si ricava sperimentalmente, ed esso è pari a:
\begin{equation}
M_S = \begin{cases}
\frac{\delta}{\sqrt{A}} = -\frac{K}{\sqrt{A}} \qquad \qquad &\text{$A$ pari, $Z$ pari}\\
\\
\frac{\delta}{\sqrt{A}} = 0 \qquad \qquad &\text{$A$ pari, $Z$ dispari}\\
\\
\frac{\delta}{\sqrt{A}} = +\frac{K}{\sqrt{A}} \qquad \qquad &\text{$A$ dispari, $Z$ dispari}
\end{cases}
\end{equation}
Mettendo insieme tutti i risultati ottenuti si conclude che nel modello a goccia
di liquido la massa può essere calcolata tramite la formula semi-empirica,
chiamata \textit{formula di Weizs\"{a}cker}:
\begin{equation}
\begin{split}
M{(A, Z)} &= M_0{(A, Z)} - M_1{(A)} + M_2{(A)} + M_3{(A, Z)} + M_4{(A, Z)} + M_5{(A)} =\\
&= \left[ m_p Z + m_N(A-Z) \right] - a_\text{vol}\ A + a_\text{sup}\ A^{\frac{2}{3}} + \frac{3}{5} a_c\ \frac{Z^2}{A^{\frac{1}{3}}} + a_\text{simm}\ \frac{\left[ \left( \frac{A}{2} - Z \right)  \right]^2}{A} + \frac{\delta}{A^{\frac{1}{2}}}
\end{split}
\end{equation}
I coefficienti $a$ non possono essere ricavati da alcuna teoria generale che
mostri accordo con i valori sperimentali (!). Dei valori frequentamente usati
sono, ad esempio:
\begin{equation}
a_\text{vol} = 15.67 \qquad a_\text{sup} = 17.23 \qquad \frac{3}{5} a_c = 0.70 \qquad a_\text{simm} = 93.15 \qquad \delta = 1.2
\end{equation}
Con questi valori si ottiene in genere una precisione media di $2$ MeV/$C^2$ che
è un ottimo risultato. Il limite di questo modello è quello di ottenere però
soltanto risultati medi. I primi tre termini $M_1$, $M_2$ e $M_3$ hanno un
analogo per la formula di una goccia di liquido.

Si può ricavare il valore di $Z$ per cui si ha il minimo della massa per una
serie di isobari con un dato $A$, utilizzando la condizione $\frac{\partial
M}{\partial Z}$  = 0. Si arriva alla formula:
\begin{equation}
\frac{6}{5} a_c \frac{Z}{A^{\frac{1}{3}}} - 2 \frac{a_\text{simm}}{A} \left( \frac{A}{2} - Z \right) - \left(m_N - m_P \right) = 0
\end{equation}
Questa equazione riproduce la cosiddetta \textit{curva di Segrè} che restituisce
la funzione $Z - N$. (vedere \pageref{allegato_2})
\subsection{Energia di fissione}
Un' \marginnote{04-02-1998} altra applicazione della formula delle masse
nucleari è il calcolo dell'energia in un processo di fissione. In un processo
del genere i frammenti di nucleo che si producono sono altamente instabili e
quindi non se ne può misurare direttamente la massa. Possiamo schematizzare il
processo di fissione considerando un nucleo a riposo di massa $M{(A, Z)}$ che si
scinde in due frammenti uguali di massa $M{(\frac{A}{2}, \frac{Z}{2})}$, cioè:
\begin{equation}
M{(A, Z)} \qquad \rightarrow \qquad M{(\frac{A}{2}, \frac{Z}{2})} + M{(\frac{A}{2}, \frac{Z}{2})}
\end{equation}
l'energia associata a questo processo è definita \textit{energia di fissione} e
si può scrivere come:
\begin{empheq}[box=\fbox]{equation}
E_F = \left[ M{(A, Z)} - 2  M{(\frac{A}{2}, \frac{Z}{2})} \right] c^2 
\end{empheq}
Dall'espressione di $M{(A, Z)}$ si trova che $E_F$ risulta positiva per $A \ge
90$. Questo implica che nuclei con $A > 90$ non dovrebbero risultare stabili.
Consideriamo infatti un nucleo di Uranio $^{236}U$, l'energia di fissione è $E_F
\simeq 184$ MeV.

Il modello a goccia di liquido riesce a spiegare anche perchè molti di questi
nuclei pesanti non subiscono una fissione spontanea ma sono sensibili soltanto a
fissione indotta (bombardamento). Secondo questo modello infatti, un nucleo che
subisce una fissione equivale ad una goccia di liquido che viene forzata a
scindersi dal resto delle molecole di liquido. Questa analogia si spiega
considerando un nucleo soggetto ad una piccola deformazione, tale che la forma
non sia più sferica ma si abbia un $R'{(\theta)} = Rf{(\theta)}$ dove $R$
rappresenta il raggio iniziale. Espandendo la funzione $f{(\theta)}$ in serie di
polinomi di Legendre:
\begin{equation}
f{(\theta)} = 1 + \alpha_1 P_1 (\cos \theta) + \alpha_2 P_2 (\cos \theta) + \alpha_3 P_3 (\cos \theta) + ...
\end{equation}
dove si ha:
\begin{equation}
\begin{split}
&P_1 (\cos \theta) = \cos \theta\\
&P_2 (\cos \theta) = \frac{1}{2} \left( 3\cos^2 \theta - 1 \right)\\ 
&P_3 (\cos \theta) = \frac{1}{2} \left( 5\cos^3 \theta - 3 \cos \theta \right)\\
&...
\end{split}
\end{equation}
e supponiamo che la deformazione mantenga la simmetria per inversione spaziale,
cioè deve valere:
\begin{equation}
f{(\pi - \theta)} = f{(\theta)}
\end{equation}
Sotto queste ipotesi si ha che $\alpha_1 = \alpha_3 = \alpha_{2n + 1} = 0$ e la
deformazione più semplice con queste caratteristiche è:
\begin{equation}
f{(\theta)} = 1 + \alpha_2 P_2 (\cos \theta)
\end{equation}
La condizione di incompressibilità richiede che il volume nucleare debba
rimanere invariato. Quindi questa deformazione può cambiare soltanto l'energia
di superficie ed il termine coulombiano, in quanto non può influenzare l'energia
di volume e quella di simmetria i cui valori dipendono soltanto dal volume
complessivo. Se si trascurano i termini di ordine superiore a $\alpha_2^2$ si
ottiene  che:
\begin{empheq}[box=\fbox]{equation}
\delta E = \delta E_\text{sup} + \delta E_{c} = \left( 2E_\text{sup} - E_c \right) \frac{\alpha_2^2}{5}
\end{empheq}
dove $E_\text{sup}$ e $E_c$ sono i valori di energia in assenza di distorsione.
Questo termine $\delta E$  si può interpretare come un'energia potenziale di
distorsione. In particolare si ha che:
\begin{equation}
\begin{split}
\delta E &> 0 \qquad \qquad \text{se}\ 2 E_\text{sup} > E_c\\
\delta E &< 0 \qquad \qquad \text{se}\ 2 E_\text{sup} < E_c
\end{split}
\end{equation}
Nel primo caso il nucleo risulterà stabile per piccole deformazioni, nel secondo
caso il nucleo sarà completamente instabile ed esso continuerà a deformarsi
sempre di più fino a frammentarsi. La situazione di confine tra questi due casi
si ha quando $2E_\text{sup} = E_c$, questa condizione implica che:
\begin{equation}
2a_\text{sup}\ A^{\frac{2}{3}}= \frac{3}{5} a_c\ \frac{Z^2}{A^{\frac{1}{3}}}
\end{equation}
e questa condizione si verifica quando:
\begin{equation}
\frac{Z^2}{A} = \left( \frac{Z^2}{A} \right)_\text{limite} = \frac{2 a_\text{sup}}{\frac{3}{5}a_c} \simeq 49
\end{equation}
quindi se $\frac{Z^2}{A} < 49$ il nucleo risulterà stabile per piccole
deformazioni e quindi non potrà subire processi di fissione spontanea. Si può
verificare che la dipendenza di $\delta E$ dal fattore $2E_\text{sup} - E_c$ non
dipende dalla particolare deformazione considerata:
\begin{equation}
\eta\ =\ \frac{\frac{Z^2}{A}}{\left( \frac{Z^2}{A} \right)_\text{limite}} \qquad \Rightarrow \qquad \begin{cases}
\eta < 1 \qquad &\text{Non soggetto a fissione spontanea}\\
\eta > 1 \qquad &\text{Soggetto a fissione spontanea}
\end{cases}
\end{equation}
Si vede che quasi tutti i nuclei fissi non superano il valore $\eta = 1$.
Ad esempio, per il nucleo di $^{236}U$ si ha:
\begin{equation}
\frac{Z^2}{A} \simeq 36 \qquad \qquad \eta \simeq 0.73
\end{equation}
Nel caso generale di fissione non spontanea possiamo considerare i due frammenti
e fare un grafico dell'energia dei due frammenti (a riposo) in funzione della
distanza dai loro centri. Si ottiene il seguente andamento:
\begin{figure}[hbtp]
\centering
\caption{Grafico dell'energia.}
\label{fig:energy}
\begin{tikzpicture}[line cap=round,line
  join=round,>=stealth,x=1.0cm,y=1.0cm,scale=2]
  \clip(18.27,-2.44) rectangle (25.13,1.62);
  \draw [->] (18.84,-1.85) -- (18.84,1.49);
  \draw [->] (18.84,-1.85) -- (24.97,-1.85);
  \draw[smooth,samples=100,domain=18.84:24.31]
  plot(\x,{sin((1.2*(\x))*180/pi)+sin(((\x))*180/pi)});
  \draw (18.49,-0.26) node[anchor=north west] {$ Mc^2 $};
  \draw [dash pattern=on 1pt off 1pt] (21.55,1.09)-- (21.55,-1.85);
  \draw [dash pattern=on 1pt off 1pt] (18.84,1.09)-- (21.55,1.09);
  \draw (18.64,1.69) node[anchor=north west] {$E$};
  \draw (18.44,1.14) node[anchor=north west] {$E_\text{max}$};
  \draw (21.23,-1.87) node[anchor=north west] {$R_1 + R_2$};
  \draw [dash pattern=on 1pt off 1pt] (18.84,-1.62)-- (24.58,-1.62);
  \draw (18.86,-1.26) node[anchor=north west] {$c^2(M_1 + M_2)$};
  \draw (24.78,-1.84) node[anchor=north west] {$R$};
  \begin{scriptsize}
	\fill [color=\MainColor] (18.84,-0.59) circle (1.5pt);
	\fill [color=\MainColor] (21.55,1.09) circle (1.5pt);
	\fill [color=\MainColor] (21.55,-1.85) circle (1.5pt);
	\fill [color=\MainColor] (18.84,1.09) circle (1.5pt);
	\fill [color=\MainColor] (18.84,-1.62) circle (1.5pt);
  \end{scriptsize}
\end{tikzpicture}

\end{figure}
$M$ è la massa del nucleo fissile, $M_1$, $M_2$, $R_1$ e $R_2$ sono le masse e i
raggi finali dei due frammenti. L'andamento per $r > R_1 + R_2$ è una iperbole
che è dovuta soltanto all'energia potenziale coulombiana tra i due frammenti.
(vedere \pageref{allegato_3}, \pageref{allegato_41} e \pageref{allegato_42})
\subsection{Energia di attivazione}
Introduciamo l'\textit{energia di attivazione} definita come:
\begin{empheq}[box=\fbox]{equation}
\Delta E = E_{max} - Mc^2
\end{empheq}
questa è l'energia di soglia che è necessario fornire dall'esterno affinchè
avvenga la fissione. Per calcolare $\Delta E$ il modello a goccia di liquido
risulta ancora utile. (vedere \pageref{allegato_5})

\section{Modello a strati}
\subsection{Numeri magici}
Il \marginnote{06-02-1998} modello a goccia di liquido non va bene per nuclei
con $A \le 10$. Questo perchè il modello ha una impostazione di carattere
statistico. In più, per certi valori di $A$ e di $Z$ si osservano
sperimentalmente grosse discontinuità mentre il modello studiato ha un andamento
pressochè continuo. Queste discontinuità si riscontrano quando $Z$ e $A - Z$
hanno valori vicini ai numeri:
\begin{equation}
2,\ 8,\ 20,\ 28,\ 50,\ 82,\ 126
\end{equation}
Questi vengono detti \textit{numeri magici} e i nuclei che hanno un numero di
protoni o neutroni pari a uno di questi numeri si dicono \textit{nuclei magici}
(doppiamente magici quelli con $Z$ e $A - Z$ entrambi magici).

Nei nuclei immediatamente successivi a quelli magici l'$n+1$-esimo nucleone è
molto poco legato al nucleo. Supponiamo che sia $E_L{(N, Z)}$ l'energia di
legame di un nucleo. Se vi è un ulteriore neutrone o protone si osserva che
l'energia con cui essi sono legati è:
\begin{equation}
\begin{split}
U_N &= E_L{(N + 1, Z)} - E_L{(N, Z)}\\
U_Z &= E_L{(N, Z + 1)} - E_L{(N, Z)}
\end{split}
\end{equation}
nei nuclei magici questa energia è notevolmente minore rispetto agli altri
nuclei. L'energia media di legame tra due nucleoni è:
\begin{equation}
U = \frac{15.67}{6}\ \text{MeV} \simeq 2.6\ \text{MeV}
\end{equation}
mentre nei nuclei magici si osserva una diminuizione di circa $2$ MeV. Il più
grande esempio di discontinuità è il nucleo doppiamente magico di $He$, infatti
si ha:
\begin{equation}
\begin{split}
U_N &= E_L{(2, 2)} - E_L{(1, 2)} \simeq 20\ \text{MeV}\\
U_Z &= E_L{(3, 2)} - E_L{(2,2)} \simeq 7\ \text{MeV}
\end{split}
\end{equation}
Quanto più grande è il numero magico, tanto minore è la discontinuità.
Ovviamente queste considerazioni rimangono valide anche per la massa dei nuclei.

Un caso di notevole interesse si osserva per il numero magico di $126$ neutroni.
Indichiamo il nucleo con $\mathcal{N}{(126, Z)}$ e ne consideriamo il
decadimento, cioè:
\begin{equation}
\mathcal{N}{(126, Z)} \qquad \rightarrow \qquad \mathcal{N}{(124, Z-2)} + \alpha
\end{equation}
e facciamo un confronto con il decadimento $\alpha$ del nucleo $\mathcal{N}{(128, Z')}$
\begin{equation}
\mathcal{N}{(128, Z')} \qquad \rightarrow \qquad \mathcal{N}{(126, Z'-2)} + \alpha
\end{equation}
Si trova sperimentalmente che i due neutroni espulsi nel secondo decadimento
hanno una energia cinetica molto più grande di quelli espulsi nel primo
decadimento. Quindi il $127$° e il $128$° neutrone hanno un'energia di legame
molto più bassa. Da tutto questo si deduce che i nuclei magici sono molto più
stabili, fatto riscontrato sperimentalmente.

Una conferma della loro stabilità è dato dalla sezione d'urto di cattura dei
neutroni. La sezione d'urto è infatti molto più piccola rispetto a quella dei
nuclei non magici. Tutte queste proprietà ricordano quelle degli atomi dei gas
nobili, in cui si usa un modello a strati per spiegarne le proprietà.

Si formula quindi un analogo \textit{modello a strati} per i nuclei. La validità
di un tale modello è confermata dal fatto che anche i nuclei possono trovarsi in
stati eccitati dai quali decadono emettendo radiazione $\gamma$. Il modello a
strati si differenzia notevolmente da quello a goccia di liquido, ogni nucleone
occupa infatti un ben preciso orbitale diverso da quello degli altri. Si
riscontrano però delle serie difficoltà nel calcolo del campo centrale, infatti
nei nuclei non si ha alcun campo centrale. Per superare questa difficoltà si
ipotizza che tutti i nucleoni generino una buca di potenziale media entro cui
ciascun nucleone, in prima approssimazione, si muova indipendentemente dagli
altri (il modello si dice anche \textit{Modello a particelle indipendenti}).


\subsection{Hamiltoniana del nucleo e  spin-orbita}
Scriviamo l'Hamiltoniana del nucleo come:
\begin{equation}
\mathcal{H} = \sum_{i=1}^A T_i + \frac{1}{2} \sum_{k=1}^A \sum_{i \ne k} \varphi_{ik}{\left( \vec{r}_{ik}\right)} 
\end{equation}
Sia $\vec{r}_i$ il vettore posizione dell'i-esimo nucleone rispetto al centro del nucleo, allora si può formalmente porre:
\begin{equation}
\frac{1}{2} \sum_{k=1}^A \sum_{i \ne k} \varphi_{ik}{\left( \vec{r}_{ik}\right)} = \sum_{i=1}^A \varphi_0{(\vec{r}_i)} + \left[ \frac{1}{2} \sum_{k=1}^A \sum_{i \ne k} \varphi_{ik}{\left( \vec{r}_{ik}\right)} - \sum_{i=1}^A \varphi_0{(\vec{r}_i)} \right] 
\end{equation}
Per trattare il nucleo quindi come un insieme di particelle indipendenti che si muovono in un campo centrale si deve trovare il potenziale $\varphi_0$ in modo che l'ultimo addendo nella formula di sopra sia trascurabile. $\varphi_0$ è chiamato \textit{potenziale fittizio}. In più $\varphi_0$ deve essere tale da riprodurre la distribuzione di massa che è poi quella che genera $\varphi_0$.

Dal momento che le interazioni sono a corto raggio si può prevedere che l'andamento di $\varphi_0$ in funzione di $r$ deve essere analogo alla dipendenza di $r$ dalla densità di massa nucleare. Seguendo questo criterio e assumendo che la densità di massa abbia simmetria sferica si può porre:
\begin{equation}
\mathcal{H} = \sum_{i=1}^A \mathcal{H}_i = \sum_{i=1}^A \left[ T_i + \varphi_0{(r_i)} \right] 
\end{equation}
dove $\varphi_0(r)$ è adesso un potenziale centrale. Si troveranno così dei livelli di energia con autovalori $E{(n, l)}$ con una certa degenerazione. Opportune scelte di $\varphi_0$ consentono di spiegare alcuni numeri magici. Ad esempio, per $n = 0$ si ha un livello che può contenere al massimo due nucleoni identici, per $n = 1$ e $n = 2$ si possono avere al massimo 6 o 12 nucleoni identici. Così si spiegano i numeri magici 2, 8 e 20.

Questo risultato tuttavia va bene soltanto per i primi nuclei magici, e si ha che per come è stata definita $\mathcal{H}$ non si può più riprodurre tutta la serie dei numeri magici. Si deve introdurre nell'Hamiltoniana di ogni nucleone il termine:
\begin{equation}
\varphi_S{(r)}\ \vec{l} \cdot \vec{s} \qquad \qquad\text{con}\ \varphi_S{(r)} = \frac{1}{2} \frac{\lambda^2}{r} \frac{\partial \varphi_0}{\partial r}
\end{equation}
dove $\lambda$ è una costante con le dimensioni di una lunghezza.
\\
Questo termine è chiamato anche \textit{termine di interazione spin-orbita} in quando è analogo a quello per l'elettrone. Con l'aggiunta di questo termine l'Hamiltoniana totale diventa:
\begin{equation}
\mathcal{H} = \sum_{i=1}^A \mathcal{H}_i = \sum_{i=1}^A \left[ T_i + \varphi_0{(r_i)} + \varphi_S{(r)}\ \vec{l}_i \cdot \vec{s}_i \right] 
\end{equation}
il modello rimane a particelle indipendenti (vedere \pageref{allegato_6}).

\subsection{Interazione di appaiamento}
Per \marginnote{09-02-1998} definizione si ha: 
\begin{equation}
\vec{J} = \vec{L} + \vec{S}
\end{equation}
e l'energia del singolo nucleone dipenderà adesso anche da $J$, cioè:
\begin{equation}
E = E{(n, l, j)}
\end{equation}
il singolo sottolivello si scinde in due, cioè:
\begin{equation}
E{(n,l)} \begin{cases}
E{(n, l, j = l + \frac{1}{2})}\\
E{(n, l, j = l - \frac{1}{2})}
\end{cases}
\end{equation}
ciascuno di questi due sottolivelli può contenere al massimo $2j + 1$ nucleoni identici. Valutiamo le variazioni di energia $\Delta E$:
\begin{equation}
\Delta E = E{(n, l, j)} - E{(n,l)}
\end{equation}
in base alla formula di Landè si può scrivere:
\begin{equation}
\begin{split}
\Delta E &\propto \left\langle \vec{L} \cdot \vec{S} \right\rangle = \left[ j(j+1) - l(l+1) - s(s+1) \right] \frac{1}{2} \hbar^2 =\\
&= \frac{\hbar ^2}{2} \left[  j(j+1) - l(l+1) - \frac{3}{4} \right]
\end{split}
\end{equation}
Da questa espressione si può dedurre che:
\begin{equation}
\Delta E \propto \begin{cases}
-(l-1) \qquad \qquad &(j = l - \frac{1}{2})\\
l \qquad &(j = l + \frac{1}{2})
\end{cases}
\end{equation}
e dalle costanti si ha:
\begin{equation}
\begin{split}
\Delta &E > 0 \qquad \qquad \text{per}\ j = l - \frac{1}{2}\\
\Delta &E < 0 \qquad \qquad \text{per}\ j = l + \frac{1}{2}
\end{split}
\end{equation}
quindi si ha un'energia maggiore quando $j$ è più piccolo.

Gli strati più influenzati sono quelli con $n \ge 3$. Ogni strato si scinde in
due, quello più basso va a cadere nello strato inferiore mentre quello più alto
rimane nel suo strato. Le energie che caratterizzano i vari strati variano al
variare del nucleo. In più differiscono anche all'interno di uno stesso nucleo a
seconda che si parli di protoni o di neutroni (l'energia dei protoni risulterà
un pò più alta per via della repulsione elettrostatica).
Dal modello a strati si possono trarre informazioni sullo spin.
In un sottolivello completo, per il principio di Pauli si ha un momento totale
di spin nullo. Quindi lo spin totale del nucleo è dato solo dai nucleoni
esterni. Questo spiega perchè i nuclei con un numero pari di protoni e neutroni
hanno spin nullo, in quanto $2j+1$ è pari, e quindi se il livello è completo il
numero di nucleoni di quel sottolivello è sicuramente pari.

Vediamo adesso perchè anche se il livello non è completo, ma con un numero pari
di protoni o neutroni, lo spin totale è nullo.
A questo proposito si deve considerare nell'Hamiltoniana il termine di interazione:
\begin{equation}
\frac{1}{2} \sum_{k=1}^A \sum_{i \ne k} \varphi_{ik}{\left( \vec{r}_{ik}\right)} = \sum_{i=1}^A \varphi_0{(\vec{r}_i)} + \left[ \frac{1}{2} \sum_{k=1}^A \sum_{i \ne k} \varphi_{ik}{\left( \vec{r}_{ik}\right)} - \sum_{i=1}^A \varphi_0{(\vec{r}_i)} \right] 
\end{equation}
fino ad ora abbiamo trascurato il termine tra parentesi:
\begin{equation}
\mathcal{H}_\text{res} = \frac{1}{2} \sum_{k=1}^A \sum_{i \ne k} \varphi_{ik}{\left( \vec{r}_{ik}\right)} - \sum_{i=1}^A \varphi_0{(\vec{r}_i)}
\end{equation}
di questa, il contributo principale è dato dall'interazione a due a due dei
momenti angolari totali, cioè dai termini dell'\textit{interazione di
appaiamento}:
\begin{equation}
a_j J_1 J_2
\end{equation}
questo termine implica che i nucleoni non sono più particelle indipendenti,
quindi $J_i$ non si conserva più. Ciascuna coppia di nucleoni interagenti si
trova in un autostato del momento angolare totale con autovalore $0 \le
J_\text{tot} \le 2J$. La configurazione a più bassa energia è quella per $j = 0$
con $a_j > 0$. Si ha che:
\begin{equation}
\left\langle \vec{J}_1 \cdot \vec{J}_2 \right\rangle = \frac{1}{2} \left\langle \vec{J}_\text{tot}^2 - \vec{J}_{1}^2 - \vec{J}_{2}^2 \right\rangle = \frac{1}{2} \hbar^2 \left[ j_\text{tot}(j_\text{tot} + 1) - 2j(j + 1) \right] 
\end{equation}
Da questa si vede che per $a_j > 0$ il minimo di $E$ si ha per $J_\text{tot} = 0$.

I nuclei con un numero pari di protoni e neutroni hanno quindi una massa (e
dunque un'energia) leggermente inferiore rispetto ai corrispondenti nuclei
isobari dispari-dispari. Questo spiega l'origine del termine empirico $M_S$.
Una conseguenza del'interazione di appaiamento è che se in un sottolivello si ha
un numero dispari di nucleoni, il momento angolare totale di spin coincide con
quello di un nucleone. Quindi se un nucleo ha massa dispari, il suo spin è
uguale a quello del nucleone spaiato. Questa previsione è confermata
sperimentalmente ad eccezione di due casi. Da questa regola si traggono
informazioni sul momento magnetico nucleare, ossia riguardo al fatto che nei
nuclei con massa dispari il momento magnetico coincide con quello del nucleone
spaiato.

\section{Modello collettivo}
\subsection{Asimmetria del nucleo}
Il modello a strati funziona bene per nuclei con strati completi o con solo 1 o 2 nucleoni addizionali. Quando gli strati sono completi, lo spin è zero e si ha simmetria centrale, quindi pieno accordo con l'ipotesi di campo centrale. L'aggiunta di qualche nucleone provoca piccole deviazioni dal campo centrale. Quando però un livello è riempito a metà si hanno deviazioni grandi rispetto all'ipotesi di campo centrale. Un modello che tiene conto di questa asimmetria è il \textit{modello collettivo}.
Questo ragionamento evidenzia come il modello a strati sia efficace fintanto che si abbia una certa simmetria sferica.
